
\chapter{The PDEs}\label{PDEs}
\section{De Giorgi-Nash-Moser Estimates}\label{Moser_iteration}
In classical Schauder estimates or $W^{2,p}$ estimates, the continuity of the coefficients is required. Therefore, the methods are not usable in nonlinear case. In 1957, De Giorgi obtained the H\"older estimate on elliptic equations of divergence form with {\it measurable} coefficients. In 1958, Nash independently got the similar estimates on parabolic equations. In 1960, Moser obtained a simplified proof and the Harnark inequality. These methods now becomes  crucial in geometric analysis.



We consider the following elliptic equation of divergence form:
\begin{equation}\label{div}
-D_j(a^{ij}D_iu)+b^iD_i u+cu=0
\end{equation}
such that 
\begin{align*}
& \lambda|\xi|^2\leq a^{ij}\xi_i\xi_j\leq\Lambda|\xi|^2,\qquad \forall x\in\Omega,\,\xi\in\R^n
\\
&\sum_{i,j}||a^{ij}||_{L^\infty}+\sum_i||b^i||_{L^\infty}+||c||_{L^\infty}\leq\Lambda.
\end{align*}
For the sake of simplicity, we only consider the following equation
\begin{equation}\label{div-1}
-D_j(a^{ij}D_iu)=0.
\end{equation}

\begin{lemma} Let $\phi(s)\in C_{loc}^{0,1}(\R)$ be a convex function
\begin{enumerate}
\item If $u$ is~\eqref{div-1}'s weak subsolution, $\phi'(s)\geq 0$. Then $v=\phi(u)$ is also a weak subsolution;
\item If $u$ is~\eqref{div-1}'s weak supersolution, $\phi'(s)\leq 0$. Then $v=\phi(u)$ is also a weak subsolution.
\end{enumerate}
\end{lemma}

\qed

\begin{lemma}[Local maximum principle] Let $v\in H^{1,2}(B_R)$ be a weak subsolution. Then for 
any $p>0$, $0<\theta<1$, we have
\[
\underset{{B_{\theta R}}}{ess\,sup}\, v\leq C\left(\fint_{B_R}(v^+)^p\right)^{1/p},
\]
where $C$ only depends on $n$, $\Lambda/\lambda$, $p$, and $(1-\theta)^{-1}$.
\end{lemma}

{\bf Proof.} The lemma is true even if $v$ is not bounded. However, for the sake of simplicity, we assume that $v$ is essentially bounded.

We first assume that $p\geq 2$. Since $v$ is a weak subsolution, so is $v^+$. Thus we can assume that $v\geq 0$.

For any testing function $\phi\geq 0$, we have
\[
\int_{B_R} a^{ij}D_ivD_j\phi \leq 0.
\]
We let $\xi\in C_0^\infty(B_R)$ and $\phi=\xi^2v^{p-1}$. Then we have
\[
(p-1)\int_{B_R}(a^{ij} D_i v D_j v)v^{p-2}\xi^2 dx
\leq -2\int_{B_R}a^{ij} v^{p-1}\xi D_i v D_j\xi dx.
\]
Thus there is a constant $C$ such that
\[
\int_{B_R}|D(\xi v^{p/2})|^2 dx\leq C\int_{B_R}|D\xi|^2v^p dx.
\]
Using the Sobolev inequality, we have
\[
\left(\int_{B_R}(\xi v^{p/2})^{2^*} dx\right)^{2/2^*}\leq C\int_{B_R}|D\xi|^2 v^p dx,
\]
where 
\[
2^*=\frac{2n}{n-2}.
\]

Let 
\[
R_k=R\left(\theta+\frac{1-\theta}{2^k}\right).
\]
Define
$\xi_k\in C_0^\infty(B_{R_k})$ such that $\xi_k\equiv 1$ on $B_{R_{k+1}}$. Then
\[
|D\xi_k|\leq\frac{2^{k+1}}{(1-\theta) R}.
\]
Using this we have
\[
\left(\int_{B_{R_{k+1}}}v^{\frac{np}{n-2}}dx\right)^{\frac{n-2}{n}}\leq\frac{C\cdot 4^k}{(1-\theta)^2R^2}\int_{B_{R_k}}v^p dx.
\]
Thus we have
\[
||v||_{L^{\frac{np}{n-2}}}(B_{R_{k+1}})\leq (C\cdot 4^k)^{1/p}||v||_{L^p}(B_{R_k}).
\]
We let
\[
p_1=\frac{np}{n-2},\cdots, p_k=\left(\frac{n}{n-2}\right)^k p.
\]
Then we have
\[
||v||_{L^{p_{k+1}}}(B_{R_{k+1}})\leq (C\cdot 4^k)^{1/p_k}||v||_{L^{p_k}}(B_{R_k}).
\]
Since 
\[
\log\left(\prod_{k=1}^\infty (C\cdot 4^k)^{1/p_k}\right)
=\sum_{k=1}^\infty\frac{1}{p_k}(\log C+k\log 4)<C,
\]
there is a constant $C$ such that
\[
||v||_{L^\infty}(B_{\theta R})\leq\frac{C}{((1-\theta)R)^{n/p}}\frac{1}{R^n}||v||_{L^p}(B_R).
\]

It remains to prove that the lemma is true for $0<p<2$. To see this we first use the inequality for $p=2$. That is
\[
\sup_{B_{\theta R}}\, v\leq\frac{C}{(1-\theta)^{n/2}}\left(\fint_{B_R}u^2\right)^{1/2}.
\]

We have
\[
\left(\int_{B_R}v^2\right)^{1/2}\leq\sup_{B_R}\, v^{1-p/2}\cdot\left(\int_{B_R}v^p\right)^{1/2}.
\]
Using Young's inequality, we have
\[
\phi(\theta R)=(1-\frac p2)\phi(R)+\frac{C}{((1-\theta)R)^{n/p}}||v||_{L^p},
\]
where 
\[
\phi(s)=\sup_{B_s}\, v.
\]

Thus we have the inequality
\[
\phi(t)\leq (1-\frac p2)\phi(s)+\frac{A}{(t-s)^\alpha}||v||_{L^p}
\]
for any $t\geq s$. Let 
\[
t_0=R_0<R,\qquad t_{i+1}=t_i+(1-\tau)t^i(R-R_0)
\]
and $0<\tau<1$ to be determined later. Then we have
\[
\phi(t_i)\leq(1-\frac p2)\phi(t_{i+1})+\frac{A}{((1-\tau)t^i(R-R_0))^\alpha}||u||_{L^2}.
\]
By iteration, we know that
\[
\phi(R_0)\leq A||u||_{L^2}\frac{1}{((1-\tau)t^i(R-R_0))^\alpha}
\left((1-\frac p2)\tau^\alpha+(1-\frac p2)^2\tau^{-2\alpha}+\cdots\right).
\]
We can choose $\tau$ close to $1$ such that $(1-p/2)t^{-\alpha}<1$. Then the 
above sequence is convergent and the theorem is proved.


\begin{lemma}[Weak Harnarck inequality] Let $v$ be the weak supersolution of~\eqref{div-1}. Then there is a $p_0>0$ such that
\[
\underset{B_{\theta R}}{{\rm ess}\,\inf}\, v\geq\frac{1}{C}\left(\fint_{B_R}v^{p_0} dx\right)^{1/p_0}
\]
\end{lemma}

{\bf Proof.} We assume that $v>0$ on $B_R$. Otherwise we can use $v+\eps$ instead.

Since $v$ is a supersolution, $v^{-1}$ is a subsolution. By the above lemma, we have
\[
\underset{B_{\theta R}}{{\rm ess}\,\sup}\, v^{-p}\leq C\fint_{B_R} v^{-p} dx,
\]
which is equivalent to 
\[
\underset{B_{\theta R}}{{\rm ess}\,\inf}\, v\geq C \left(\fint_{B_R} v^{-p} dx\right)^{-\frac 1p}
\]
We  write 
\begin{align*}
&
\underset{B_{\theta R}}{{\rm ess}\,\inf}\, v\geq C \left(\fint_{B_R} v^{-p} dx\right)^{-\frac 1p}\\
&
=C\left(\fint_{B_R}v^{-p} dx\cdot\fint_{B_R}v^p dx\right)^{-1/p}\cdot\left(\fint_{B_R}v^p dx\right)^{1/p}.
\end{align*}

We just need to prove that there exists a $p_0>0$ such that
\[
\fint_{B_R}v^{-p} dx\cdot\fint_{B_R}v^p dx\leq C.
\]

We let $w=\log v-\beta$, where $\beta$ is a constant to be determined. We wish to prove that
\begin{equation}\label{wish}
\int_{B_R}e^{p|w|} dx\leq C
\end{equation}
for some $p>0$. If the above inequality is true then we have
\[
\int_{B_R}e^{p\log v-p\beta} dx\leq C,\qquad 
\int_{B_R}e^{-p\log v+p\beta} dx\leq C,
\]
and the lemma follows.

The proof of ~\eqref{wish} is somewhat complicated but  the basic idea is still Moser iteration. Let $v^{-1}\xi^2$ be a test function, where $\xi \in C_0^\infty(B_R)$, we have
\[
0\geq\int a^{ij} v_i(v^{-1}\xi^2)_j=-\int \xi^2a^{ij}w_i w_j+2\int a^{ij} w_i \xi\xi_j.
\]
Without loss of generality, we assume that $R=1$. Then we have
\[
\int_{B_\sigma}|Dw|^2\leq C
\]
for any  $0<\sigma<1$. By choosing suitable $\beta$, we have the Poincar\'e inequality
\[
\int_{B_\sigma}|w|^2 dx\leq C.
\]

Let $\xi\in C_0^\infty(B_1)$ and let $\phi=\xi^2|w|^{2q}$. Then we have
\[
\int_{B_\sigma}\xi^2|w|^{2q}|D_i w|^2\leq 2q\int_{B_\sigma}\xi^2|w|^{2q-1} D_i w D_i|w|+2\int_{B_\sigma}\xi|w|^{2q}D_i wD_i \xi.
\]
Using the Young's inequality, we have
\[
2q|w|^{2q-1}\leq\frac{2q-1}{2q}|w|^{2q}+(2q)^{2q-1}.
\]
Thus we have
\[
\frac{1}{2q}\int_{B_\sigma}\xi^2|w|^{2q}|D_iw|^2\leq (2q)^{2q-1}\int_{B_\sigma}|D_i w|^2+4q\int_{B_\sigma}|w|^{2q}|D_i\xi|^2.
\]

Finally, we get
\[
\int_{B_\sigma}|D(\xi^2|w|^{2q})|^2 dx\leq C(2q)^{2q}+C\tau^{-2}q^2\int_{B_{\sigma+\tau}}|w|^{2q} dx,
\]
where $C$ only depends on $n$, $\Lambda/\lambda$, and $(1-\sigma)^{-1}$. Let $\kappa=n/(n-1)$. By Sobolev inequality, we have
\[
\left(\int_{B_\sigma}|w|^{2q\kappa} dx\right)^{1/\kappa}\leq C(2q)^{2q}+C\tau^{-2}q^2\int_{B_{\sigma+\tau}}|w|^{2q} dx.
\]
We let 
\[
q_i=\kappa^{i-1},\quad\delta_0=\sigma,\quad \delta_i=\delta_{i-1}-\frac{\sigma-1}{2^i}
\]

Then
\[
\left(\int_{B_{\delta_i}}|w|^{2\kappa^i} dx\right)^{1/\kappa}
\leq C(\kappa)^{2(i-1)\kappa^{i-1}}+C(4\kappa)^i\int_{B_{\delta_{i-1}}}|w|^{2\kappa^{i-1}}dx
\]

Let
\[
I_j=\left(\int_{B_{\delta_j}}|w|^{2\kappa^j}\right)^{1/2\kappa^j}.
\]

Then
\[
I_i\leq C^{1/\kappa^{i-1}}\kappa^{i-1}+C^{1/\kappa^{i-1}}(4\kappa)^{i/\kappa^{i-1}}I_{i-1}.
\]
Running the iteration we get
\[
I_j\leq C\kappa^j+cI_0.
\]
Thus we have 
\[
||w||_{L^q(B_1)}\leq Cq.
\]
If $pCe<1$, then we must have
\[
\int_{B_1}e^{p|w|}\leq C.
\]

\qed



\begin{theorem}[Harnack Inequality] Let $u$ be a nonnegative solution of ~\eqref{div-1}. Then we have
\[
\underset{B_{\theta R}}{{\rm ess}\,\sup}\,u\leq C \underset{B_{\theta R}}{{\rm ess}\,\inf}\,u
\]
where $C$ only depends on $n, \lambda/\lambda, (1-\theta)^{-1}$.
\end{theorem}


\begin{theorem}[H\"older estimate]  Using the above notations, $u$ is H\"older continuous in $B_{\theta R}$.
\end{theorem}

{\bf Proof.}
Let
\[
M(R)=\underset{B_{\theta R}}{{\rm ess}\,\sup}\,u,\quad m(R)=\underset{B_{\theta R}}{{\rm ess}\,\inf}\,u.
\]
Let $w(R)=M(R)-m(R)$. Then
\[
u-m(R)\geq 0
\]
on $B_{\theta R}$. Thus
\[
M(\theta R)-m(R)\leq C(m(\theta R)-m(R)).
\]
Therefore
\[
w(\theta R)\leq \left(1-\frac 1C\right)w(R)
\]
hence
\[
w(\theta^s R)\leq \left(1-\frac 1C\right)^sw(R).
\]
Let $R'<R$. Let $s$ be such that 
\[
\theta^{s+1}R\leq R'<\theta^s R.
\]
Then
\[
w(R')\leq w(\theta^sR)\leq \left(1-\frac 1C\right)^sw(R).
\]
Since 
\[
s=\left[\frac{\log (R'/R)}{\log\theta}\right]-1,
\]
We have
\[
w(R')\leq\left(\frac{R'}{R}\right)^rw(R),
\]
where 
\[
r=\left(1-\frac 1C\right)^{(\log\theta)^{-1}}<1
\]

Let $x,y\in  B(\theta R)$. Then we have
\[
|u(x)-u(y)|\leq 
C|x-y|^r
\]
and the H\"older continuity follows.

\qed


\section{Aleksandrov maximum principle}\label{ABP_principle}
The contents of this section is essentially from (cite chen)
\begin{definition}\label{def8} Let $\Omega$ be a bounded domain in $\R^n$ with smooth boundary. Let $y\in\Omega$. Define
\[
\chi(y)=\{p\in\R^n\mid u(x)\leq u(y)+p(x-y),\,\forall x\in \Omega\}.
\]
Then $\chi$ defines a map from $\Omega$ to the set of subsets of $\R^n$. We call it the \emph{normal map}.
\end{definition}

Here is the geometric interpretation of the normal map. Let the lower graph of $u$ in $\R^{n+1}$ be
\[
\{(x,z)\in\R^n\times \R\mid x\in\Omega, -\infty<z<u(x)\}. 
\]
If $p\in\chi(y)$, then the hyperplane
\[
z=u(y)+p\cdot(x-y)
\]
is a supporting plane of the graph of $u$. $\chi(y)$ is the set of $p$ such that $u(y)+p\cdot(x-y)$ is a supporting plane, and $(-p,1)$ is the normal vector of the corresponding the plane.

\begin{definition}\label{def9} Let $u\in \mathcal C(\Omega)$. Let 
\[
\Gamma_u=\{y\in\Omega\mid \chi(y)\neq\emptyset\}=\{y\in\Omega\mid \exists \,p\in\R^n, u(x)\leq u(y)+p\cdot(x-y)\}.
\]
We call $\Gamma_u$ the \emph{contact set} of $u$.
\end{definition}

We consider the convex hull of the lower-graph of $u$ and let $\hat u$ be the corresponding function.
Then $\Gamma_u$ is the intersection  of the graphs of $u$ and $\hat u$ projecting to $\{z=0\}$, which justifies its name: contact set.

If $u\in \mathcal C^1(\Omega)$, $y\in\Gamma_u$, then $\chi(y)=\{Du(y)\}$. If $u\in \mathcal C^2(\Omega)$, then $-D^2u(y)\geq 0$. In fact, a little bit more is true:

\begin{lemma} Let $u\in W_{{\rm loc}}^{2,1}(\Omega)\cap \mathcal C(\Omega)$. Then
\[
\chi(y)=\{Du(y)\},\qquad -D^2u(y)\geq 0,\quad a.e.\quad y\in\Gamma_u.
\]
\end{lemma}

\begin{definition} Let
\[
\chi(\Omega)=\chi(\Gamma_u)=\bigcup_{y\in\Omega}\chi(y).
\]
\end{definition}

\begin{example} Let $\Omega=B_d(x_0)$, where $B_d(x_0)$ is the ball of radiou $d$ centered at $x_0$. Consider the function
\[
u(x)=\frac{\lambda}{d}(d-|x-x_0|).
\]
The graph of $u$ is a cone with $(x_0,\lambda)$ as the vertex and with $B_d(x_0)$ as the base. We have $\Gamma_u=\Omega$ and 
\[
\chi(y)=\left\{
\begin{array}{ll}
B_{\lambda/d}(0) & y=x_0;\\
-\frac{\lambda}{d}\cdot\frac{y-x_0}{|y-x_0|} &y\neq x_0.
\end{array}
\right.
\]
\end{example}

\begin{definition} Let $\Omega\subset \R^n$, $x_0\in\Omega$. Construct a cone in $\R^n$ with $(x_0,\lambda)$ being the vertex and $\{x\in\Omega, z=0\}$ being the base. Let $\omega$ be the function of the graph. Define
\[
\Omega[x_0,\lambda]=\chi_\omega(\Omega)
\]
\end{definition}



\begin{lemma}\label{15} Let $u\in \mathcal C(\Omega)$. Then
\begin{enumerate} 
\item for any $y\in\Gamma_u$,
\[
|p|\leq\frac{2\sup|u|}{{\rm dist}(y,\pa\Omega)},\qquad \forall p\in\chi(y),
\]
\item the normal map maps a compact subset of $\Omega$ to a closed set of $\R^n$.
\end{enumerate}
\end{lemma}


\begin{proof}
For all $y\in\Gamma_u$, we have
\[
u(y)+p\cdot(x-y)\geq u(x),\qquad \forall x\in\Omega.
\]
We draw a ray starting from $y$ with direction $-p$. Suppose the ray hits $\pa\Omega$ at $x_0$. 
Since $x_0-y$ is proportional to $p$ (and in the opposite direction), we have
\[
u(x_0)\leq u(y)+p(x_0-y)=u(y)-|x_0-y|\cdot|p|.
\]
Thus
\[
|p|\leq\frac{2\sup|u|}{|x_0-y|}\leq\frac{2\sup|u|}{{\rm dist}(y,\pa\Omega)}.
\]

To prove (2), let $F$ be  a compact subset of $\Omega$, let  $\{p_n\}\subset\chi(F)$, and let $p_n\to p$ as $n\to\infty$. Since $p_n\in\chi(F)$, there is a sequence $y_n\in F$ such that $p_n\in\chi(y_n)$. By definition, we have
\[
u(y_n)+p_n(x-y_n)\geq u(x),\qquad\forall x\in\Omega
\]
By passing the subsequence if necessary, we may assume that $p_n\to p_0, y_n\to y_0$. 
Then we have $p_0\in\chi(y_0)$.

\end{proof}

\begin{lemma}\label{16} (1). Let $\Omega,A$ be open domains of $\R^n$, $\Omega\subset A$. Then if $x_0\in\Omega$,
\[
\Omega[x_0,\lambda]\supset A[x_0,\lambda].
\]

(2). Let $d$ be the diameter of $\Omega$. Then
\[
|\Omega[x_0,\lambda]|\geq\left(\frac{\lambda}{d}\right)^nw_n,
\]
where $w_n$ is the volume of the unit ball in $\R^n$.
\end{lemma}

\begin{proof} (1) is obvious and (2) follows from the above example.

\end{proof}

\begin{lemma}\label{17-2} Let $u\in \mathcal C^2(\bar\Omega), g\in \mathcal C(\bar\Omega), g\geq 0$, and let $E$ be a measurable subset of $\Gamma_u$. Then
\[
\int_{Du(E)}g(x(p))dp\leq\int_E g(x)\det(-D^2u) dx,
\]
where $x(p)=(Du)^{-1}(p)$ is defined and is continuous outside a measure zero set.
\end{lemma}

\begin{proof} Let $J(x)=\det(-D^2u)$, $S=\{x\in\Omega\mid J(x)=0\}$. By the Sard's theorem, $|Du(S)|=0$.

We first assume that $E$ is an open set; $E\backslash S$ is also open. We can find a sequence 
of parallel, disjoint cubes $\{C_\ell\}_{\ell=1}^\infty$ such that 
\[
E\backslash S=\bigcup_{\ell=1}^\infty C_\ell.
\]
Moreover, we assume that $C_\ell$ are so small such that
\[
Du: C_\ell\to Du(C_\ell)
\]
are differmorphisms. Therefore
\[
\int_{Du(C_\ell)}g(x(p)) dp=\int_{C_\ell} g(x)J(x) dx.
\]
Thus we have
\begin{align*}
&
\int_{Du(E\backslash S)} g(x(p))dp\leq\sum_\ell\int_{Du(C_\ell)}g(x(p))dp\\
&=\sum_\ell\int_{C_\ell} g(x)J(x) dx=\int_{E\backslash S} g(x)J(x) dx.
\end{align*}

By the Sard's Theorem, $|Du(S)|=0$. Thus the result is true when $E$ is an open set. 

If $E$ is a measurable subset of $\Gamma_u$, then there is an open set $G\supset E\backslash S$ such that on $G$, $J(x)>0$. On the other side, since $E\backslash S$ is measurable, we can find open sets $\{O_\ell\}$ such that $E\backslash S\subset O_\ell$, and $|O_\ell\backslash(E\backslash S))|\to 0$ as $\ell\to\infty$. Thus we have
\[
\int_{Du(E)}g(x(p))dp\leq\int_{Du(G\cap O_\ell)} g(x(p))dp\leq\int_{G\cap O_\ell} g(x)J(x) dx.
\]
The lemma follows by letting  $\ell\to\infty$.

\end{proof}

\begin{lemma}\label{18} Let $u\in \mathcal C(\bar\Omega)$. Assume that on $\pa\Omega$, $u\leq 0$; and there is an  $x_0\in\Omega$ such that $u(x_0)>0$. Then
\[
\Omega[x_0,u(x_0)]\subset\chi(\Gamma^+_u),
\]
where $\Gamma_u^+=\Gamma_u\cap\{u\geq 0\}$.
\end{lemma}


\begin{proof} Let $p\in\Omega[x_0,u(x_0)]$. Then by definition
\[
u(x_0)+p\cdot (x-x_0)\geq\hat u(x)\geq 0.
\]

Define
\[
\lambda_0=\inf\{\lambda\mid\lambda+p\cdot (x-x_0)\geq u(x),\forall x\in\bar\Omega\}.
\]
By the continuity of $u$, we have
\[
\lambda_0+p\cdot (x-x_0)\geq u(x),\qquad\forall x\in\bar\Omega,
\]
and there is a $\xi$ such that 
\[
\lambda_0+p\cdot (\xi-x_0)=u(\xi).
\]

\begin{enumerate}
\item if $\lambda_0=u(x_0)$, then $x_0\in\Gamma_u^+, p\in\chi(\Gamma_u^+)$;
\item if $\lambda_0>u(x_0)$, then $\xi\not\in\pa\Omega$, otherwise
\[
u(\xi)>u(x_0)+p\cdot (\xi-x_0)\geq 0
\]
contradicting to $u\leq 0$ on $\pa\Omega$.  Since we have
\[
u(\xi)+p\cdot (x-\xi)\geq u(x), \forall x\in\bar\Omega,
\]
where $\xi\in\Gamma_u^+$ and $p\in\chi(\Gamma_u^+)$.
\end{enumerate}

\end{proof}

With the above lemma we  immediately get the following

\begin{lemma}[Aleksandrov type estimate]\label{19} Let $u\in\mathcal  C^2(\bar\Omega)$ and $u|_{\pa\Omega}\leq 0$. Then
\[
\sup_\Omega u\leq\frac{d}{\sqrt[n]{w_n}}\left[\int_{\Gamma_u^+}\det(-D^2 u) dx\right]^{1/n},
\]
where $d={\rm diam}\,\Omega$.
\end{lemma}

\begin{proof} By Lemma~\ref{16} and Lemma~\ref{18}, we have
\[
|\chi(\Gamma_u^+)|\geq w_n\left[\frac{u(x_0)}{d}\right]^n.
\]
Thus
\[
u(x_0)\leq\frac{d}{\sqrt[n]{w_n}}|\chi(\Gamma_u^+)|^{1/n}.
\]
Using Lemma~\ref{17-2}, we have
\[
|\chi(\Gamma_u^+)|\leq \int_{\chi(\Gamma_u^+)} dp\leq\int_{\Gamma_u^+}\det(-D^2 u) dx.
\]

\end{proof}


We can weaken the smoothness of the function $u$ to get the following

\begin{theorem}\label{thm28} Let $u\in \mathcal C(\bar\Omega)\cap W_{{\rm loc}}^{2,n}(\Omega)$. Then
\[
\sup_\Omega u\leq\sup_{\pa\Omega} u+\frac{d}{\sqrt[n]{w_n}}\left[\int_{\Gamma_v^+}\det(-D^2 u) dx\right]^{1/n},
\]
where $v=u-\sup_{\pa\Omega} u$.
\end{theorem}

\begin{proof}  By assumption, there is a sequence $\{u_m\}\subset \mathcal C^2(\bar\Omega)$ such that $u_m\to u$ in $W_{{\rm loc}}^{2,n}(\Omega)$. That is, for any $\Omega'\subset\subset\Omega$, if $m\to\infty, u_m\to u$ in $W^{2,n}(\Omega')$. For any $\eps>0$, let
\[
\Omega_\eps=\{x\in\Omega\mid {\rm dist}(x,\pa\Omega)>\eps\}.
\]
Let
\begin{align*}
&v_\eps=u-\sup_{\Omega\backslash\Omega_\eps} u-\eps
\\
& v_{m,\eps}=u_m-\sup_{\Omega\backslash\Omega_\eps} u_m-\eps
\end{align*}
By Theorem~\ref{19}, we have
\[
\sup_\Omega v_{m,\eps}\leq\frac{d}{\sqrt[n]{w_n}}\left(\int_{\Gamma_{v_{m,\eps}}^+}\det(-D^2 u_m) dx\right)^{1/n}.
\]
Since  $\Gamma^+_{m,\eps}\subset\Omega_\eps$, we write
\begin{align*}
&
\int_{\Gamma_{v_{m,\eps}}^+}\det(-D^2 u_m) dx\\
&
=\int_{\Gamma_{v_{m,\eps}}^+}(\det(-D^2 u_m) -\det(-D^2 u))dx+\int_{\Gamma_{v_{m,\eps}}^+}\det(-D^2 u) dx\\
&=I_{1,m}+I_{2,m}
\end{align*}
Obviously,
we have $I_{1,m}\to 0$ as $m\to\infty$. On the other hand, we have
\[
\varlimsup_{m\to\infty}\Gamma_{v_{m,\eps}}^+\subset\Gamma_{v_\eps}^+.
\]
In fact, if 
\[
x_0\in\varlimsup_{m\to\infty}\Gamma_{v_{m,\eps}}^+.
\]
Then there is a subsequence $m_k$ such that $x_0\in\Gamma_{v_{m_k,\eps}}$. By Lemma~\ref{15},

\[
|p_k|\leq\frac{4\sup |u_m|}{{\rm dist}(x,\pa\Omega)}\leq\frac{4}{\eps}\sup |u_m|
\]
for any $p_k\in\chi_{v_{m_k,\eps}}(x_0)$.

Assuming $p_k\to p_0$. Then $p_0\in\chi_{v_\eps}(x_0)$, hence $x_0\in\Gamma^+_{v_\eps}$. Thus
\[
\varlimsup_{m\to\infty} I_{2,m}\leq\int_{\Gamma^+_{v_\eps}}\det(-D^2 u) dx.
\]
The theorem is proved.

\end{proof}

\begin{theorem}\label{thm35}
Let $u\in \mathcal \mathcal C(\bar\Omega)\cap W_{{\rm loc}}^{2,n}(\Omega)$ and let 
\[
-a^{ij}D_{ij} u\leq f.
\]
Then
\[
\sup_\Omega u(x)\leq\sup_{\pa\Omega} u(x)+\frac{d}{n\sqrt[n]{w_n}}\left|\left|f^+/\mathcal D^*\right|\right|_{L^n(\Gamma_u^+)},
\]
where $v=u-\sup_{\pa\Omega} u(x)$, and $\mathcal D^*=(\det a^{ij})^{1/n}$.
\end{theorem}

\begin{proof} We know that on $\Gamma_v$, $-D^2 u\geq 0$ almost everywhere. Thus we have
\[
-a^{ij}D_{ij} u\geq n(\det A(-D^2u))^{1/n}=n\mathcal D^*\det (-D^2 u)^{1/n}.
\]
Thus we have
\[
\det(-D^2 u)\leq\left(\frac{f^+}{n\mathcal D^*}\right)^n,
\]
and the theorem is proved.

\end{proof}

\begin{theorem}[Aleksandrov maximum principle] Let $u\in \mathcal C(\bar\Omega)\cap W_{{\rm loc}}^{2,n}(\Omega)$ 
such that
\[
Lu\leq f,
\]
where
\[
Lu=-a^{ij}D_{ij} u+b^i D_i u+cu=f.
\]
Assume that 
\begin{align*}
& (a^{ij})\geq 0\qquad \text{ on } \Omega\\
&\sum_i\left|\left|\frac{b^i}{\mathcal D^*}\right|\right|_{L^n(\Omega)}\leq\beta\\
& c\geq 0
\end{align*}
Then
\[
\sup_\Omega u\leq\sup_{\pa\Omega} u^++C\left|\left|\frac{b^i}{\mathcal D^*}\right|\right|_{L^n(\Gamma_u)},
\]
where $C$ only depends on $n,\beta$ and ${\rm diam}\,\Omega$.
\end{theorem}



\section{Krylov-Safonov Estimates}\label{Krylov-Safonov}
For elliptic equations of non-divergence form, in 1980, Krylov-Safonov obtained the H\"older estimate, which was almost 20 years after de Giorgi-Nash-Moser's results on elliptic equations of divergence form.

We shall use similar steps as in the de Giorgi-Nash-Moser case: we start from local maximum principle, weak Harnark inequality,  Harnark inequality, and the $\mathcal C^\alpha$-estimate. The contents of this section is essentially from (Cite chen).\\

For the sake of simplicity, we only consider  the following equation
\begin{equation}\label{div-2}
Lu=-a^{ij} D_{ij} u=f
\end{equation}
on $\Omega$, where $(a^{ij})$ satisfies the uniform ellipticity condition:
\begin{equation}\label{uni-4}
\lambda>0, \Lambda/\lambda\leq\gamma
\end{equation}
on $\Omega$, where $\lambda, \Lambda$ are the minimum and maximum eigenvalues of $(a^{ij})$
respectively, and $\gamma$ is a positive number.

\begin{theorem}[local maximum principle] Assume equation~\eqref{div-2}. Let $u\in W^{2,n}(\Omega)$ and let $Lu\leq f$ almost everywhere on $\Omega$. Let $f/\lambda\in L^n(\Omega)$. Then for any $p>0$, $B_{2R}\subset\Omega$, we have
\[
\sup_{B_R(y)} u\leq C\left[\fint_{B_{2R}(y)}(u^+)^p dx\right]^{1/p}+R\left|\left|\frac{f}{\lambda}\right|\right|_{L^n(B_{2R}(y))},
\]
where $C$ only depends on $n,\gamma,p$.
\end{theorem}

\begin{proof} By using the transformation
\[
x\mapsto\frac{x-y}{2R},
\]
we can assume that $y=0$ and $R=1/2$. Taking
\[
\eta(x)=(1-|x|^2)^\beta
\]
as the cut-off function, then we have
\begin{align}\label{26}
\begin{split}
& D_i\eta=-2\beta x_i(1-|x|^2)^{\beta-1};\\
&D_{ij}\eta=-2\beta\delta_{ij}(1-|x|^2)^{\beta-1}+4\beta(\beta-1)x_ix_j(1-|x|^2)^{\beta-2};\\
& L\eta=[2\beta\, \sum_{i=1}^na^{ii}(1-|x|^2)-4\beta(\beta-1)\sum_{i,j=1}^na^{ij}x_ix_j](1-|x|^2)^{\beta-2}.
\end{split}
\end{align}

Let $v=\eta u$. Then we have
\[
Lv=uL\eta+\eta Lu-2a^{ij} D_i\eta D_j u.
\]
On the contact set $\Gamma_v$ defined in Definition~\ref{def9}, we have
\[
|Dv|\leq\frac{v(x)}{{\rm dist}(x,\pa B_1)}\leq\frac{v(x)}{1-|x|}
\]
by Lemma~\ref{15}.
Then  we have
\[
|Du|\leq\frac1\eta|Dv-uD\eta|\leq\frac 1\eta\left[\frac{v(x)}{1-|x|}+u|D\eta|\right]\leq 2(1+\beta)
\eta^{-1/\beta} u.
\]
Thus we have
\[
Lv\leq\eta f+(16\beta^2+2n\beta)\Lambda\eta^{-2/\beta} v\leq C\lambda\eta^{-2/\beta} v+f
\]
almost everywhere on $\Gamma_v$, where $C$ only depends on $n,\beta,\gamma$.

By Theroem~\ref{thm35}, we have
\begin{align*}
&\sup _{B_1} v\leq  C\left(||\eta^{-2/\beta}v^+||_{L^n(B_1)}+\left|\left|\frac{f}{\lambda}\right|\right|_{L^n(B_1)}\right)\\
&\leq C\left((\sup v^+)^{1-2/\beta}||(u^+)^{2/\beta}||_{L^n(B_1)}+\left|\left|\frac{f}{\lambda}\right|\right|_{L^n(B_1)}\right).
\end{align*}
Using the Young's inequality, we have

\[
\sup_{B_1} v\leq C \left|\left|\frac{f}{\lambda}\right|\right|_{L^n(B_1)}+\eps\sup_{B_1} v^++C_\eps||(u^+)^{2/\beta}||^{\beta/2}_{L^n(B_1)}.
\]
We take $\eps=1/2$ and if $p<n$, we let $\beta=2n/p>2$. Then we have
\[
\sup_{B_{1/2}} u\leq C\left(
||u^+||_{L^p(B_1)}+\left|\left|\frac{f}{\lambda}\right|\right|_{L^n(B_1)}\right).
\]
If $p\geq n$, then the estimate follows from the H\"older inequality.

\end{proof}

The most complicated part of the estimate is the weak Harnark inequality. Let $K_R(y)$ be a tube on $\R^n$ with center in $y$ and side length $2R$. 
The following argument was introduced by Krylov-Safonov. We start  with the following




\begin{lemma}\label{20} Let $K_0$ be a tube in $\R^n$ and let $\Gamma$ be a measurable subset of $K_0$. Let $0<\delta<1$. Let
\[
\Gamma_\delta=\bigcup\{ K_{3R}(y)\cap K_0\mid K_R(y)\subset K_0, |\Gamma\cap K_R(y)|\geq\delta|K_R(y)|\}.
\]
If $\Gamma_\delta\neq K_0$, then
\[
|\Gamma|\leq\delta|\Gamma_\delta|.
\]
\end{lemma}

\begin{proof} If $|K_0\cap \Gamma|>\delta|K_0|$, then $\Gamma_\delta=K_0$. Thus if $\Gamma_\delta\neq K_0$, we must have
\[
|K_0\cap\Gamma|\leq\delta|K_0|.
\]

Dividing $K_0$ into $2^n$ congruent sub-cubes and denoting them as $\{K(i_1)\}_{i_1=1}^{2^n}$. On each $K(i_1)$, we have either of the following
\begin{enumerate}
\item $|\Gamma\cap K(i_1)|\leq\delta|K(i_1)|$;
\item $|\Gamma\cap K(i_1)|>\delta|K(i_1)|$.
\end{enumerate}
Let the collection of $K(i_1)$ satisfying (2) be $\mathcal F_1$. If $K(i_1)$ satisfies (1), continue the subdivision such that
\[
K(i_1)=\bigcup_{i_2=1}^{2^n}K(i_1,i_2).
\]
For each $K(i_1,i_2)$, we still have two cases. We let $\mathcal F_2$ be the collection of all $K(i_1,i_2)$ satisfying (2). On the other hand, if $K(i_1,i_2)$ satisfying (1), continue the subdivision. Thus we have the sets $\mathcal F_1,\cdots,\mathcal F_m,\cdots $. Let
\[
\mathcal F=\{K(i_1,\cdots,i_{m-1})\mid K(i_1,\cdots,i_{m-1}, i_m)\in\mathcal F_m\}.
\]
Thus if $K(i_1,\cdots,i_m)\in\mathcal F_m$, then
\begin{align*}
& |K(i_1,\cdots,i_m)\cap\Gamma|>\delta|K(i_1,\cdots,i_m)|;\\
&|K(i_1,\cdots,i_{m-1})\cap\Gamma|\leq \delta |K(i_1,\cdots,i_{m-1})|.
\end{align*}
By the definition of $\Gamma_\delta$, we have
\[
K(i_1,\cdots,i_{m-1})\subset\Gamma_\delta.
\]
Let
\[
\tilde\Gamma_\delta=\bigcup_{K\in\mathcal F} K\subset\Gamma_\delta.
\]
Then we have
\[
|\tilde\Gamma_\delta\cap\Gamma|=\sum_{K\in\mathcal F}|K\cap \Gamma|\leq\delta\sum_{K\in\mathcal F}|K|=\delta|\tilde\Gamma_\delta|\leq\delta|\Gamma_\delta|.
\]
On the other hand, by the definition of $\tilde\Gamma_\delta$, since  $\Gamma$ is measurable, by the Lebesgue Theorem, we know that for almost every point of $\Gamma$ is dense. Thus we have
\[
|\Gamma|=|\Gamma\cap\tilde\Gamma_\delta|\leq\delta|\Gamma_\delta|.
\]

\end{proof}

\begin{theorem}[Weak Harnark inequality] Suppose $L$ satisfies the uniform ellipticity conditions~\eqref{uni-4}; $u\in W^{2,n}(\Omega)$; $Lu\geq f$; $f/\lambda\in L^n(\Omega)$; $u\geq 0$ on $B_{2R}(y)\subset\Omega$. Then there is a $p>0$, $C>1$ such that
\[
\left(\int_{B_R(y)}|u|^p dx\right)^{1/p}\leq C\left[\inf_{B_R(y)} u+R
\left|\left|\frac{f}{\lambda}\right|\right|_{L^n(B_{2R}(y))}\right],
\]
where $p,C$ only depends on $n,R$.
\end{theorem}


\begin{proof} The proof  of this theorem is quite long so we divide it into the following 5 steps.\\

{\bf Step 1.} Using rescaling
\[
x\mapsto\frac{x-y}{2R},
\]
we assume that $y=0$, $R=1/2$. Let 
\[
\tilde u=u+\left|\left|\frac{f}{\lambda}\right|\right|_{L^n(B_1)},\quad \Gamma=\{x\in B_1\mid\tilde u(x)\geq 1\}.
\]
We want to prove that there exists $C>0$, $0<\delta<1$ such that whenever
\begin{equation}\label{27-1}
|\Gamma\cap K_\alpha|\geq\delta|K_\alpha|,
\end{equation}
we have
\begin{equation}\label{27}
\inf_{K_{3\alpha}}\,\tilde u\geq C^{-1},
\end{equation}
where $K_\alpha$ is the tube centered at $0$ with side length $2\alpha$, $\alpha=1/(6\sqrt n)$, $C,\delta$ only depends on $n,\gamma$.

For any $\eps>0$. Let 
\[
w=\log(\tilde u+\eps)^{-1}, g=\frac{f}{\tilde u+\eps}.
\]
Then we have
\[
-a^{ij}D_{ij} w=\frac{1}{\tilde u+\eps}a^{ij} D_{ij} u-a^{ij}D_iwD_jw
\leq -g-a^{ij} D_iw D_jw.
\]
Let $\eta=(1-|x|^2)^\beta$. Let $v=\eta w$. Then we have
\begin{align*}
& Lv=\eta Lw+w L\eta-2a^{ij}D_i wD_j \eta\\
&\leq -\eta g-\eta a^{ij}D_i w D_j w-2a^{ij} D_iw D_j\eta+vL\eta/\eta\\
&\leq -\eta g+\frac 1\eta a^{ij} D_i\eta D_j\eta+\frac{v L\eta}{\eta}.
\end{align*}
By~\eqref{26}, we know when
\[
(2\beta-1)\lambda|x|^2\geq n\Lambda,
\]
we have $L\eta\leq 0$. Let $\alpha\in(0,1)$. If $|x|>\alpha$, we take 
\[
\beta-1\geq\frac{n\gamma}{2\alpha^2}.
\]
Then we have $L\eta\leq 0$ for $|x|\geq \alpha$. Thus we have
\[
Lv\leq|g|+4\beta^2\Lambda+\sup_{B_\alpha}\left(\frac{L\eta}{\eta}\right)\chi(B_\alpha) v,
\]
where $\chi(B_\alpha)$ is the characteristic function of $B_\alpha$. Note that
\[
\left|\left|\frac{g}{\lambda}\right|\right|_{L^n(B_1)}\leq 1.
\]
By the Aleksandrov maximum principle, we have
\begin{equation}\label{new-1}
\sup_{B_1} v\leq C[1+||v^+||_{L^n(B_\alpha)}],
\end{equation}
where $C$ only depends on $n,\gamma,\alpha$.

In order to use the measure theory, we change ball into cube. We have 
\[
\sup_{B_1} v\leq C[1+||v^+||_{L^n(B_\alpha)}]\leq C(1+|K_\alpha^+|^{1/n}\sup_{B_1} v),
\]
where 
\[
K_\alpha^+=\{x\in K_\alpha\mid v>0\}=\{x\in K_\alpha\mid\tilde u+\eps<1\}.
\]
If 
\[
\frac{|K_\alpha^+|}{K_\alpha}\leq\theta\overset{\Delta}{=}\frac{1}{(2C)^n|K_\alpha|}=\frac{1}{(4C\alpha)^n}.
\]
Then we have
\[
\sup_{B_1} v\leq 2C,
\]
or equivalently
\[
\inf_{B_{1/2}}\,(\tilde u+\eps)\geq\frac 1C.
\]
Letting 
$\eps\to 0$, we get~\eqref{27}. On the other hand, if $\frac{|K_\alpha^+|}{K_\alpha}>\theta$, we let $\delta=1-\theta$, $\alpha=\frac{1}{6\sqrt n}$. Then we have $K_{3\alpha}\subset B_{1/2}$. If $|\Gamma\cap K_\alpha|\geq\delta|K_\alpha|$, then by Lemma~\ref{20}, we have
\[
|K_\alpha^+|=|K_\alpha\backslash (\Gamma\cap K_\alpha)|\leq\theta|K_\alpha|
\]
which contradicts to the assumption~\eqref{27-1}.\\

{\bf Step 2.} For any positive integer $m$, if
\[
|\Gamma\cap K_\alpha|\geq\delta^m|K_\alpha|,
\]
then
\[
\inf_{K_\alpha}\tilde u\geq C^{-m},
\]
where $C$ is the constant in Step 1.

If $m=1$, the statement is proved  by the above step. Assume that the claim is true for $m\geq 1$. Assume that 
\[
|\Gamma\cap K_\alpha|\geq\delta^{m+1}|K_\alpha|.
\]
Let $\tilde K_0=K_\alpha$,
\[
\Gamma_\delta=\bigcup\{K_{3r}(x)\cap \tilde K_0\mid K_r(x)\subset\tilde K_0, |\Gamma\cap K_r(x)|\geq\delta|K_r(x)|\}.
\]
By Lemma~\ref{20}, we have either 
\[
\Gamma_\delta=\tilde K_0,\quad {\rm or}\quad |\Gamma\cap\tilde K_0|\leq\delta|\Gamma_\delta|.
\]
By the definition of $\Gamma_\delta$ and Step 1, we have
\[
\inf_{\Gamma_\delta}\tilde u\geq C^{-1}.
\]
If $\Gamma_\delta=\tilde K_0=K_\alpha$, then the above implies the claim. If $|\Gamma\cap\tilde K_0|\leq\delta|\Gamma_\delta|$. Let $v=Cu$. Then $v$ satisfiws
\[
-a^{ij}D_{ij} v\geq Cf.
\]
Let
\[
\tilde v=v+\left|\left|\frac{Cf}{\lambda}\right|\right|_{L^n(B_1)}=C\tilde u.
\]
Let
\[
\tilde\Gamma=\{x\in B_1\mid\tilde v\geq 1\}.
\]
Then $\Gamma_\delta\subset\tilde\Gamma$ and we have
\[
|\tilde\Gamma\cap\tilde K_0|\geq|\Gamma_\delta|\geq\frac{1}{\delta}|\Gamma\cap\tilde K_0|
=\frac 1\delta|\Gamma\cap K_\alpha|\geq\delta^m|K_\alpha|=\delta^m|\tilde K_0|.
\]
By the inductive assumption, we have
\[
\inf_{K_\alpha}\tilde v\geq C^{-m},
\]
which is equivalent to 
\[
\inf_{K_\alpha}\tilde u\geq C^{-(m+1)}.
\]\\

{\bf Step 3.} 
Let
\[
\Gamma_t=\{x\in B_1\mid\tilde u(x)>t\}.
\]
Then there is a $C>1, \mu>0$ such that 
\begin{equation}\label{plk}
|B_\alpha\cap \Gamma_t|\leq C|B_\alpha|\left(\frac{\inf_{B_\alpha}\tilde u}{t}\right)^\mu,
\end{equation}
where $C$ and $\mu$ depend only on $n,\gamma$.

Let $v=u/t$ and $\tilde v=\tilde u/t$.
\[
\tilde\Gamma\overset{\Delta}{=}\{x\in B_1\mid\tilde v(x)>1\}=\Gamma_t.
\]
If $|B_\alpha\cap\Gamma_t|=0$, then~\eqref{plk} is obvious. Now assume that $|B_\alpha\cap\Gamma_t|\neq 0$, then there is a positive number $m$ such that
\[
\delta^m|K_\alpha|\leq|\tilde\Gamma\cap K_\alpha|\leq\delta^{m-1}|K_\alpha|.
\]
That it, 
\[
\log\frac{|\tilde\Gamma\cap K_\alpha|}{K_\alpha}\cdot(\log\delta)^{-1}\leq m\leq
1+\log\frac{|\tilde\Gamma\cap K_\alpha|}{K_\alpha}\cdot(\log\delta)^{-1}.
\]
By Step 2, we have
\[
\inf_{K_\alpha}\tilde v\geq C^{-m}\geq C^{-1}\left[\frac{|\tilde\Gamma\cap K_\alpha|}{K_\alpha}\right]^{\frac{\log C}{\log \delta^{-1}}}.
\]
Let $\mu=\log\delta^{-1}/\log C$.
Then we have
\[
|\Gamma_t\cap K_\alpha|\leq (C\inf_{K_\alpha}\tilde v)^\mu|K_\alpha|.
\]\\


{\bf Step 4.} There is a $p>0$ such that
\[
\left(\fint_{B_\alpha}u^p dx\right)^{1/p}\leq C\left(
\inf_{B_\alpha} u+\left|\left|\frac f\lambda\right|\right|_{L^n(B_1)}\right).
\]

We have
\[
\int_{B_{\alpha}} u^p dx=p\int_0^\infty t^{p-1}|B_\alpha\cap\Gamma_t| dt
=p\int_0^b  t^{p-1}|B_\alpha\cap\Gamma_t| dt+p\int_b^\infty  t^{p-1}|B_\alpha\cap\Gamma_t| dt,
\]
where $b$ is a number to be determined. Therefore, we have
\[
\int_{B_{\alpha}} u^p dx\leq p\int_0^b\int_0^b t^{p-1}|B_\alpha| dt+p\int_b^\infty Cm_0^\mu|B_\alpha| t^{p-\mu-1} dt,
\]
where $m_0={\displaystyle \inf_{B_\alpha}}\,\tilde u$. Let $p=\mu/2$. Then
\[
\int_{B_\alpha} u^p dx\leq b^p|B_\alpha|+Cm_0^{2p}b^{-p}|B_\alpha|.
\]
Let $b=C^{1/2p}m_0$. Then we have
\[
\int_{B_\alpha}u^p dx\leq 2C^{1/2}m_0^p\leq 2C^{1/2}\left(\inf_{B_\alpha} u+\left|\left|\frac f\lambda\right|\right|_{L^n(B_1)}\right)^p.
\]\\



{\bf Step 5.} Using covering technique to prove
\[
\left(\int_{B_{1/2}}u^p dx\right)^{1/p}\leq C\left(\inf_{B_{1/2}} \, u+\left|\left|\frac f\lambda\right|\right|_{L^n(B_1)}\right).
\]

We note that $u\in W^{2,n}(B_{\frac 12})$, there is a $x_0\in\bar{B_{1/2}}$ such that
\[
u(x_0)=\inf_{B_{\frac 12}}\,u.
\]
Consider the integration
\[
\int_{B_{\frac 12+\frac\alpha 4}}dy\cdot\int_{B_{\frac\alpha 4}(y)} u^p dx\geq|B_{\frac \alpha 4}|
\int_{B_{\frac 12}}u^p(y) dy.
\]
Using the mean value theorem, there is a $y$ such that
\[
y_0\in {B_{\frac 12+\frac\alpha 4}}
\]
such that 
\[
\int_{B(y_0)} u^p dx\geq\int_{B_{\frac 12}}u^p dx.
\]
The theorem is proved.

\end{proof}

\begin{theorem}[Harnark inequality] Let the coefficients of $L$ satisfy the uniform elliptic condition. $u\in W^{2,n}(\Omega)$ satisfies $Lu=f$. Let $f/\lambda\in L^n(\Omega)$. Suppose $u\geq 0$ on $B_{2R}(y)\subset\Omega$. Then
\[
\sup_{B_{\frac R2}(y)}\, u\leq C\left(\inf_{B_{\frac R2}(y)}\, u+R\left|\left|\frac f\lambda\right|\right|_{L^n(B_{2R}(y)}\right).
\]
\end{theorem}

\begin{proof}
This follows from combining the local maximum principle and the weak Harnark inequality. 
\end{proof}

\begin{theorem} Using the above notations, we have
\[
\underset{B_R(y)}{{\rm osc}}\, u\leq C\left(\frac{R}{R_0}\right)^\alpha\left(\underset{B_{R_0}(y)}{{\rm osc}}\, u
+R_0\left|\left|\frac f\lambda\right|\right|_{L^n(\Omega)}\right).
\]
In particular, $u$ is  H\"older continuous.
\end{theorem}
