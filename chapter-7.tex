

\chapter{The Hodge Thoerem}\label{Hodge_theorem}

\section{The Poincar\'e Lemma}\label{Poincare_Lemma}
In this section, we prove the following

\begin{theorem} Let $U$ be the unit ball of the original point of $\R^n$. Let $\omega$ be a $p$-form on $U$ with $p\geq 1$. Then there is a $p-1$ form $\eta$ such that $\omega=d\eta$.
\end{theorem}

{\bf Proof.} 
If $\omega$ is a $1$-form, the proof is straightforward. Let 
\[
f(x)=\int_{\ell}\omega,
\]
where $\ell$ is a curve connecting the original point to $x$. Then
\[
df=\omega.
\]

In general, we use the math induction. We write
\[
\omega=\omega_1+dx_n\wedge\omega_2,
\]
where $\omega_1,\omega_2$ are forms without the factor $dx_n$. Let
\[
\omega=\sum a_{i_1\cdots i_{p-1}}(x_1,\cdots,x_n)dx_{i_1}\wedge\cdots\wedge dx_{i_{p-1}}.
\]
Define
\[
\eta=\int a_{i_1\cdots i_{p-1}}(x_1,\cdots,x_n)dx_n\, dx_{i_1}\wedge\cdots\wedge dx_{i_{p-1}}.
\]
Then $d(\omega-d\eta)=0$ and $\omega-d\eta$ has not $dx_n$ factor. As a result, the coefficients of $\omega-d\eta$ are independent to $x_n$. Thus the theorem follows from the inductive assumption.

\qed

The following theorem is a nonlinear generalization of the above result:

\begin{theorem} 
Let $U$ be the unit tube of the original point of $\R^n$.
Let $\omega$ is a skew-symmetric matrix valued $1$-form on $U$. Assume that
\[
d\omega+\omega\wedge\omega=0.
\]
Then on $U$, the equation
\[
dg+\omega g=0
\]
is solvable.
\end{theorem}

{\bf Proof.} We assume that $g(0)=1$. Assume that for $r>0$, we can define $g(x_1,\cdots,x_r,0,\cdots,0)$ such that $dg+\omega g=0$ on $\{x_{r+1}=\cdots x_n=0\}$. We define $g(x_1,\cdots, x_{r+1},0,\cdots,0)$ by the following ODE:
\[
\left\{
\begin{array}{l}
\frac{\pa g}{\pa x_{r+1}}+\omega_{r+1} g=0\\
g(x_1,\cdots,x_r,0,\cdots,0) \text{ were defined. }
\end{array}
\right.
\]
To complete the proof, we need to prove that for any $k\leq r$, we have
\begin{equation}\label{b-1}
\frac{\pa g}{\pa x_k}+\omega_k g=0.
\end{equation}
Taking the derivative  with respect to $x_{r+1}$, we get
\[
\frac{\pa ^2 g}{\pa x_{r+1}\pa x_k}+\frac{\pa\omega_k}{\pa x_{r+1}}g+\omega_k\frac{\pa g}{\pa x_{r+1}}.
\]
Using the ODE, we get
\[
-\frac{\omega_{r+1}}{\pa x_k}g-\omega_{r+1}\frac{\pa g}{\pa x_k}+\frac{\pa\omega_k}{\pa x_{r+1}}g+\omega_k\frac{\pa g}{\pa x_{r+1}}.
\]
Using the fact that $d\omega+\omega\wedge\omega=0$, we can prove that the expression in~\eqref{b-1} satisfies the ODE with the zero initial value, hence must be zero.

\qed












\section{de Rham Theorem}\label{de-Rham_theorem}

The differential operator $d:\Lambda^p(M)\to\Lambda^{p+1}(M)$ satisfies the relation $d^2=0$. Using this, we can define the de Rham cohomology groups as follows:

\begin{definition}
Let
\begin{align*}
& Z^k(M)=\{\omega\in\Lambda^k(M)\mid d\omega=0\},\\
&B^k(M)=\{d\eta\mid \eta\in\Lambda^{k-1}(M)\}.
\end{align*}
Then $B^k(M)\subset Z^k(M)$, and the quotient spaces
\[
H_{DR}^k(M)=Z^k(M)/B^k(M),
\]
are called the de Rham cohomology groups.
\end{definition}


We have the following result
\begin{theorem}[de Rham] For any 
$k\geq 0$, we have
\[
H_{DR}^k(M)=H_{sing}^k(M),
\]
where $H_{sing}^k(M)$ is the singular cohomology groups.
\end{theorem}

{\bf Proof.} There are many fancy proofs of the above result. The proof provided below is the best one (and can be made rigorous).


Let $S_k(M)$ be the vector spaces generated by all dimension $k$ submanifolds of $M$ with boundaries. The map
\[
\pa: S_k(M)\to S_{k-1}(M)
\]
sends those submanifolds into their boundaries. Obviously, $\pa^2=0$. That is, the boundary of the boundary of a submanifold has no boundary.
We can define the homology groups as
\[
H_k(M)=\frac{\{V\mid \pa V=0,  V\in S_k(M)\}}{\{\pa W\mid W\in S_{k+1}(M)\}}
\]
for all $k\geq 0$. 
We {\it assume} that 
\[
H_{sing}^k(M)=(H_k(M))^*,
\]
which is a combinatorial result and the proof can be found in any book in algebraic topology.

There is a natural non-degenerated coupling between $\Lambda^k(M)$ and $S_k(M)$ by the integration
\[
\Lambda^k(M)\times S_k(M)\to \mathbb R,\qquad (\omega,V)\mapsto \int_V\omega.
\]
By the Stokes theorem, the above coupling descends to the map
\[
H^k_{DR}(M)\times H_k(M)\to\mathbb R.
\]
To complete the proof, we need to prove that the above coupling is non-degenerated. What we need to prove is that, for any $p$-form $\omega$ such that $d\omega=0$ and for any $p$-dimensional submanifold $K$ without boundary $\int_K\omega=0$, then there is a $p-1$ form $\eta$ such that $\omega=d\eta$. Let $U,V$ be two open sets and suppose that the theorem is proved for $U,V$. Let $\omega=d\eta_1$ on $U$ and $\omega=d\eta_2$ on $V$. Then on $U\cap V$, $d(\eta_1-\eta_2)=0$.
We need to prove that for any $p-1$ dimensional 
submanifold $L$ in $U\cap V$, if $\int _L(\eta_1-\eta_2)=0$, then there is a $\xi$ such that $\eta_1-\eta_2=d\xi$. Thus $\eta_1-d\xi=\eta_2$ provides the required solution.



\qed


\begin{ex} Provide the details of the proof.
\end{ex}

\section{Formal Hodge Theorem}\label{Formal_Hodge}

The de Rham  Theorem  introduced  in the last section is among the very fundamental results of
compact manifolds. However, in terms of the
concrete computation of the cohomology groups, they are not
very efficient.
The elements of the groups are classes of differential forms.

The Hodge Theorem provides the way to pick up the best possible candidates in the classes. It asserts that, among the class of the form
$\omega+d\eta$, there is a unique representative, called
the {\it harmonic form}, which is better than other representatives of the class in the
following sense.




We consider the following variational problem for a fixed
element $[a]$ of $\Lambda^p(M)$: finding an $\eta_0$ such
that
\begin{equation}\label{2-1}
||\eta_0||^2
=\underset{\eta\in[a]}{\rm inf}\,\int_X\langle\eta,\eta\rangle dV_g.
\end{equation}
By definition,
each $\eta\in [a]$ can be represented  by
\[
\eta=\xi_0+d\xi_1.
\]
If the space of $\xi_1$ were of finite dimensional, then since
$||\eta||^2\geq 0$, we could have found a $\xi_1'$
such that
\[
||\xi_0+d\xi_1'||^2=\underset{\eta\in[a]}{\rm inf}
||\eta||^2,
\]
and the variational problem~\eqref{2-1} would have been solved.

Now let's {\it assume}
that there is a unique $p$  form $\eta_0$
that solves  problem~\eqref{2-1}. Let $\xi_1$ be an arbitrary
$p-1$ form.  Then we have the
inequality
\[
||\eta_0+\eps d\xi_1||^2\geq||\eta_0||^2.
\]
Since $\eps$ is arbitrary, we have
\[
(\eta_0,d\xi_1)=0.
\]
Then
\begin{equation}\label{2-2}
\delta\eta_0=0.
\end{equation}
Then from~\eqref{2-2}, $\eta_0$ is harmonic, i.e.,
$\Delta\eta_0=0
$. Thus the solution of the problem~\eqref{2-1} must be
harmonic.
 For any element $\eta$
in a fixed cohomologyical class $[a]$, we have the following
Hodge decomposition
\[
\eta=\eta_0+d \eta_1,
\]
where $\eta_0$ is the harmonic form, and $\eta_1$ is a $p-1$ form.
In general, for arbitrary $p$ form $\eta$, we have the following Hodge
decomposition:
\[
\eta=\eta_0+d\xi_1+\delta\xi_2,
\]
where $\eta_0$, $\xi_1$, and $\xi_2$ are smooth 
$p$, $p-1$, and $p+1$ forms, respectively,
and $\eta_0$ is harmonic. 

\section{The Hodge theorem}\label{State_Hodge}
Unfortunately, it is far from trivial that the variational
problem~\eqref{2-1} can be solved. The PDE theory, especially
the elliptic regularity theory kicks in here to make the above
arguments  rigid. 




\begin{theorem}[real Hodge Theorem] 
Let $M$ be a compact orientable Riemannian manifold. Let $p\geq 0$. Define
\[
\mathcal
H^p(M)=\{\phi\in\Lambda^p(M))\mid \Delta\phi=0\}.
\]
Then
\begin{enumerate}
\item $\dim \mathcal H^p(M)<+\infty$;
\item Let $\eta$ be an arbitrary smooth $p$-form. Then
we have
\[
\eta=\eta_0+d\eta_1+\delta\eta_2,
\]
where $\eta_0$ is a harmonic $p$-form; $\eta_1$ and $\eta_2$
are $(p-1)$ and $(p+1)$ forms, respectively,
\item $H^p_{DR}(M)=\mathcal
H^p(M)$ for $p\geq 0$.
\end{enumerate}
\end{theorem}


\section{Proof of the Hodge Theorem}\label{Proof_Hodge}
The proof of the (real) Hodge theorem heavily
depends on the analysis of the Laplacian
$\Delta$. The three basic PDE
tools we are going to use  are the Sobolev Lemma, the
Rellich Lemma, and the G\aa rding inequality. For the first
two lemmas, we need to introduce the  Sobolev $s$-norms. The
proof of the G\aa rding inequality
depends on the Weitzenb\"ock formula.



Let $M$ be a compact Riemannian manifold.
 $X=\bigcup U_\alpha$ be a  finite cover such that
each $(U_\alpha, \{x_\alpha^1,\cdots, x^{n}_\alpha\})$
is a real  local coordinate system.

Let $\{\rho_\alpha\}$ be the partition of unity subordinating
to the cover $\{U_\alpha\}$. Let $S$ be a smooth
function. For
any nonnegative integer
$s$, define the Sobolev norm of $S$ to be
\[
\|S\|_s^2=\sum_\alpha\sum_{|K|\leq s}\int_M
|D_\alpha^K (\rho_\alpha S)|^2 dV_M,
\]
where $K=(k_1,\cdots,k_{n})$ is the multiple index;
$|K|=\sum k_i$; and
\[
D_\alpha^K S=\frac{\pa^{k_1+\cdots+k_{n}}}{
\pa (x_\alpha^1)^{k_1}\cdots \pa (x_\alpha^{n})^{k_{n}}}
S.
\]

Obviously, the definition of the Sobolev norms depend on
the choice of the cover $\{U_\alpha\}$ and the partition
of unity $\{\rho_\alpha\}$. 

\begin{ex}
Prove that the Sobolev norms
$||\quad||_s$, defined by different covers and partitions
of unity, are equivalent. In particular, the norm $\|\quad\|_0$
is equivalent to the $L^2$ inner product in~\eqref{p-3}.
\end{ex}

There is a way to generalize the notation of Sobolev norms
$||\quad||_s$
from nonnegative integers $s$ to
 any nonnegative numbers $s$, using pseudo-differential
operators, or using the ``elementary'' definition as follows:

We first define the Sobolev $s$-norms for $\rho_\alpha
S$ on each $U_\alpha$. Since $U_\alpha$ is a coordinate
patch, we can assume, without loss of  generality, that
$U_\alpha$ is an open set of a  torus $T=T_{n}=\R^{n}/\Z^{n}$.
Thus the section  $\rho_\alpha S$ can be extended as a
smooth $\C^r$-valued function on $T$. Let
\[
\rho_\alpha S=\sum_{\xi\in\mathbb Z^{n}}S_\xi e^{i\langle\xi,x\rangle}
\]
be the Fourier expansion of $S_\alpha$. We define
\[
||\rho_\alpha S||_s^2=\sum_{\xi\in\mathbb Z^{n}}(1+||\xi||^2)^s
|(\rho_\alpha S)_\xi|^2.
\]
The Sobolev $s$-norm of $S$ is defined as
\[
||S||_s^2=\sum_\alpha||\rho_\alpha S||_s^2.
\]
Define $\mathcal H_s^{p}(M)$ to be the completion of the
smooth
$p$-forms under the norm $||\quad||_s$.
As before, it is independent of the
choice of the cover and the partition of unity.

\begin{theorem}[Sobolev Lemma]\index{Sobolev lemma}
Using the above notations, we have
\[
\underset{s}{\bigcap}\,\mathcal H^{p}_s(M)=\Lambda^{p}(M).
\]
\end{theorem}



\begin{proof} The theorem being local, we assume that $S$
is a $\C^r$-valued smooth function of $T$. By definition,
for any $s>0$, we have
\[
\sum_{\xi\in \mathbb Z^{n}} (1+||\xi||^2)^s |S_\xi|^2\leq C
<\infty.
\]
Let $K=(k_1,\cdots,k_{n})$ be a multiple index. We consider
the Fourier expansion of
\[
D^K S\sim \sum_{\xi\in \mathbb Z^{n}} (i)^{|K|}\xi_1^{k_1}\cdots\xi_{n}^{k_{n}}
S_\xi e^{i\langle\xi,x\rangle}.
\]
By definition, if $S\in\bigcap_s \mathcal H^{p}_s(M)$, then
\[
\sum_{\xi\in \mathbb Z^n} |\xi_1^{k_1}\cdots \xi_{n}^{k_{n}}|^2
 |S_\xi|^2\leq  \sum_{\xi\in \mathbb Z^{n}} (1+||\xi||^2)^s |S_\xi|^2<+\infty
\]
for any $K$.
Thus $D^K S$ is well-defined. Since the differential operators $D^K$ are closed operators, we conclude that $S$ is $K$ differentiable and hence prove the result.

\end{proof}

In order to prove the Rellich Lemma, we  give
the following definition of
{\it compact operators}\index{compact operator}.

\begin{definition}
Let $\mathcal B_1,\mathcal B_2$ be two Banach spaces.
Let $A: \mathcal B_1\rightarrow\mathcal B_2$ be a linear
operator from $\mathcal B_1$ to $\mathcal B_2$. $A$ is a
compact
operator, if the image of the unit ball of $\mathcal B_1$
under $T$
is sequential compact in $\mathcal B_2$. In other word, if
$\{x_i\}$ is a bounded sequence of $\mathcal B_1$,
then  a subsequence of $\{Ax_i\}$ converges to some point
of $\mathcal B_2$.
\end{definition}

The following is the basic fact about a self-adjoin compact operator. 

\begin{theorem}
Let $\mathcal H$ be a Hilbert space and let $A: \mathcal H\to\mathcal H$ be  a compact operator. Then the essential spectrum 
$\sigma_{\rm ess}(A)\subset\{0\}$. Moreover, if $\sigma_{\rm ess}(A)=\emptyset$, then $\mathcal H$ is the direct sum of eigenspaces of each non-zero eigenvalues, that is, if $E_m$ is the eigenspaces of the eigenvalue $\lambda_m$. Then we have
\[
\mathcal H=\bigoplus_m E_m.
\]
\end{theorem}

\begin{proof}
If $\lambda\in\sigma_{rm ess}(A)$ and $\lambda\neq 0$. Then for any $\eps>0$, by the Weyl's criterion, there exists an infinite dimensional vector space $G_\eps$ such that for any $\phi\in G_\eps$, we have
\[
\|A\phi-\lambda\phi\|<\eps.
\]
We choose an orthonormal set $\{\phi_i\}\subset G_\eps$. Since $A$ is compact, we may assume that $A\phi_i$ is convergent. Assume that for $i,j$ big enough but $i\neq j$, we have
\[
\|A\phi_i-A\phi_j\|\leq \eps.
\]
Then we have
\[
\lambda\|\phi_i-\phi_j\|\leq \|A(\phi_i-\phi_j)\|+\|A\phi_i-\lambda\phi_i\|+\|A\phi_j-\lambda\phi_j\|\leq 3\eps.
\]
Since $\phi\perp\phi_j$, we have $\|\phi_i-\phi_j\|=\sqrt 2$. Thus we have a contradiction 
\[
\sqrt 2\,\lambda\leq 3\eps.
\]
This proves the first part of the theorem. If $\sigma_{\rm ess}(A)=\emptyset$, then is follows from the spectrum  theorem that the whole space is the direct sum of the eigenspaces. 

\end{proof}

\begin{theorem} [Rellich Lemma] \index{Rellich lemma}
For $s>r$, the inclusion
\[
\mathcal \mathcal H_s^p(M) \rightarrow \mathcal H^{p}_r (M)
\]
is compact.
\end{theorem}

\begin{proof} Similar to the previous theorem, we can
work on a torus $T$. Assume
that $\{ u_k\}$ is a bounded sequence of $\mathcal
H^{p}_s(T)$. We wish to prove that a subsequence
of $\{u_k\}$ will converge in the $||\quad ||_r$ norm.

Let $\{(u_k)_\xi\}$ be the Fourier coefficients of $u_k$ for
$k\geq 1$. Then
\begin{equation}\label{2-5}
\sum_{\xi\in \mathbb Z^{n}} (1+||\xi||^2)^s |(u_k)_\xi|^2\leq C
\end{equation}
for some constant $C$ independent of $k$.
For any fixed $\xi$, a subsequence of
$(1+||\xi||^2)^{s/2} (u_k)_\xi$ converges. By the standard
diagonalization, we can find a subsequence of $\{u_k\}$,
which we still denote as $\{u_k\}$ for abusing of notations,
such that for each
$\xi$,
$(1+||
\xi||^2)^{s/2} (u_k)_\xi$ is convergent.

We prove that $\{u_k\}$ is a Cauchy sequence under the normal
$||\quad||_r$. For any $\eps>0$, we choose $R$ large
so that
\begin{equation}\label{2-26}
\frac{1}{(1+R^2)^{s-r}}<\eps.
\end{equation}
We consider all $\xi$ with $||\xi||\leq R$.
There are only finitely many such $\xi$'s.
By the
assumption on the sequence $\{u_k\}$, there is an $N>0$,
such that
\[
\sum_{||\xi||\leq R} (1+||\xi||^2)^r
|(u_k)_\xi-(u_l)_\xi|^2<\eps
\]
for $k,l>N$. Thus we have
\[
||u_k-u_l||_r^2\leq\eps+\sum_{||\xi||>R}
(1+||\xi||^2)^r |(u_k)_\xi-(u_l)_\xi|^2.
\]
Using~\eqref{2-5} and ~\eqref{2-26}, we have
\[
||u_k-u_l||^2_s<(1+2C)\eps,
\]
and the theorem is proved.

\end{proof}


\begin{ex} Let $M$ be a compact manifold. Define two Banach spaces of $M$. Let $\mathcal C^0(M)$ be the space of continuous functions with the maximum norm, and let $\mathcal C^1(M)$ be the space of $\mathcal C^1(M)$ functions with the norm
\[
\max(|f|+|\nabla f|).
\]
Prove the Arzel\'a-Ascoli Theorem: the inclusion map
\[
\iota: \mathcal C^1(M)\to \mathcal C^0(M)
\]
is a compact operator.
\end{ex}



The standard method in proving the above theorem, like in that of the Rellich Lemma, is to use the diagonal subsequence method, which is widely used. In the following, we give a different proof, using the Tychonoff Theorem.

We consider $F(M)$, the space of all bounded functions with bound $1$ on $M$. By the identification
\[
F(M)=\prod_{x\in M}\mathbb [-1,1],
\] and by the Tychonoff Theorem, we know that $F(M)$ is compact under the product topology. On the other hand, on $\mathcal C^0(M)$, we can define the so-called compact-open topology, whose subbase composed of the set of the form
\[
C(K,U)=\{f\mid F(K)\subset U\},
\]
where $K$ is a compact subset of $M$ and $U$ is an open set of $\mathbb R$.
We can prove that when we restrict to the space $\mathcal C^1(M)$, the product topology is the same as the compact-open topology. On the other hand, on $\mathcal C^0(M)$, compact-open topology defines the same topology as in the Banach space $\mathcal C^0(M)$, and the theorem is proved.

\begin{ex} Provide the details.
\end{ex}

 




Next we shall introduce the G\aa rding inequality. 
Let
\[
\phi,\psi\in\Gamma(X,\Lambda^p(M)),
\]
and let
\[
\mathfrak D (\phi,\psi)=(\phi,\psi)
+(d\phi, d\psi)+(\delta\phi,
\delta\psi)=(\phi, (1+\Delta)\psi).
\]
Then $\mathfrak D(\phi,\psi)$ defines an inner product
on $\Lambda^p(M)$. Apparently we have the
inequality
\[
\mathfrak D(\phi,\phi)\leq C||\phi||_1^2
\]
for some constant $C$. The G\aa rding inequality states
that the inverse inequality is also true so that
the norms $\sqrt{\mathfrak D(\phi,\phi)}$ and $||\phi||_1$
are equivalent.




\begin{theorem} [G\aa rding Inequality]\index{G\aa rding Inequality}
For $\phi\in\Lambda^p(M)$, we have
\begin{equation}\label{2-10}
||\phi||_1^2\leq C\,\mathfrak D(\phi,\phi),
\end{equation}
for some constant $C>0$.
\end{theorem}



\begin{proof} The proof of the
inequality depends  on the Weitzenb\"ock formula, Theorem~\ref{thm6}.
Let $\phi\in\Lambda^p(M)$. Then
\begin{equation}\Delta\phi=-\nabla^*\nabla\phi+E(\phi),
\end{equation}
where $E$ is a zero-th differential operator.
Using  integration
by parts, we get
\[
(\Delta\phi,\phi)\geq\int|\nabla\phi|^2-C(\phi,\phi).
\]
It follows that
\[
\mathfrak D(\phi,\phi)\geq (\phi,\phi)+\frac{1}{2C}(\Delta\phi,\phi)\geq
\frac{1}{2C}\int|\nabla\phi|^2+\frac 12 (\phi,\phi),
\]
and the inequality is proved.



\end{proof}

\begin{proof}[Proof of the Hodge Theorem] For
$\phi\in\Lambda^p(M)$, we define a linear
functional
\[
l(\psi)=(\phi,\psi)
\]
for $\psi\in\Lambda^p(M)$.
With respect to the inner product $\mathfrak D(\phi,\psi)$,
$l$ is bounded:
\[
|l(\psi)|\leq ||\phi||_0\sqrt{\mathfrak D(\psi,\psi)}.
\]
Thus by the Riesz representation theorem, there is an $\eta$
such that
\begin{enumerate}
\item $(\phi,\psi)=(\eta, (1+\Delta)\psi)$ for any
$\psi\in\Gamma(X,\mathcal A^{p,q}(E))$;
\item $||\eta||_1^2\leq C\mathfrak D(\eta,\eta)<+\infty$.
\qquad \text{(G\aa rding inequality)}
\end{enumerate}

Let $A: \Lambda^p(M)\rightarrow\Lambda^p(M)$ be the  linear operator defined by sending
$\phi\in\Lambda^p(M)$ to the unique $\eta$ defined
above. Since $A$ is a bounded operator, it can be extended to an operator on $L^2(M)$. Then since
$\eta$ is actually in $\mathcal H^{p}(M)$, the operator
$A$ is a compact operator by the Rellich Lemma.
 On the other hand, it is
not hard to see that  $A$ is a self-adjoint
operator\footnote
{Since $A$ is a bounded operator, in order to prove that
$A$ is self-adjoint, we just need to verify the
equality $(A\phi,\psi)=(\phi,A\psi)$ for smooth $\phi$ and
$\psi$, which follows easily from integration by parts.}.

According to the spectral theorem for compact, self-adjoint
operators, there is a Hilbert-space decomposition
\[
\Lambda^p(M)=\bigoplus_m E(\rho_m),
\]
where $\rho_m$ are the eigenvalues of $A$ and $E(\rho_m)$
are the finite-dimensional eigenspaces corresponding to the
eigenvalue $\rho_m$.
Since
$A$ is one-to-one, all $\rho_m\neq 0$.

We claim that
\[
\mathcal H^p(M)=E(1),
\]
where $\mathcal H^{p}(M)$ is the space of harmonic forms.
To see this, we assume that $\phi\in\mathcal
H^{p}(M)$,
and $A\phi=\phi$.
By the Weitzenb\"ock formula, $\Delta$ is an elliptic
operator.
Using the Sobolev Lemma and Schauder estimate, we can prove that $\phi$ is smooth. Thus $(1+\Delta)\phi=\phi$ and $\phi$ is harmonic.
Since $E(1)$ is of finite dimensional, we have
\[
\dim\,\mathcal H^{p}(M)<+\infty.
\]
This proves the first assertion of the theorem. To prove the
second assertion of the theorem, we note that the biggest
possible eigenvalue of $A$ is $1$. By the property of
compact operators, there is a gap between the eigenvalue $1$
and the rest of the eigenvalues. Thus
if $\phi\in\mathcal H^{p}(M)^\perp$ and if $\phi$ is
smooth, we have
\[
||\Delta\phi||_0\geq\eps||\phi||_0
\]
for some $\eps>0$. Thus we can define the inverse operator
$G$ on $\mathcal H^{p}(M)^\perp$, called the Green's
operator, such that
$G=\Delta^{-1}|_{\mathcal H^{p}(M)^\perp}$.

If we compare $G$ to the operator $A$ (they  have
the same eigenspaces), we shall see that
$G$ is also a compact  operator because $||G||\leq
C||A||$ for some constant $C$.
Furthermore, by the elliptic regularity, $G$ maps
smooth $p$-forms to smooth  $p$
forms.
Using this
fact, the Hodge decomposition is given by
\[
\phi=\eta_0+d\delta G\phi+\delta dG\phi.
\]
This completes the proof of the Hodge Theorem.

\end{proof}

\section{More about elliptic regularity}\label{Ellipticity}
Let $M$ be a compact Riemannian manifold and let $\phi\in \mathcal H^p_1(M)$ be a $p$-form. We say 
$\phi$ is a weak solution of the equation
\[
(\Delta+I)\phi=\psi
\]
for $\psi$ a $C^\infty$ $p$-form, if
\begin{enumerate}
\item $\phi\in \mathcal H^p_1(M)$;
\item for any $\eta$ a smooth $p$-form, we have
\[
(\phi,(\Delta+I)\eta)=(\psi,\eta).
\]
\end{enumerate}

In this section, we prove the following result.
\begin{theorem} Let $\phi$ be a weak solution of the above equation, then $\phi$ must be smooth.
\end{theorem}

\begin{proof}

Let $\rho$ be a smooth function. Then by the Weitzenb\"ock formula, we have
\begin{equation}\label{rst}
(\Delta+I)(\rho\eta)=\rho(\Delta+I)\eta+J(\eta),
\end{equation}
where $J$ is some first order differential operator (depending on $\rho$) on the space of $p$-forms. Thus we have
\begin{equation}\label{pqr}
(\rho\phi, (\Delta+I)\eta)=(\psi,\rho\eta)-(\phi,J(\eta))
\end{equation}
for any smooth form $\eta\in\Lambda^p(M)$.

Now we specify our choice of $\rho$. Let $x\in M$ be a fixed point. We choose a local normal coordinate system $(U,(x_1,\cdots,x_n))$. We choose a smooth function $\rho$ such that
the support of $\rho$ is within $U$, and is constant $1$ in a smaller neighborhood $V$ of $x$ such that $\bar V\subset U$. Without loss of generality, we may assume that $\rho\phi,\eta$ are $p$-forms of a  torus $T^n$.

Let $(\,\,,\,\,)_0$ be the inner product on $\Lambda^p(T^n)$ induced by the flat Riemannian metric and let $||\quad||_0$ be the corresponding norm. Then there is a constant $C>0$ such that
\[
C^{-1}||\quad||_0\leq ||\quad||\leq C||\quad||_0.
\]
In fact, since we choose the local normal coordinate system, we may assume that near $x$, the Riemannian metrics are very close. 

Let $A$ be the zero-th differential operator such that
\[
(\rho\phi,(\Delta_0+I)\eta)_0=(\rho\phi, A(\Delta_0+I)\eta).
\]
Let $P$ be the operator
\[
P=A(\Delta_0+I)-(\Delta+I),
\]
where $\Delta_0$ is the Laplacian with respect to the flat metric of $T^n$. Then we have
\[
(\rho\phi,(\Delta_0+I)\eta)_0=(\rho\phi, P\eta)+(\psi,\rho\eta)-(\phi, J(\eta)).
\]



Let
\[
\rho\phi\sim\sum_{\xi\in\mathbb Z^n}a_\xi e^{i\langle x,\xi\rangle}
\]
be the Fourier expansion of $\rho\phi$, where $a_\xi$ are  $p$-forms.
Let $R,s$ be  large real numbers and let 
\[
\eta_R=\sum_{\|\xi\|<R}a_\xi e^{i\langle x,\xi\rangle}(1+||\xi||^2)^{s}
\]
Since
\[
\Delta_0=-\sum_{j=1}^n\frac{\pa^2}{\pa x_j^2},
\]
We have
\[
\Delta_0e^{i\<x,\xi\>}=\|\xi\|^2e^{i\<x,\xi\>}.
\]
Then we have
\begin{equation}\label{12}
\sum_{\|\xi\|<R}|a_\xi|^2(1+\|\xi\|^2)^{s+1}=(\eta_R,(\Delta_0+I)\eta_R)_0.
\end{equation}

Let $\tilde\Delta$ be the raw Laplacian. Let
\[
\tilde P=A(\Delta_0+I)-(\tilde\Delta+I).
\]
Then we have
\[
(\rho\phi, P\eta)=(\rho\phi,\tilde P\eta)+(\rho\phi, E\eta),
\]
where $E$ is the zero-th operator in the Weitzenb\"ock formula. 

We claim that
\begin{enumerate}
\item $|(\rho\phi, \tilde P\eta_R)|\leq\eps\|\phi\|^2_{\mathcal H_{(s+1)/2}}$ for $0<\eps\ll1$;
\item $|(\rho\phi, E\eta_R)|\leq C\|\phi\|^2_{\mathcal H_{s/2}}$;
\item $|(\psi, \rho\eta_R)|\leq C \|\phi\|_{\mathcal H_{s/2}}$;
\item $|(\phi, J(\eta_R))|\leq C\|\phi\|^2_{\mathcal H_{s/2+1/4}}$.
\end{enumerate}

We will postpone the proof of the above claim in the following. Now let's assume it is correct. 
Letting $R\to\infty$ in~\eqref{12}, we have
\[
\|\phi\|^2_{\mathcal H_{(s+1)/2}}\leq \eps \|\phi\|^2_{\mathcal H_{(s+1)/2}}+C\|\phi\|^2_{\mathcal H_{s/2}}+C\|\phi\|_{\mathcal H_{s/2}}+C\|\phi\|^2_{\mathcal H_{s/2+1/4}}.
\]
This by induction, $\phi\in \mathcal H_s(M)$ for any $s$, and hence by the Sobolev Lemma, $\phi$ has to be smooth.

It remains to prove the claim. We shall only prove (1), the others begin similar.

$\tilde P$ is a differential operator with small coefficients. Let
\[
\tilde P=a_{ij}\frac{\pa^2}{\pa x_i\pa x_j}+b_i\frac{\pa}{\pa x_i}+c.
\]
Then we have
\[
\tilde \eta_R=\sum_{\|\xi\|<R}a_\xi e^{i\langle x,\xi\rangle}(1+\|\xi\|^2)^s\cdot F(x,\xi),
\]
where
\[
F(x,\xi)=-a_{ij}\xi_i\xi_j+\sqrt{-1}b_i\xi_i+c.
\]
We write
\[
\overline{(\rho\phi,\tilde P\eta_R)}=\sum_{\tilde\xi\in\mathbb Z^n} \sum_{\|\xi\|<R}\bar a_{\tilde\xi} \,a_\xi (1+\|\xi\|^2)^s\, \int_M F(x,\xi) e^{i\langle x,(\xi-\tilde\xi)\rangle}.
\]
By the Riemann-Lebesgue lemma, we have
\[
\left|\int_M F(x,\xi) e^{i\langle x,(\xi-\tilde\xi)\rangle}\right|\leq C\frac{1+\|\xi\|^2}{\|\xi-\tilde\xi\|^K}
\]
for any $K>0$. Below we shall take $K=(s+1)/2$ and use the inequality that
\[
(1+\|\xi\|^2)\cdot\|\xi-\tilde\xi\|\leq C\,(1+\|\tilde\xi\|^2).
\]
Thus we have
\[
\begin{split}
&
\left|\sum_{\xi\neq\tilde\xi}\bar a_{\tilde\xi} \,a_\xi (1+\|\xi\|^2)^s\, \int_M F(x,\xi) e^{i\langle x,(\xi-\tilde\xi)\rangle}
\right|\\
&\leq\sum_{\xi\neq\tilde\xi}\bar a_{\tilde\xi} \,a_\xi (1+\|\xi\|^2)^{(s+1)/2}\cdot (1+\|\tilde\xi\|^2)^{(s+1)/2},
\end{split}
\]
and this completes the proof of the claim.
\end{proof}

