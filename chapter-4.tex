
\chapter{Heat kernel and Green's functions on complete manifold}\label{Heat_kernel}
\section{Heat kernel}\label{Heat_kernel_def}
Let $M$ be a Riemannian manifold. The Laplace operator $ \Delta $ can be extended as a densely defined 
self-adjoint operator. Thus by the spectrum theorem, we can write
%
\[ \Delta = \int \lambda d E\]
%
where $ E$ is the corresponding spectrum measure. Using functional analysis, we define 
%
\[ e ^{- \Delta t} = \int ^\infty _0 e ^{- \lambda t } \, dE\]
%
to be the heat operator, and 
%\
\[\int^\infty _ 0 e ^{-\Delta t} \, dt \]
%  
to be the Green's operator.

The heat semi-group, or the heat operator, has a kernel.

\begin{theorem} Let $M$ be a complete Riemannian, then there is a heat kernel $ H (x, y, t) \in C^\infty (M \times M \times \mathbb{R} ^+ )$ such that 
%
\[ (e ^{- \Delta t} f) (x) = \int_M  H (x, y, t )f (y) \]
%
$ \forall f \in L^2 (M) $ such that 
%
\begin{enumerate}
\item    [{\rm (1)}]  $H(x, y, t) = H (y, x, t)$  
\item   [{\rm (2)}]  $ \lim_{t \rightarrow 0^{+} } H (x, y, t) = \delta _x (y)$
\item   [{\rm (3)}]  $ \left( \Delta - \frac{\partial }{\partial t} \right) H = 0$
\item   [{\rm (4)}]  $ H(x, y, t) =  \int H (x, z, t - s) H (z, y, s) \, dz.$
\end{enumerate}
\end{theorem}

\begin{remark}
Let $M$ be a compact manifold and let $ \{ f_i\} $ be an orthonormal basis of eigenfunctions. Let $ \lambda _i $ be the corresponding eigenvalues. Then
%
\[ H(x, y, t) = \sum^\infty_{k=1} e ^{-\lambda _{k} t} f _k (x) f_k  (y) .\]

As in the case of harmonic functions, for positive solutions of the heat equations
%
\[ \left( \Delta - \frac{\partial }{\partial t} \right) u= 0\]
%
we also have the differentiable Harnack inequality. The  theorem is as follows:
\end{remark}

\begin{theorem} Let $M$ be a compact Riemannian manifold  with boundary, Ric$(M) \geq 0$. If $ \partial M \neq \emptyset $, we assume that $ \partial M $ is convex. In this case, we assume that $ u (x, t)$ satisfies the Neumann boundary.
%
\[ \frac{\partial u}{\partial n} = 0\]
%
on $ \partial M x (0, \infty) $, where $   \frac{\partial }{\partial n}  $ is the outer normal direction. Then on $ M x (0, \infty)$, we have 
%
\[ \frac{|\nabla u|^2}{u^2} - \frac{ut}{u} \leq \frac{n}{2t} .\]
\end{theorem}

For manifold with   Ric$(M) \geq - k $, the estimates are more complicated but the same principle applies.

The Harnack inequalities follows from the gradient estimates:

Using the gradient estimates and the Harnack inequality we obtain:

\begin{theorem}Let $ M$ be a complete manifold without boundary, Ric$(M) \geq - k , k \geq 0$. Let $ H (x, y, t) $ be the fundamental solution of the heat equation
%
\[ \left( \Delta - \frac{\partial }{\partial t} \right)  u (x, t) = 0 .\]
%
Then for any $ \delta \in (0, 1) $, we have 
%
\[H(x, y, t) \leq C(\delta, n) V^{-\frac{1}{2}} _x ( \sqrt{t}) V_y ^{-\frac{1}{2}} ( \sqrt{t}) \exp 
\left( - \frac{r^2 (x, y)}{(4 + \delta)t} + C_1 \delta k t\right) .\]
\end{theorem}

For the proof, see the book of Yau and Schoen. We give some applications of the above theorem.

\begin{theorem}[Gromov] Let $ M$ be a compact manifold without boundary, let   Ric$(M) \geq 0 $ and $ d$ be the diameter of $M$. Then 
%
\[ \lambda _k \geq \frac{C(n)}{d^2} ( k + 1)^{\frac{2}{n}} \]
%
where $ \lambda _k$ is the $kth$ eigenvalue of the manifold.
\end{theorem}

\noindent{\bf Proof} We set $ x = y $ and $ k = 0 $ in the above theorem. Then we have
%
\[H(x, x, t) \leq C(n) V^{-1} _x (\sqrt{t}).\]
%
Since 
%
\[ H(x, y, t)= \sum^\infty_{k=1} e ^{-\lambda k t} f_k (x) f _k (y) \]
%
we have 
\[ H(x, x, t)= \sum^\infty_{k=1} e ^{-\lambda _{k} t} . \]
%
As a result we have 
\[ \sum^\infty_{k=1} e ^{-\lambda k t} \leq C(n) \int_M V^{-1} _x (\sqrt{t}) \, dx. \]
%
If $ \sqrt{t} \geq d $, then $  V^{-1} _x (\sqrt{x}) = {\rm vol} (M)$, if $ \sqrt{t} \leq d $, then 
by the Bishop volume comparison theorem 

%
\[  \frac{V_x  (\sqrt{t}}{ V _x (d)} \geq \left( \frac{\sqrt{t}}{d}\right) ^n .\]
%
Thus 
%
\begin{equation}\sum^\infty_{k=1} e ^{-\lambda _{k} t} \leq C(n)\left\{ \begin{array}{ll} \left(\frac{d}{\sqrt{t}}\right) ^n &t \leq d^2\\
 1  &t > d^2 \end{array} \right. .\tag{$\star$}\end{equation}
%
For fixed $k$, by the monotonicity, we have 
\[ke ^{-\lambda _{k} t} \leq C(n)\left\{ \begin{array}{ll} \left(\frac{d}{\sqrt{t}}\right) ^n &t \leq d^2\\
 1  &t > d^2 \end{array} \right. .\]
%
We let
%
\[ \sqrt{t} = k ^{-\frac{1}{n}} d \]
%
and the result follows.

We can compare the above eigenvalue estimate with the result of Cheng-Li.

From ($\star$), we can get 
%
\[ \sum^\infty_{k=1} e ^{- \lambda _{k} t} \leq C t ^{-\frac{n}{2}}\;\; t \leq d^2\]
%
where $C$ is the absolute $d$  constant. On the other hand, if we consider
%
\begin{eqnarray*}
\frac{\partial }{\partial t} \int_M H (x, y, t) ^2 d y &= & 2 \int_M H (x, y, t) \Delta H (x , y, t) \, dy\\ &  =&- 2 \int | \nabla y H (x, y, t)  | ^2 \, dy\\
&  \leq & - 2 C \left( \int_M |H (x, y, t) | ^{\frac{2n}{n-2}} \, dy \right) ^{\frac{n-2}{n}} \end{eqnarray*}
%
where $C$ is the Sobolev constant. Since for any $ t > 0 $
%
\[\int H (x, z, t) \, d z \leq 1.\]
%
So we have 
\[ \left|\int _M |H | (x, y, t) | ^{\frac{2n}{n-2}} \, dy \right| ^{\frac{n-2}{n}}\geq \left( \int_M |H (x, y, t) | ^{2} \, dy \right) ^{\frac{2+n}{n}}\]
\qed

\noindent({\bf Proof.} Let $ f = H (x, y, t) $. Then 
\[ \int f ^2 = \int f ^{\frac{2n}{n+2}+ \frac{4}{n+2}}\leq \left(  \int f^{\frac{2n}{n-2}}\right) ^{\frac{n+2}{2n}}\left. \left( \int f \right) ^{\frac{4}{n+2}}\right).\]
Thus we have 
%
\[\frac{\partial }{\partial t}\int _M H(x, y, t)^2 \, dy  \leq - C \left( \int_M |H (x, y, t) ^2 | dy \right) ^{\frac{2+n}{n}}.\]
%
Since
%
\[\lim_{t \rightarrow 0} H (x, x, t) = \infty.\]
%
Thus 
%
\[ H (x, x, 2t) \leq \left( \frac{4}{n} Ct \right) ^{-\frac{n}{2}}.\]
%
Using the same method as before, we have 
%
\[\lambda _k \geq C \left( \frac{k}{{\rm vol} (M)}\right) ^{\frac{2}{n}}.\]
%

The second application of the heat kernel estimate is the following resolvent estimate.

\begin{theorem} For any $ \beta > 0 $ there is $ n \in N, \alpha < 0 , C \subset \infty $ such that the integral kernel $ g _{\alpha} ^{(\frac{n}{2})}(x, y) $ of $(\Delta - \alpha) ^{-\frac{n}{2}} $ satisfies.
%
\[g _{\alpha} ^{(\frac{n}{2})}(x, y)  \leq C \varphi (x) ^2 e ^{-\beta d(x, y)}\]
%
where $ \varphi (x) = (v (B_1 (x))  ^{-\frac{1}{2}} $.
\end{theorem}
\begin{proof} First, using the heat kernel $ H (x, t)$ we can write 
%
\[g _{\alpha} ^{(\frac{n}{2})}(x, y) = C _2 \int ^\infty_0 H (x, y, t) t  ^{ \frac{n}{2}-1   } e ^{\alpha t} \, dt.\]
%
By the volume comparison theorem, we have 
%
\begin{eqnarray*}
v ^{ \frac{n}{2}} (B_{\sqrt{t}}(x)) v ^{ \frac{-1}{2} }(B_{\sqrt{t}}(y))&\leq & C_2 \varphi (x)^2 \sup (1, t  ^{-\frac{n}{2}})\\
 &\times & e ^{\beta _{2} d (x, y)}. \end{eqnarray*}
By an element calculation, for any $\beta _3 > 0 $, we have 
%
\[\exp \left( - \frac{d^2 (x, y)}{C_1 4t} \right) \leq \exp (  \beta_3 d (x, y) ) \exp (C_1 \beta ^2_3 t ). \]
%
Thus for any $ \beta > 0 $, there exists $ \alpha < 0 , C_3 < \infty $ such that 
%
\[ H (x, y, t) \leq C_3 \varphi ^2 (x) e ^{-\beta _{2} d (x, y)} \sup \{ t  ^{ -\frac{n}{2} } , 1 \} t  ^{ - (\alpha + 1 ) t} \]
%
which implies 
%
\[g _{\alpha} ^{(\frac{n}{2})}  (x, y) \leq C_t \phi ^2 (x)   e ^{-\beta  d (x, y)}.\]
%
Cheeger-Yau's heat kernel comparison theorem. 
\end{proof}

\begin{theorem} Let $M$ be a complete Riemannian manifold such that  Ric$(M) \geq 0$. Fixing $ x \in M , r_0 > 0 $. The heat kernel $ H (x, y, t) $ in $ B(x, r_0) $ and the heat kernel $ \varepsilon (r (x, y) , t) $ in $ V (k, r_0) $ satisfies the following
%
\[ \varepsilon (r (x, y) , t) \leq H (x, y, t). \]
%
(For both Dirichlet and Neumann conditions).
\end{theorem}
\begin{proof} Using the property of the heat kernel, we have 
%
\begin{eqnarray*}
\lefteqn{H(x, y, t) - \varepsilon (x, y, t)} \\
&&= \int ^t _0 \int _{B (x_1 r_0)} \frac{d}{ds}  ( \varepsilon (x, z, t - s) H (z, y, s) ) \, d z ds\\
&&= - \int ^t _0 \int _{B (x_1 r_0) }\left( \frac{d}{ds}\varepsilon (r (x, z), t-s)\right) H(z, y, s)  \, d z ds\\
&&+ \int^t_0 \int _{B (x_1 r_0) }\varepsilon (r (x, z), t-s)\frac{d}{ds} H(z, y, s)  \, d z ds \\
&&= - \int^t_0 \int _{B (x_1 r_0)} \tilde{\Delta} \varepsilon (r (x, z), t - s) H(z, y, s)  \, d z ds\\
&&+ \int^t_0 \int _{B (x_1 r_0)}\varepsilon (r (x, z), t - s) \Delta H(z, y, s)  \, d z ds.
\end{eqnarray*}
Using the Green's formula, under either the Dirichlet or Neumann boundary condition, we have 
%
\begin{eqnarray*}
\lefteqn{\int _{B (x_1 r_0)}\varepsilon (r (x, z), t - s) \Delta H(z, y, s)  \, d z}\\
&&= \int _{B (x_1 r_0)}\Delta \varepsilon (r (x, z), t - s)  H(z, y, s)  \, d z.
\end{eqnarray*}
%
Since $ H(\xi, y, s)  > 0 $, we just need to prove that
%
\[ \tilde{\Delta} \varepsilon (r (x, z), t - s)\leq \Delta  \varepsilon (r (x, z), t - s).\]
%
This essentially follows from the Laplacian comparison theorem: Let $ x = (r, \zeta ) , \zeta \in S^{n-1} $. Then 
%
\begin{eqnarray*}
\tilde{\Delta } & = & \frac{\partial ^2}{\partial r^2} + m (r) \frac{\partial }{\partial r} , m (r) \frac{\partial }{\partial r} \log {\rm det} \sqrt{\tilde{g}}\\
\Delta & = & \frac{\partial ^2}{\partial r^2} + m (r,\zeta) \frac{\partial }{\partial r} , m (r,\zeta) \frac{\partial }{\partial r} \log {\rm det} \sqrt{{g}}.
\end{eqnarray*}
%
Since Ric$(M) \geq (n-1)_k$, using the volume comparison theorem, we have 

\[ m (r, \zeta ) \leq m (r). \]
%
Since 
\[ \frac{\partial \varepsilon}{\partial r} < 0 \]
%
we have 
%
\[ \tilde{\Delta} \varepsilon (r, t-s) \leq \Delta \varepsilon (r, t - s). \]
\qed

In the above proof, we didn't take the cut-locus into account. Using some kind of limiting process, we can overcome the difficulty.
\end{proof}

\begin{theorem}Let $M$ be a complete Riemannian manifold such that $ Ricci (M) \geq (n-1) k , n = \dim M$. We use $ B (x_0, r) $ to denote the ball centered at $ x _0$ with radius $r$. Let $ V (k, r)$ be the ball of radius $r$ in a simply connected space form. Then with the Dirichlet boundary condition, we have 
%
\[ \lambda _1 (B (x_0, r)) \leq \lambda _1 (V (k, r)).\]
\end{theorem}

\noindent{\bf Proof} Let $H(x, y, t) $ and $ \varepsilon (x, y, t) $ be the corresponding heat kernel. Then we have 
%
\begin{eqnarray*}
 H (x, y, t) &=& \sum e^{-\lambda _1t}  \varphi ^2 _1 (x) \\
\varepsilon (x, y, t) & = & \sum e^{-\tilde{\lambda} _1t}  \tilde{\varphi} ^2 _1 (x).
\end{eqnarray*}
If we let $ t \rightarrow \infty$, we get $ \lambda _1 \leq \tilde{\lambda}_1 $.
S.Y. Cheng concretely computed the upper bounds of the eigenvalue
\begin{enumerate}
  \item if   Ric$(M) \geq 0$, then $ \lambda _1 \leq \frac{2n(n+4)}{d^2} $ 
  \item if   Ric$(M) \geq n-1$, then $ \lambda _1  \leq n\frac{\pi^2}{d^2} $
  \item if   Ric$(M) \geq (n-1) (-k) $, then 
  \end{enumerate}  
\[ \lambda_1 \leq \frac{1}{4} k + \frac{C_n}{d^2} \;\;\; C_k = 2n (n+4).\]
%
Unfortunately, for Neumann boundary condition, we don't have the comparison theorem for the 1st eigenvalues {\bf directly}. Under, the Neumann boundary condition, the {\bf first} eigenvalues are always 0. The comparison theorem for that is trivial.

It is thus interesting to have the following result.

\begin{theorem}If $M$ is a compact manifold without boundary, then 
%
\[ \lambda _1 (M) \leq \lambda _1 \left(V \left(k, \frac{d}{2}\right)\right)\]
%
where $d$ is the diameter of $M$.
\end{theorem}

%%%%%%%%%% supp

Proof of a theorem of Brascamp and Lieb.

\begin{theorem} Let $ \Omega $ be a bounded convex domain of $ \mathbb{R}^n$. Let $u$ be the first Dirichlet eigenfunction. Let 
%
\[ \Delta u = - \lambda , u  \;\; u > 0 .\]
%
Then $ \log u$ must be concave.

\end{theorem}
\begin{proof} We choose any function $ u_0 > 0 , u _0 |_{\partial \Omega} = 0 $ such that $ - 
\log u_0 $ is concave. Such a function always exists. For example, we can take the convex hull of the graph of $ - \log u$.

Consider the flow
%
\[ \frac{\partial u}{\partial t}  = \Delta u + \lambda _i u \;\; \frac{u}{\partial \Omega} = 0 .\]
%
We {\bf assume} that $ u_t \rightarrow u$, the first eigenfunction. We are going to use the maximum principle. Let $T$ be the biggest number such that det$(- \Delta ^2 \log u)$ is degenerated. Thus there is an $ x_0 \in M $ and a direction $i$ such that 
%
\[ - ( \log u) _{ii} (x_0) = 0 \]
%
and for other $j, (- \log u) _{jj} (x_0) \geq 0 $. Let $ \varphi = - \log u $. Then the envolution of $ \varphi$ is 
%
\[ \frac{\partial \varphi}{\partial t}  = \Delta \varphi - |\nabla \varphi |^2 - \lambda _1 . \]
By the maximum principle, $ \varphi_{iik}  = 0 , \frac{\partial \varphi_{ii}}{\partial t} \leq 0 \Delta \varphi _{ii} \geq 0 $. Thus
%
\begin{eqnarray*} 0 \geq \frac{\partial \varphi_{ii}}{\partial t} & = &\Delta \varphi_{ii} -2  \varphi _k \varphi_{kii} - 2 \varphi ^2_{ki} \\
& \geq & - 2 \varphi_{ki}^2.\end{eqnarray*}
%
However, by convexity, $ \varphi_{ki}^2 \leq \varphi_{ii} \cdot \varphi_{kk}=0, \varphi_{ki}  \equiv 0 $. 
%
The theorem follows from strong maximum principle.

Let $ \Omega $ be a bounded domain with smooth boundary. Let $ u$ be the first Dirichlet eigenfunction with eigenvalue $ \lambda _1$. Then

\end{proof}

\begin{lemma} $ u \geq 0. $ 
\end{lemma}

\noindent {\bf Proof.} Otherwise, we may use $ |u| $ in place of $u$. From the Kato's inequality
%
\[|\nabla |u|| \leq |\nabla u| \]
%
we have 
\[ \frac{\int |\nabla |u| |^2}{\int u^2} \leq \frac{\int |\nabla u|^2}{\int u^2}. \]
%
Since $\frac{\int |\nabla u|^2}{\int u^2} $ is minimal, we must have 
\[ \frac{\int |\nabla |u| |^2}{\int u^2} = \frac{\int |\nabla u|^2}{\int u^2} \]
%
and thus $u$ is an eigenfunction.
\qed

\begin{lemma} $ u > 0 $.\end{lemma}

If not, assume that $ u (x_0) = 0 $. Let 
%
\[ u(x) = p^N (x) + 0 ( x ^{N+\varepsilon} ). \]
%
By the equation $ \Delta u = - \lambda_1 u $, we must have 
%
\[ \Delta p^N (x) = 0 .\]
%
Since $ u (x) \geq 0 $, we have $ p^N (x) \geq 0 $. This contradicts to the maximum principle.

\begin{lemma} 
$ \frac{\partial u}{\partial n} \neq 0. $
\end{lemma}

\begin{proof}
   Using Taylor's expansion
\[ u(x) = p ^N (x) + 0 ( x ^{N+ \epsilon)} ).\]
%
Again, $ \Delta p^N (x) = 0 , p^N (x) > 0 $ in $ \Omega $. By the Hopf lemma 
%
\[ \frac{\partial u}{\partial n} = \frac{\partial p^N}{\partial n} \neq 0. \]

Consider the flow 
%
\[ \frac{\partial u}{\partial t} = \Delta u = \lambda _1 u, \frac{u}{\partial \Omega} = 0 .\]
\end{proof}
We have 
\begin{lemma} 
For any smooth initial $ u _0 > 0 $, the flow exists and converges to the first eigenfunction. 
\end{lemma}

\begin{proof}
Let 
%
\[ u_0 = \sum a _i f _i (x).\]
%
Then
%
\[ u ( t, x) = \sum a_i e ^{(-a_{i} -  \lambda _{1} )t} f_i (x) \]
%
solves the flow equation. Obviously,
%
\[ \lim_{t \rightarrow  \infty} u(t, x) =  a _i f_1 (x). \]
%
If  we choose the $u_0$ such that $ a _i \neq 0 $. Then the flow converges to the first eigenfunction.
\end{proof}

\begin{lemma} If $ u_0 > 0 $, then $ u (t, x) >0 $.
\end{lemma}

\begin{proof} Maximum principle.
\end{proof}

\begin{lemma} If $ \Omega$ is convex, then near $ \partial \Omega  (-\log u) $ is convex.
\end{lemma}

\begin{proof}
We solve $ u ( x_1 \ldots, x_n) = 0 $ to get 
%
\[ x _n = x_n (x _1 , \ldots , x _{n-1} ) \]
%
$ \Omega$ is convex 
%
\[ \frac{\partial ^ 2 x _n }{\partial x_i \partial x_j} > 0 \;\;\;\mbox{(positive definite matrix)}.\]
%
Using implicitly function theorem, we have 
%
\[ -\frac{u_{ij}}{u_n} + \frac{u_{in} u_j}{u^2_n} + \frac{u_{jn} u_i}{u^2_n} - \frac{u_i u_j u_{nn}}{u_n^3} > 0 .\]
%
Thus for any $ (a _1, \cdots , a_n) $ if $ \sum a _i u_i = 0 $, we have
%
\[ - u _{ij} a _i a_j \geq \varepsilon |a|^2.\]
%
For any $a + \mu b , b = (u _1 \ldots, u_n ) $, we have 
\begin{eqnarray*}
& - & u_{ij} (a_i + \mu b_i )( a_j + \mu b_j ) \\
&& + \frac{1}{u} |\mu | ^2 |b| ^4 > 0 .
\end{eqnarray*}
Thus $ - \log  u$ is convex.

For the maximal principle, see page 84--A. 

For initial function, take any $ \frac{u}{\partial \Omega } = 0 \   u e ^{- c \sum x_i ^2}$.

\end{proof}



%%%%%%%%%%%%%%%%%%%%%%%%%%%%%%%%%%%%%%%%%%%%%%%%%%%%%%%%%%%%%%%%%%%%%%%



\section{Green's function and parobolicity}\label{Green_function}

Let $M$ be a complete non-compact Riemannian manifold. The Green's function is a smooth function on 
\[ M \times M  \backslash {\rm diag} (M) \]
%
such that 
%
\begin{enumerate}
\item $G (x, y)   =   G (y, x) $ and fixing $ y$, we have $ \Delta _x G (x, y)  = 0 \, \forall x \neq y $. 
\item $G(x, y) \geq 0 $.
\item Fixing $y$, when $ x \rightarrow y$, we have the following 
  \end{enumerate} 
%
\[ G(x, y) = \left\{ \begin{array}{ll} \rho _y (x) ^{2-n} (1 + 0(1) )& n>2 \\
- \log \rho_y (x) (1 + 0(1)) & n = 2 \end{array} \right. .\]
%
This last asymptotical expansion of the Green's function also implies that $ \Delta _x G(x, y) = - \delta _{x \cdot y} $.

From the asymptotical behavior we can find that $ n = 2$ and $ n> 2$, the Green's functions are very different.

We make the following.

\begin{definition} A complete manifold is said to be parabolic, if and only if it doesn't admit a positive Green's function. Otherwise it is said to be non-parabolic.
\end{definition}

\begin{definition}
An End, $E$, with respect to a compact subset $ \Omega \subset M  $ is an unbounded connected  component of $ M \backslash \Omega $. The number of ends with respect to $ \Omega $, denoted by $ N_{\Omega}(M)$, is the number of unbounded connected component of $ M \backslash \Omega $.
\end{definition}

\begin{definition} An End $E$ is said to be parabolic, if it doesn't admit a positive harmonic function $f$ satisfying 
%
\[ f \equiv  \ \mbox{on} \ \partial E\]
and
\[ \frac{\lim}{n\rightarrow E (\infty)} f(y) <  1 \]
where $ E(\infty)$ denotes the infinity of $E$. Otherwise, $E$ is said to be non-parabolic and the function $f$ is said to be a barrier function of $E$.

We prove the following result:
\end{definition}

\begin{theorem} Let $E$ be an parabolic end. Let $ A(R) = E \cap \partial B(R) $ where $ B(R)$ is the ball of radius $R$ with respect to some reference point. Let $f$ be a harmonic function on $E$ such that 
%
\[ f|_{\partial E} = 1 , f | _{A(R) } = 0 .\]
%
Then 
%
\[ \lim_{R\rightarrow \infty} \int_E |\nabla f|^2 \rightarrow 0. \]
\end{theorem}

\begin{proof}
Using the Green's formula, we have 
%
\[\int _E |\nabla f|^2 = - \int _E \frac{\partial f}{\partial r}  \]
%
where $\frac{\partial }{\partial r} $ is the outer  normal direction. We claim that $ \frac{\partial f}{\partial r} \rightarrow 0$.
To see this, we take a sequence $ R_1 < R_2 < \ldots R_k \rightarrow \infty$. The corresponding harmonic function $f_i = f_{R_{i}}$. By the maximal principle, $f_i$ are increasing sequence on any compact set of $E$. Let 
%
\[ \lim_{i \rightarrow \infty} f_i = f. \]
%
Then $f$ must be a positive harmonic function. By the parabolicity, $f \equiv 1$. By the maximal principle again 
%
\[ \frac{\partial f_i}{\partial r} \rightarrow 0\]
%
as $ i \rightarrow \infty$.
\end{proof}

\noindent Examples: $ \mathbb{R}^2 $ and $ \mathbb{R}^n $.

Let $ R > 0 $ be a big number. Let 
\[F(R) = \{ f \in C^\infty _0 (\mathbb{R}^n) | f = 1 \; \mbox{for}\; |x| < R, f \; \mbox{rotational symmetric} \}.\]

If $n > 2 $, then for any $ C > 0 $, there exists an $R_0$ such that for any $ R > R_0 $
%
\[ \int_{\mathbb{R}^n}|\nabla f | ^2 > C \]
%
for any $ f \in F(R) $. If $ n = 2 $, then for any $ \varepsilon > 0 $ there exists $ R_0 > 0 $ such that for any $ R> R_0 $, we can find an $ f_R \in F(R) $ for which 
%
\[ \int_{\mathbb{R}^2} | \nabla f|^2 < 
\varepsilon.\]

\begin{proof} If $ n > 2 $, then 
%
\[ \int ^\infty_{\mathbb{R}^2} \frac{1}{r^{n-1}} \, d r = \frac{1}{n-2} \frac{1}{R^{n-2}}.\]
Thus we have 
\begin{eqnarray*}
\int_{\mathbb{R}^n} | \nabla f|^2 &\geq &(n-2) \subset R^{n-2} \int ^\infty_R r ^{n-1} \left( \frac{\partial f}{\partial r} \right)^2 dr \int^\infty_0 \frac{1}{r^{n-1}} dr \\
& \geq& (n-1) \subset R^{n-2} \rightarrow + \infty .
\end{eqnarray*}
However, for $ n = 2$, we define $ f_R = \sigma _R (|x|( $ such that 
%
\[ \sigma _R (t) = \left\{ \begin{array} {ll} 1 & t \leq R\\
\left( 1 - \frac{\log R}{R}\right) ^{-1} \left( \frac{\log R}{\log t} - \frac{\log R}{R} \right) & R < t < e ^R\\
0 & t \geq e^R 
\end{array} \right. .\]
%
Then a straightforward computation gives
%
\[\int^\infty_0 + | \sigma ^\prime _R (t) |^2 dt \leq \frac{4}{3} \frac{1}{\log R} \; \;\;\; \mbox{for}\; R >>0 .\]
The main result of this section is the following characterization of parabolicity.
\end{proof}

\begin{theorem} Let $M$ be a complete manifold. If $ M$ is non-parabolic, then for any point $ p \in M$, we have 
%
\[ \int^\infty_1 \frac{dt}{Ap(t)} < + \infty\]
%
where $ A_p (r) $ denotes the area of $ \partial B_p (r) $.
\end{theorem}

\begin{proof} For $ p\in M $, let $ G(p, y) $ be the Green's function (assuming that it exists). Let $ G_i (p, y) $ be the Green's function on $ B_p (R_i) $ with the Dirichlet boundary condition, where $ R_i \rightarrow   \infty $.
\end{proof}

For any $ |< R< R_i $, let's denote that 
\begin{eqnarray*}
S_i(1) & = & \sup_{y 
\in \partial B_{p}(1)} \; G_i (p, y) \\
i_i(R) & = & \inf_{y 
\in \partial B_{p}(R)} \; G_i (p, y)  .
\end{eqnarray*}
Let $f$ be the harmonic function defined on $ B_p (R) \backslash B_p (1) $ satisfying the boundary conditions
%
\begin{eqnarray*} &&f(y) = S_i (1) \; \;\mbox{on}\; \partial B_p(1) \\
&&f(y) = G_i (p, y) \;\;\mbox{on}\; \partial B_p(R) .
\end{eqnarray*}
The maximum principle implies that 
%
\[ f (y) \geq G_i (p, y) \;\mbox{on} \; B_p (R) \backslash B_p (1) .\]
In particular, we have 
%
\[ \frac{\partial f}{\partial r} \leq \frac{\partial G_i}{\partial r} \; \mbox{on}\; \partial B_p (R) .\]
%
On the other hand, since $ f(y)$ is harmonic, Stokes theorem implies 
%
\[ 0 = \int _{B_{p} (R) | B_{p} (1) } \Delta f = \int _{\partial B_{p} (R)} \frac{\partial f}{\partial r} - 
\int _{\partial B_{p} (1)}\frac{\partial f}{\partial r}.\]
%
Also, we observe that 
\[\int _{\partial B_{p} (R)}  \frac{\partial G_{i}}{\partial r} = \int _{ B_{p} (R)}  \Delta G_i = - 1 .\]
%
Thus we get 
%
\[ \int _{\partial B_{p} (1)}  \frac{\partial f}{\partial r} \leq - 1 .\]

 consider $h$ to be the harmonic function defined on $ B_p (R) \backslash B_p (1) $ satisfying the boundary conditions
%
\begin{eqnarray*}
h(y) & = & S_i (1) \; \mbox{on}\; \partial B_p (1) \\
h(y) & = & i_i (R) \; \mbox{on}\; \partial B_p (R) .
\end{eqnarray*}
%
Again, the maximum principle implies that 
%
\[ h(y) \leq f (y) \; \mbox{on}\; B_p (R) \backslash B_p (1) \]
and
\[ \frac{\partial h}{\partial r} \leq \frac{\partial h}{\partial r} \; \mbox{on}\; B_p (1)  .\]
%
Thus we have 
%
\[  \int _{\partial B_{p} (R)} \frac{\partial h}{\partial r} = \int _{\partial B_{p} (1) }\frac{\partial h}{\partial r}
 \leq - 1 .\]
%
Define the function 
%
\[g(r) = (s_i (1) - i_i (R) ) \left( \int^R_1 \frac{dt}{A_p (t)} \right) ^{-1} \int ^R_r \frac{dt}{A_p (t) } + i_i (R). \]
Then $ g (r (y)) $ will have the same boundary conditions as $ h(y)$. The Dirichlet integral minimizing property for harmonic functions implies that 
%
\begin{align*}
&\int_{B_{p} (R) \backslash B_{p} (1)} |\nabla h |^2 \leq \int_{B_{p} (R) \backslash B_{p} (1)}  |\nabla g|^2 \\
& =  \int^R_1 \left( ( s_i (1) - i _i (R)) \left( \int^R_1 \frac{dt}{A_p (t)} \right) ^{-1}  \frac{1}{A_p (r)} \right) ^2 \, A_p (r) \, dr \\
& =  \left(  s_i (1) - i _i (R)\right) \left( \int^R_1 \frac{dt}{A_p (t)} \right) ^{-1}.
\end{align*}
%
On the other hand, integration by parts yields 
%
\begin{eqnarray*}
\int_{B_{p} (R) \backslash B_{p} (1)} |\nabla h |^2 & = & i_i(R)\int_{\partial B_{p} (R) } \frac{\partial h}{\partial r } -s_i(1) \int_{\partial B_{p} (R) }\frac{\partial h}{\partial r }\\
& \leq &    s_i (1) - i _i (R))  .
\end{eqnarray*}
%
By taking the limit
%
\[\int ^k_1 \frac{dt}{A_p (t)}  \leq \sup _{y \in \partial B_{p} (1) } G (p, y) - \inf _{y \in \partial B_{p} (R)} G (p , y). \]

\begin{corollary} Let $ M$ be a complete manifold such that 
%
\[ {\rm vol} (B (R)) \leq CR^2 \]
then $M$ has to be parabolic.
\end{corollary}

\begin{corollary}
Let $M$ be a Riemann surface such that 
%
\[ \int_M |k| < + \infty\]
%
Then $M$ is parabolic.
\end{corollary}
Proof of a theorem of Brascamp and Lieb.

\begin{theorem} Let $ \Omega $ be a bounded convex domain of $ \mathbb{R} ^n$.  Let $ u$ be the first Dirichlet eigenfunction. Let 
%
\[ \Delta u = - \lambda _1 u \;\; u > 0 .\]
%
Then $ \log u $ must be concave.
\end{theorem}
\noindent{\bf Proof}  We choose any function $ u_0 > 0 , u_0 |_{\partial \Omega} = 0 $ such that $ - \log u_0$ is concave. Such a function always exists. For example, we can take the convex hull of the graph of $-\log u$.

Consider the flow 
\[\frac{\partial u}{\partial t} = \Delta u + \lambda _1 u \;\; \frac{u}{\partial \Omega} = 0 .\]
%
We {\bf assume} that $ u_t \rightarrow u$, the first eigenfunction. We are going to use the maximum principle. Let $ T$ be the biggest number such that $ \det (- \nabla ^2 \log u) $ is degenerated. Thus there is an $ x_0 \in M $ and a direction $ i$ such that 
%
\[ - ( \log u) _{ii} (x_0) = 0 \]
%
and for other $ j, ( - \log u) _{jj} (x_0) \geq 0$. Let $ \varphi = - \log u $. Then the envolution of $ \varphi $ is 
%
\[ \frac{\partial \varphi}{\partial t} = \Delta \varphi - |\nabla \varphi |^2 - \lambda _1 .\]
%
By the maximum principle, $ \varphi _{ii} k = 0 , \frac{\partial \varphi _{ii}}{\partial t} \leq 0 \; \Delta \varphi _{ii} \geq 0 $. Thus
%
\begin{eqnarray*}
0 &\geq & \frac{\partial \varphi _{ii}}{\partial t} = \Delta \varphi_{ii} - 2 \varphi _k \varphi_{kii} - 2 \varphi ^2_{kl}\\
&\geq & - 2 \varphi _{ki}^2. \end{eqnarray*}
%
However by convexity, $ \varphi_{ki}^2 - \varphi _{ii} \varphi_{kk}= 0, \varphi_{ki} \equiv 0 $. The theorem follows from the strong maximum principle. 

%%%%%%%%%%%%%%%%%%%%%%%%%%%%%%%%%%%%%%%%%%%%%%%
%%%%%%%%%%%%%%%%%%%%%%%%%%%%%%%%%%%%%%%%%%%%%%%5



%%%%%%%%%%%%%%%%
%%%%%%%%%%%%%%%%%%%%%%%%%%%%%%%%%%%%%%%%%%%%%
%%%%%%%%%%%%%%%%%%%%%%%%%%%%%%%%%%%%%%%%%%%%%%%%5
