

\chapter{Essential Spectrum on complete non-compact manifold}\label{Essential_spec}
\section{Spectrum on complete non-compact manifolds}\label{Spectrum_on_manifold}

Unlike in the case of compact manifold, in general, a complete manifold doesn't admit any eigenvalues. For example, there are no $ L^2$-eigenvalue on $\mathbb{R}^n$. That is, $ \forall \lambda \in \mathbb{R} $. If 
%
\[ \Delta f + \lambda f = 0 \]
%
and $ f \in L^2 (\mathbb{R}^n)$, then $ f \equiv 0$.

Escobar proved that if $M$ has a rotational symmetric metric, then there is no $ L^2$-eigenvalue. Let $ \Delta $ be the Laplace operator on a complete non-compact manifold $M$. By the argument before, $\Delta$ naturally extends to a self-adjoint densely defined operator, which we still denote $\Delta$ for the sake of simplicity.

It is well-known that there is a spectrum measure such that 
%
\[\Delta = \int^\infty _0 \lambda d E\]
(Here we assume $\Delta$ is the geometric Laplacian, which is a positive operator).

Define $ e^{-\Delta t} $ for any $ t > 0 $. Obviously, it is a bounded operator. Thus by the Hahn-Banach
$ e^{-\Delta t} $ is a bounded operator.

The heat kernel is defined as  
%
\[   e^{-\Delta t}   f(x) = \int H(x, y, t) f (y) \, dy.\]
%
The Green's function is defined as 
%
\[ G(x, y) = \int^\infty_0 H (x, y, t) \, dt .\]
Of course, we need to prove the existence of these functions when $M$ is complete non-compact.

The pure point spectrum of $ \Delta $ are those $ \lambda \in \mathbb{R} $ such that 
\begin{enumerate}
  \item There exist a $ L^2 $ function $ f \neq 0 $ such that 
   \[ \Delta f + \lambda f = 0 \]
  \item The multiplicity of $ \lambda $ is finite
  \item In a neighborhood of $ \lambda $, it is the only spectrum point.
 \end{enumerate} 
  We define 
%
\[ \rho (\Delta ) = \left\{ y \in \mathbb{R} \left| \begin{array}{r} (\Delta - y ) ^{-1} \ \mbox{is a bounded }\\
 \mbox{operator} \end{array} \right. \right\} \]
%
$ \sigma (\Delta ) = \mathbb{R} - \rho (\Delta) $ is the spectrum set of $ \Delta $. From the above discussion, $ \sigma (\Delta ) $ decomposes $ \emptyset $ as the union of pure point spectrum, and the so-called essential spectrum, which is by definition, the complement of pure point spectrum.

Using the above definition, $ \lambda \in \sigma (\Delta ) $ belongs to the set $ \sigma _{ess} ( \Delta ) $, if either 
\begin{enumerate}
  \item $\lambda $ is an eigenvalue of infinite multiplicity, or 
  \item $ \lambda $ is the limiting point of $ \sigma (\Delta ) .$ \end{enumerate}  


The following theorems in functional analysis characterizing the essential spectrum.

\begin{theorem} A necessary and sufficient condition for the interval $ ( - \infty , \lambda ) $ to intersect the essential spec of an self-adjoint densely defined operator $A$ is that, for all $ \varepsilon > 0 $,  there exists an infinite dimensional subspace $ G _\varepsilon \subset {\rm Dom} (A) $, for which $ ( A f - \lambda f - \varepsilon f, f) < 0 $. 
\end{theorem} 

\begin{theorem}
A necessary and sufficient condition for the interval $ ( \lambda - \sigma , \lambda + \sigma ) $ to intersect the essential spectrum of $A$ is that there exists an infinite dimensional subspace $ G < {\rm Dom} (A) $ for which $ \Vert A - \lambda I) f \Vert < \sigma\Vert f\Vert, f \in G$.
\end{theorem}

For reference, see Donnelly, Topology 20, 1--14, 1981. 

Using the above result, we give the following variational characterization of the lower bound of spectrum and the lower bound of essential spectrum.

\begin{theorem}Let 
%
\[ \lambda_0 = \inf_{f \in C^\infty_0 (M)} \frac{\int_M |\nabla f| ^2}{\int_M f ^2}\]
and
\[ \int_{ess} =  \sup_k \inf_{f \in C^\infty_0 (M\backslash k)} \frac{\int_M |\nabla f| ^2}{\int_M f ^2}\]
%
where $k$ is a compact set running through an exhaustion of the manifold. Then $ \lambda _0 $ and $ \lambda_{ess} $ are the lower bound of $ \sigma (\Delta )$ and $ \sigma _{ess} ( \Delta )$ respectively.
\end{theorem}

\noindent{\bf Proof} We first prove the formula for $ \lambda _{ess} $. Let
%
\[ \lambda _{ess} ^\prime = \inf \sigma_{ess} ( \Delta ) \]
%
is $ \sigma _{ess} (\Delta ) = \emptyset$, we define,  $ \lambda _{ess} ^\prime  = + \infty$. By the definition, $ \forall \epsilon > 0 , ( - \infty, \lambda ^\prime _{ess} + \varepsilon ) \cap \sigma _{ess} (\Delta ) \neq \emptyset$.
Then we can find infinite dimensional space $V$such that for any $ f \in V  $
%
\begin{equation} \langle \Delta f - ( \lambda _{ess} ^\prime + \varepsilon ) f , f\rangle <0. \tag{$\star$}\end{equation}
 %
Without loss of generality, we may assume that 
%
\[ \int_M f ^2 = 1 .\]
%
Also, without loss of generality, we may assume that all elements in $V$ are smooth.

We leave as an exercise, to prove that ($\star$) implies 
%
\[ \int_M |\nabla f | ^2 \leq  ( \lambda _{ess}^\prime  + \varepsilon ).
\]
%
Now we assume that $k$ is a compact set. Let $ k  ^\prime $ be a larger ball containing $k$. Let $ \rho$ be the cut-off function such that $ \rho \equiv 1 $ on $k$ but $ \rho \equiv 0 $ outside $k^\prime $. We claim (***) that $ \forall \varepsilon > 0$, there  is an $ f \in V $ with $ \int f^2 = 1 $ but
%
\[ \int _M \rho ^2 f^2 < \varepsilon. \]
%
If the above is not true, then for any $ f \in V .$ 
%
\[ \int_M \rho^2 f ^2 \geq \varepsilon _0, \; \mbox{if} \; \int f ^2 = 1 .\]
Since the set $ f \in V $ is of infinite dimensional, the set $ \rho f $ is of infinite dimensional. Thus we can find an orthogonal basis
%
\[ \int\rho^2 f _i f _j = 0 \;\; i \neq j \]
%
while we can still keep  $  \boxed{\int f _i^2 = 1 }$. We consider
%
\begin{eqnarray*}
\int | \nabla (\varphi f_i )| ^2 & \leq & 2 \int | \nabla \rho |^2 f^2_i + 2 \int \rho ^2 |\nabla f_i |^2 \\
& \leq & 2 C + 2 (\lambda _{ess} ^\prime + \varepsilon ) .
\end{eqnarray*}
%
Thus 
%
\[ \frac{\int \nabla ( \rho f_i )^2}{\int ( \rho f_i )^2 } \leq \frac{2C + 2 ( \lambda _{ess} ^\prime + \varepsilon }{\varepsilon _0} \]
%
for infinitely dimensional space. This is a contradiction because on the compact set $ k ^\prime$  the eigenvalues go to infinity.

With the above preparation, we can prove our theorem $ \forall \varepsilon > 0 $. We find an $f$ with $ \int f ^2 = 1 $ but 
%
\[ \int \rho ^2 f ^2 < \varepsilon.\]
%
Consider $ \rho_1 = 1 - \rho$
%
\begin{eqnarray*} \int ( \nabla (\rho _1 f ) |^2 &= &\int \rho^2_1 | \nabla f |^2 + 2 \int \rho_1 f  \nabla \rho _1 \nabla f \\
&& + \int f ^2 |\nabla \rho_1 |^2 .
\end{eqnarray*}
%
Using the above inequality, we have 
\begin{eqnarray*}
\lefteqn{\int f ^2 |\nabla \rho_1 |^2 \leq C \varepsilon }\\
&& 2 \int \rho_1 f \nabla \rho _1 \nabla f = - \frac{1}{2} \int \Delta \rho^2 _1 \cdot f ^2\\
&& \leq C \varepsilon 
\end{eqnarray*}
%
because the supp of $ \nabla \rho_1, \Delta \rho^2_1, $ are within $k^\prime $. Thus
%
\begin{eqnarray*}
\int |\nabla ( \rho _1 f ) |^2 &\leq & ( \lambda _{ess}^\prime + \varepsilon ) + C \varepsilon\\
\int \rho^2 _1 f ^2 & \geq & 1 - \varepsilon. 
\end{eqnarray*}
%
Using the definition, we get 
%
\[\lambda _{ess} \leq \lambda _{ess} ^\prime + \varepsilon \]
%
and thus $ \lambda _{ess} \leq \lambda _{ess} ^\prime $. The other direction is easier to prove.

Using the same method, we can prove the case for $ \lambda _0 $, provided that we need a similar theorem like 
Theorem 1.

It is an interesting question to compute the set of essential. It is particularly interesting to get the lower 
bound estimate for the essential spectrum because of the following theorem.

\begin{theorem}Suppose $ \lambda _0 < \lambda _{ess} $, then $ \lambda _ 0 $ is an eigenvalue of $M$ with finite dimensional eigenspace.

That is, there exists an $ L^2 $ function $ f \neq 0 $, such that 
%
\[ \Delta f = - \lambda _0 f \]
%
which is a very strong result.
\end{theorem}
In what follows we use $ \mathbb{R}^n $ to elaborate our theorem.

First let's compute $ \lambda _{ess} $ for $ \mathbb{R}^n $. We claim that $ \lambda _{ess} = 0 $. By definition, this is equivalent to say that $ \forall \varepsilon > 0  \, \forall \, R > 0 $, there is a function $ f \in C^\infty_0 ( \mathbb{R} ^n - B (R)) $ such that 
%
\[\int |\nabla f |^2 < \varepsilon \int f ^2.\]
%
This is obvious: let $ B (x_0, R ^\prime )$ be a ball of radius $R^\prime $ with center $ x _0 $ such that 
$ |x_0| >  2 R+ 1 + R^\prime $. Then
%
\[B(x_0, R^\prime ) \subset \mathbb{R}^n \backslash (R) .\]
%
Let $ f$ be the first Dirichlet eigenfunction of $ B (X_0, R^\prime )$.
Then if $ R^\prime \rightarrow \infty$ 
%
\[ \int |\nabla f |^2 /\int f ^2 < \varepsilon \]
%
zero extending $ f$ to $ \mathbb{R} ^n \backslash B(R) $ we get the result. 

Using Theorem 2, we can even prove that 
%
\[ \sigma _{ess} ( \Delta ) = [ 0, + \infty) .\]

To prove the above, we make the following observation $ \forall \lambda > 0 , \forall m \in \mathbb{Z} $, we can find a square of size $ m \bar{\pi} \sqrt{ \frac{\lambda }{n}} $ such that 
%
\[ f = \sin \sqrt{ \frac{\lambda }{n}} x _1 \cdot \sin \sqrt{ \frac{\lambda }{n}} x_2 \ldots \sin \sqrt{ \frac{\lambda }{n}} x_n \]
%
is an eigenfunction with Dirichlet condition: $ \Delta f + \lambda f = 0 $.

However, we can't use Theorem 2 directly. The reason is that $f$ is {\bf not} second differentiable near the boundary. Thus we need to use cut-off function. Without loss of generality, we may assume that the square is in the first quadratic. We denote such a square to be $s$.

\begin{figure}[h!]
\vspace{0.2in}
\caption{S Square}
\end{figure}

\vspace{0.5in}

 Let $ s _1 $ be a square with the same center as $ s_1$ but with smaller size. We assume that the distance of the boundary of $s_1$ to $s$ is $d$, which is to be determined later.

We choose a cut-off function $\rho$ such that $ \rho\equiv 1$ on $ s_1 $ and $ \rho \equiv 0$ outside $s$. By the same argument as before,  we can prove that 
%
\[\Vert \Delta ( \rho f ) + \lambda \rho f \Vert _{L_{2}} \subset \varepsilon \Vert \rho  f \Vert_{L_{2}}.\]
%
Thus $ ess (\Delta ) = [ 0 , + \infty)$.

Unfortunately, the above argument doesn't apply to the general case. Thus the following result of ***
is very surprising and interesting. 

\begin{theorem}[J-P. Wang] Let $M$ be a complete manifold with non-negative Ricci curvature. Then 
%
\[ ess (\Delta ) = [ 0, \infty) \]
\end{theorem}
Note that $ \lambda _{ess} $ was known before, e.g., P.Li-Wang, Brooks.

%%%%%%%%%%%%%%%%%%%%%%%%%%%%%%%%%%%%%

In this lecture, we study the essential spectrum of a complete non-compact manifold with non-negative Ricci curvature. We are going to prove that the set of essential spectrum is $ [ 0 , \infty)$. The reference papers for this lecture are

\begin{enumerate}
  \item Sturm, J. Funct. Anal. 118, 442--453, 1993
  \item Wang, Math. Research Letters. 4, 473--479, 1997
 \end{enumerate}

Through this lecture, $M$ is a complete Riemannian manifold with non-negative Ricci curvature. We say the volume $ (M, g)$ grows uniformly sub-exponentially, if for any $ \varepsilon > 0 $, there is a constant $ C < \infty$ such that for all $ r > 0 $ and all $ x \in M $ 
%
\[ v (B_r (x)) \leq C e ^{\varepsilon r} v (B, (x))\]

\begin{theorem}If the volume of $ (M, g) $ grow uniformly sub exponentially, then the spectrum $ \sigma (\Delta p ) $ of $ \Delta p $ acting on $ L^p (M) $ is independent of $ p \in [1, \infty) $. In particular, it is a subset of the real line.

Note that by the Bishop volume comparison theorem, Ricci non-negative implies uniformly sub-exponentially volume growth.
\end{theorem}

The theorem can be proved using the resolvent estimates, which are based on the previous heat kernel estimate. 

We begin with the following.

\begin{lemma} If the volume of $ (M, g) $ grows uniformly subexponentially, then for any $ \varepsilon > 0 $ 
\[ \sup_{x \in M} \int_M e ^{-\varepsilon d (x, y)} (v (B_1 (x))) ^{-\frac{1}{2}} ( v (B _1 (y))) ^{-\frac{1}{2}} d \nu (y) < \infty.\]
\end{lemma}

\noindent{\bf Proof:} We take $ r = d (x, y) $. Then since 
%
\[ B_1 (y) \subset B_{r+1} (x) \]
%
we must have 
%
\[\nu (B_1 (y)) ^{-\frac{1}{2}} \geq \nu (B_{r+1} (x)) ^{-\frac{1}{2}} \geq C e ^{-\frac{1}{2}(r+1)}\nu (B_1(x)) ^{-\frac{1}{2}}\]
%
for any $ x$. Thus the integration in the lemma if less than
%
\[ C \int_M e^{-\varepsilon  r} \cdot e ^{\frac{1}{2}\varepsilon (r + 1)} \nu (B_1(x)) ^{-1} dy .\]
%
We let $ f (r) = {\rm vol} (\partial B_r (x)) $ and $ F (r) = \int^r_0 f(t) \, dt $. Then up to a constant, the above expression is less than 
%
\[ (\nu (B_1 (x)) ) ^{-1} \int^\infty_0 e ^{-\frac{1}{2}\varepsilon r} f (f) \, dr. \]
%
By volume comparison again $ f(r) \leq c r ^{n-1} \nu (B _1 (x)) .$
The lemma follows.

\begin{lemma} For any $ \beta > 0 $, there is an $ n \in N, \alpha < 0 , c < \infty $ such that the integral kernel $ g _\alpha ^{(\frac{n}{2})} (x, y) $ of $ (\Delta - \alpha) ^{-\frac{n}{2}} $ exists and satisfies 
%
\[g_\alpha ^{\frac{n}{2}} (x, y) \leq C e ^{-\beta d (x, y)} \varphi(x)^2 \]
%
where $ \varphi (x) = ( \nu (B_1 (x))) ^{-\frac{1}{2}} $.

The lemma was proved on page 67, using the heat kernel estimates.
\end{lemma}
Before going further, let's make some remarks on the kernel of an operator. Let $ A$ be an operator on functions. If there is a function $ g(x, y) $ such that
%
\[ A f (x) = \int_M g (x, y) f (y) \, dy \]
%
then we call $ g (x, y) $ the ``kernel'' of $A$. However, in general, the kernel doesn't exist.

To see why the kernel in general does not exist, we let $ \xi \in \rho (\Delta ) $. To be more specific, $ \Delta $ is an operator on $ L^2 (M) $ so we assume that $ \xi \in \rho (\Delta_2) $. The operator $ ( \Delta _2 - \xi) ^{-1} $ is called the resolvent. It is a bounded operator form
$ L^2(M) \rightarrow L^2 (M) $. (By Hahn-Banach theorem, it can be extended to whole $ L^2(M)$. However, in general, the kernel doesn't exist: if not, we let $ f_i \rightarrow \delta $ be a sequence converges to the $ \delta$-function in $ L^1(M) $ then $ \Delta _2f_i $ could have been bounded. We need an  estimate to extend $ \Delta _2 $ from $ L^2 (M) $ to $ L^1(M)$.


Lemma 2 told us that for $ \alpha < 0 , (\Delta - \alpha ) ^{- \frac{n}{2}} $ has a kernel. The operator can be extended to $ L^1(M)$. Further more, the kernel exponentially decays. 

The Laplacian we used here is the geometric Laplacian. That is, it is a positive operator.

For our purpose, we just need to prove $ \sigma (\Delta _1) \subset \sigma (\Delta _2) $, which is also the major part of the paper of Sturm.

Recall that $ x \in (A) $, if, $ (A - x I ) ^{-1} $ is a bounded operator. We define the spectrum $ \sigma (A) = \mathbb{R} - \rho (A)$.
Note that $ \sigma (\Delta _2) \subset [ 0 , \infty) $. Then for $ \alpha < 0 , (\Delta _2 - \alpha I) ^{-1} $ is bounded. Since 
%
\[ \Delta - \alpha I = (\Delta - \zeta I) ( 1 - (\alpha - \zeta ) ( \Delta - \zeta I)^{-1}). \]
we have
\[ (\Delta - \zeta I ) ^{-1} =   ( 1 - (\alpha - \zeta ) ( \Delta - \zeta )^{-1}  ( \Delta - \alpha I)^{-1}. \]
%
or for any $ n$ 
%
\[ ( \Delta - \zeta I ) ^{-n} =  ( 1 - (\alpha - \zeta ) ( \Delta - \zeta )^{-1})^n (\Delta - \alpha I) ^{-n} .\]
From the above identity, the kernel for $ (\Delta - \zeta I) ^{-n} $ exists. Let 
%
\[A = ( 1 - ( \alpha - \zeta )(
\Delta - \zeta )^{-1} )^n .\]
%
Then
%
\[ A _x ( g _{\alpha} ^{(\frac{n}{2})} (x, y ) \]
%
is a kernel of $ (\Delta - \zeta I) ^{-n} $.

\begin{lemma} Let $ g (x, y) $ be the kernel of $ (\Delta - \zeta I) ^{-n} $. Then 
%
\[ |g (x, y)|\leq C e ^{- \varepsilon d (x, y)} \varphi (x) \varphi (y) \]
%
where $ \varphi (x) = v (B _1 (x))  ^{-\frac{1}{2}} $. 

We omit the proof of the above lemma. By Lemma 1,
%
\[\sup_x \int | g (x, y) \\, dy < + \infty.\]
%
Thus $ (\Delta _2 - \zeta )^{-n} $ is an operator from $ L^1 \rightarrow L^1 $.

Since $ \Delta _2 $ and $ \Delta _1 $ are the same acting on $ C^\infty $ functions, $(\Delta _1 - \zeta )^{-n} $ is a bounded operator from $ L^1 (M) \rightarrow L^1 (M) $. 
\end{lemma}
\begin{lemma} If $( \Delta_1 - \zeta) ^{-n} $ is a bounded operator, then $ \zeta \in \rho (\Delta _1)$.
\end{lemma}

\noindent{\bf Proof.} Since $ ( \Delta _1 - \zeta )^{-n} $ is bounded, there is a neighborhood of $ \zeta$ such that for any $ \zeta^\prime $ in the neighborhood, $ (\Delta _1 - \zeta ^\prime )^{-n} $ is also bounded. By 
%
\[ ( \Delta _1 - \zeta )^{-1} = ( \Delta _1 - \zeta^\prime )^{-1} ( 1 - ( \zeta - \zeta ^\prime )( \Delta _1 - \zeta^\prime  )^{-1}) ^{-1} \]
%
and the fact that the latter can be expended to a convergent series, $ ( \Delta _1 - \zeta )^{-1}  $ is bounded in $ L^\prime (M)$. Thus $ \zeta \in \rho (\Delta _1) $. 

From the above lemma, we have 
%
\[ \sigma (\Delta _1) \subset \sigma (\Delta _2 ) \subset [ 0, \infty). \]

The theorem thus follows from 

\begin{theorem}[Wang] Let $ M$ be a complete Riemannian manifold with non-negative Ricci curvature. Then
%
\[ \sigma (\Delta_1) = [0, \infty). \]
Before giving the proof, we use the following theorem.
\end{theorem}
\begin{theorem}Let $M$ be a complete non-compact Riemannian manifold. Ric$(M) \geq 0 $. Then 
%
\[ {\rm vol} B_p (R) \geq C (n , {\rm vol} B_p (1) ) R .\]
%
 \end{theorem}
\noindent{\bf Proof.} Fixing $ x_0 \in \partial B_p (R) $. Using the comparison theorem, we have 
%
\[ \Delta p^2 \leq 2 n .\]
For any $ \varphi \in C^\infty _0 (M), \varphi \geq 0 $, we have 
%
\[ \int _M \varphi \Delta \rho^2 \leq  2 n \int_M \varphi.\]
%
We choose a standard cut-off function $ \varphi = \psi ( \rho (x)) $, where
%
\[ \psi (t) = \left\{ \begin{array} {ll} 1 & 0 \leq t \leq R - 1\\
\frac{1}{2} (R+1 - t), & R-1 \leq t \leq R+1 \\
0 & t \geq R + 1 
\end{array} \right. .\]
%
By Stokes theorem
%
\begin{eqnarray*}
\int_M \varphi \Delta \rho^2 & = & - \int \nabla \varphi \nabla \rho^2 = - 2 \int \psi ^\prime \rho |\nabla \rho| ^2 \\
 & = & \int_{Bx_{0} (R + 1) \backslash B x_{0} (R-1) } \rho\\
& \geq & (R - 1) {\rm vol} ( x_0 (R+1) \backslash Bx_0 (R-1)) .
\end{eqnarray*}
%
Thus 
%
\[ (R-1) {\rm vol} (B x_{0} (R+1)- B x_{0} (R-1) ) \leq 2 n \int \varphi \leq 2 n \, {\rm vol} B x_{0} (R+1)\]
%
Obviously
%
\[ B_p (1) \subset B x_{0} (R+1) \backslash B x_{0} (R-1) .\]
%
Thus
%
\[ 2 n\,  {\rm vol} B x_{0} (R+1) \geq (R-1) {\rm vol} B _p (1) .\]
%
Since $ B_p (2 (R+1) \supset B x_{0} (R+1) $, we have 
\begin{eqnarray*}
 2 n \, {\rm vol} B _p(2 (R+1))& \geq & (R-1) {\rm vol} B _p (1) \\
  {\rm vol} B_p (2 ( R+1)) & \geq & \frac{R-1}{2n} {\rm vol} B_p (1). \end{eqnarray*}
Wang modified the above argument and proved that 

\begin{lemma} There is a constant $ C(n) $ such that for $ a \leq r \leq R$ \end{lemma}
%
\begin{equation}
V_q (r) \leq C \frac{r}{R} V_1 (R). \tag{$\star$} 
\end{equation}

If we choose $ r = \varepsilon R $ such that $ C \varepsilon < \frac{1}{2}$, we have 
%
\[ V_q (\varepsilon ) < \frac{1}{2} V_q (R) \]
Furthermore, we have 
%
\begin{equation}
A_q (\varepsilon r) \leq \frac{C}{R} (V_q (2r) - V_q (\varepsilon r)). \tag{$\star \star$} \end{equation}
\hspace{1.9in} $\uparrow$  \hfill 

%
This is inverse Laplacian comparison theorem!!!

We pick a cut-off function $ \psi $ such that $ \psi (r) = 1 $ for $ 2 \varepsilon < r < 2 , \psi (r) = 0 $ for $ r > 2 , r < \varepsilon , 0 \leq \psi \leq 1, |\psi ^\prime |+| \psi ^{\prime \prime } < C(s) $. From now on, we fix $ \varepsilon > 0 $. Let 
%
\[ \varphi _k = \psi \left( \frac{r(x)}{k} \right) \sin \sqrt{\lambda} r.\]
%
Then $ \{ \varphi _k\}$ forms an infinite dimensional vector space. 

A straightforward computation gives 
%
\[ |\Delta \varphi _k + \lambda \varphi_k |\leq \frac{C}{k} + C | \Delta r|.\]
%
We have known $ \Delta r \leq \frac{n-1}{r} $. Thus 
%
\[|\Delta r| \leqq \frac{n-1}{r} - \Delta r + \frac{n-1}{r} = \frac{2(n-1)}{r} - \Delta r .\]
Thus we have 
%
\begin{eqnarray*}  |\Delta \varphi _k + \lambda \varphi _k |&\leq& \frac{C}{k} - \frac{C}{k} \Delta r \\
|\Delta \varphi _k + \lambda \varphi _k |_{L^\prime (M)}&\leq& \frac{C}{k}   (V (2k) - V ( \varepsilon k)) - \frac{C}{k} \int_{B(2k) - B (\varepsilon k) }\Delta r \end{eqnarray*}
%
On the other hand, 
%
\[ | \varphi _k |_{L^{\prime}} \leq C ( V(k) - V ( \varepsilon k)) . \]
%
By the volume comparison ($\star$) 
%
\[ \frac{C}{k}( V(2k) - V (2 (\varepsilon k)) \leq \frac{C}{k}|\varphi _k |_{L^{\prime}}\]
On the other hand 
\begin{eqnarray*}
- \int_{B(2k) - B (\varepsilon k) }  = & \int_{\partial B(\varepsilon k)  } 1 - \int_{\partial B(2k)   } 1 \\
^ \leq & \int_{\partial B(\varepsilon k) } 1 = A_q (\varepsilon k) .
\end{eqnarray*}
Using ($\star$)($\star$), we get the desired estimate.



Further readings:

Griffiths-Harris: Principle of algebraic geometry.



%**********************Lecture 1************************************************

%**********************Lecture 1************************************************
%**********************Lecture 3************************************************
\section{The $L^p$-spectrum of the Laplacian}\label{Lp_spectrum}

\subsection{The Laplacian on $L^p$ space}

\begin{definition}
A one-parameter semi-group on a complex Banach space $B$ is a family
$T_t$ of bounded linear operators, where $T_t: B\rightarrow B$ parameterized by
real numbers $t\geqslant 0$ and satisfies the following relations:
\begin{itemize}
\item[\ding{172}.] $T_0 = 1$;
\item[\ding{173}.] If $0\leqslant s_1 t < +\infty$, then
\[
T_s T_t = T_{s+t}
\]
\item[\ding{174}.] The map
\[
t_1 f \mapsto T_t f
\]
from $[0, +\infty)\times B$ to $B$ is jointly continuous.
\end{itemize}
\end{definition}

The (infinitesimal) generator $Z$ of a one-parameter semi-group $T_t$ is defined
by
\[
Zf = \lim_{t\rightarrow 0^+} t^{-1}(T_t f - f)
\]
The domain $Dom(Z)$ of $Z$ being the set of $f$ for which the limit exists. It
is evident that $Dom(Z)$ is a linear space. Moreover, we have
%\setcounter{lemma}{0}
\begin{lemma}
The subspace $Dom(Z)$ is dense in $B$, and is invariant under $T_t$ in the sense
that
\[
T_t(Dom(Z)) \subset Dom(Z)
\]
for all $t\geqslant 0$. Moreover
\[
T_t Zf = ZT_t f
\]
for all $f\in Dom(Z)$ and $t \geqslant 0$.
\end{lemma}
{\bf Proof.}
If $f\in B$, we define
\[
f_t = \int_0^t T_x fdx
\]
The above integration exists in the following sense: since $T_x f$ is a
continuous function of $x$, we can define the integration as the limit of the
corresponding Riemann sums. In a Banach space, absolute convergence implies the
conditional convergence. Thus in order to prove the convergence of the Riemann
sums, we only need to verify that
\[
\int_0^t \|T_x f\|dx
\]
is convergent. But this follows easily from the joint continuity in the
definition of semi-group. $\|T_x f\|$ must be uniformly bounded for small $x$.

We compute
\begin{eqnarray}
\nonumber & & \lim_{h\rightarrow 0^+} h^{-1} (T_h f_t - f_t) \\
\nonumber & = & \lim_{h\rightarrow 0^+}\left\{h^{-1}\int_h^{t+h} T_x fdx - h^{-1}\int_0^t T_xfdx\right\} \\
\nonumber & = & \lim_{h\rightarrow 0^+}\left\{h^{-1}\int_t^{t+h} T_x fdx - h^{-1}\int_0^h T_xfdx\right\} \\
\nonumber & = & T_t f - f
\end{eqnarray}
Therefore, $f_t \in Dom(Z)$ and
\[
Z(f_t) = T_t f - f
\]
Since $t^{-1}f_t \rightarrow f$ in norm as $t\rightarrow 0^+$, we see that
$Dom(Z)$ is dense in $B$.
\qed

The generator $Z$, in general, is not a bounded operator. However, we can prove
the following
\begin{lemma}
The generator $Z$ is a closed operator.
\end{lemma}
{\bf Proof.}
We first observe that
\[
T_t f - f = \int_0^t T_x Z f dx
\]
if $f\in Dom(Z)$. To see this, we consider the function 
$r(t) = T_t f - f - \int_0^t T_x Zfdx$. Obviously we have $r(0) = 0$, and
$r'(t)\equiv 0$. Thus $r(t)\equiv 0$.
\qed

Using the above formula, we have
\[
T_f f - f = \lim_{n\rightarrow\infty}(T_t f_n - f_n) = \lim_{n\rightarrow\infty}
\int_0^t T_x Z f_n dx
\]
By the Lebegue theorem, the above limit is equal to
\[
\int_0^t T_x g dx
\]
Thus we have
\[
\lim_{t\rightarrow 0^+}t^{-1}(T_t f - f) = g
\]
and therefore $f\in Dom(Z)$, $Zf = g$.
\\

\begin{lemma}
If $B$ is a Hilbert space, then $Z$ must be densely defined and self-adjoint.
\end{lemma}

Let $M$ be a manifold of dimension $n$, not necessarily compact or complete. The
semi-group can formally be defined as
\[
e^{\Delta t}
\]

More precisely, the following result is true
\begin{theorem}
Let $M$ be a manifold, then there is a heat kernel
\[
H(x,y,t) \in C^{\infty}(M\times M\times \mathbb{R}^t)
\]
such that
\[
(T_t f)(x) = \int_M H(x,y,t)f(y)dy
\]
satisfying
\begin{itemize}
\item[\ding{172}.] $H(x,y,t) = H(y,x,t)$.
\item[\ding{173}.] $\lim_{t\rightarrow 0^+} H(x,y,t) = \delta_x(y)$.
\item[\ding{174}.] $(\Delta - \frac{\partial}{\partial t})H = 0$.
\item[\ding{175}.] $H(x,y,t) = \int_M H(x,z,t-s)H(z,y,s)dz$.
\end{itemize}
\end{theorem}

In [Getzler], the above theorem was proved. One of the feature of the above
theorem is that the proof is independent to the fact that $\Delta$ can be
extended as a densely defined self-adjoint operator on $L^2(M)$. In particular,
we don't need to assume $M$ to be complete. The infinitesimal generator on
$L^2(M)$ is in fact the Dirichlet Laplacian.
\\

We let $\Delta_p$ denote the Laplacian on $L^p$ space. With this notation, for
most of the theorems in linear differential geometry, the completeness 
assumption can be removed.

\begin{example}
Let $f\in Dom(\Delta_2)$ such that $f\in L^2(M)$ and $\Delta f = 0$. Then $f$
has to be a constant.
\end{example}

When $M$ is a complete manifold, the above is a theorem of Yau. However, it is
interesting to see that even when $M$ is incomplete, the above result is still
true, and the proof is exactly the same as the original proof of Yau.
\\

Examining some special cases of the above setting is interesting.
\begin{itemize}
\item[\textcircled{A}.] If $\partial M\neq \emptyset$ and if $\partial M$ is an
$(n-1)$-dimensional manifold, then $A_2$ is the Dirichlet Laplacian.
\item[\textcircled{B}.] If $M = \mathbb{R}^n - \{0\}$. Then if $f\in L^2(M)$, 
$\Delta f = 0$ and $f \in Dom(\Delta_2)$. Then $f(0) = 0$ and $f$ must be
bounded near $0$. By the removable singularity theorem, $f$ extends to a 
harmonic function on $\mathbb{R}^n$, which must be a constant.
\item[\textcircled{C}.] $\Delta_2$ is particularly useful on moduli spaces,
where it is very difficult to describe the boundary.
\end{itemize}

\subsubsection{Variational characterization of spectrum}

Unlike in the case of compact manifold, in general, a complete manifold doesn't
admit any pure point spectrum. For example, there are no $L^2$-eigenvalues on
$\mathbb{R}^n$. That is, for any $\lambda\in\mathbb{R}$, if 
$\Delta f + \lambda f = 0$ and $f \in L^2(\mathbb{R}^n)$, then we have 
$f\equiv 0$.
\\

The above well-known result was generalized by Escobar, who proved that if $M$
has a rotational symmetric metric, then there is no $L^2$-eigenvalue.
\\

Let $\Delta$ be the Laplacian on a complete non-compact manifold $M$. By the
argument in the previous section, $\Delta$ naturally extends to a self-adjoint
densely defined operator, which we still denote as $\Delta$ for the sake of
simplicity.
\\

It is well-known that there is a spectrum measure $E$ such that
\[
-\Delta = \int_0^{\infty} \lambda dE
\]
The heat kernel is defined as
\[
e^{\Delta t}f(x) = \int H(x,y,t)f(y)dy
\]
and the Green's function is defined as
\[
G(x,y) = \int_0^\infty H(x,y,t)dt
\]

The pure-point spectrum of $\Delta$ are these $\lambda\in\mathbb{R}$ such that
\begin{itemize}
\item[\ding{172}] There exists an $L^2$ function $f\neq 0$ such that
\[
\Delta f + \lambda f = 0
\]
\item[\ding{173}] The multiplicity of $\lambda$ is finite.
\item[\ding{174}] In a neighborhood of $\lambda_1$ it is the only spectrum 
point.
\end{itemize}

We define
\[
\rho(\Delta) = \{y\in\mathbb{R}|(\Delta-y)^{-1}\mbox{ is a bounded operator}\}
\]
and we define $\sigma(\Delta) = \mathbb{R} - \rho(\Delta)$ to be the spectrum
of $\Delta$. From the above discussion, $\sigma(\Delta)$ decomposes as the
union of pure point spectrum, and the so-called essential spectrum, which is,
by definition, the complement of the pure-point spectrum.
\\

The set of the essential spectrum is denoted as $\sigma_{ess}(\Delta)$. Using
the above definition, $\lambda\in\sigma_{ess}(\Delta)$, if either
\begin{itemize}
\item[\ding{172}] $\lambda$ is an eigenvalue of infinite multiplicity, or
\item[\ding{173}] $\lambda$ is the limiting point of $\sigma(\Delta)$.
\end{itemize}

The following theorems in functional analysis are well-known. For reference,
see Donnelly.

\begin{theorem}
A necessary and sufficient condition for the interval $(-\infty, \lambda)$ to
intersect the essential spectrum of an self-adjoint densely defined operator
$A$ is that, for all $\varepsilon > 0$, there exists an infinite dimensional
subspace $G_\varepsilon \subset Dom(A)$, for which
\[
(Af - \lambda f - \varepsilon f, f) < 0
\]
\end{theorem}

\begin{theorem}
A necessary and sufficient condition for the interval $(\lambda-a,\lambda+a)$
to intersect the essential spectrum of $A$ is that there exists an infinite
dimensional subspace $G\subset Dom(A)$ for which 
$\|(A-\lambda I)f\|\leqslant a \|f\|$ for all $f\in G$.
\end{theorem}

Using the above result, we give the following variational characterization of
the lower bound of spectrum and the lower bound of essential spectrum.

\begin{theorem}
Using the above notations, define
\[
\lambda_0 = \inf_{f\in C_0^\infty(M)}\frac{\int_M |\nabla f|^2}{\int_M f^2}
\]
and
\[
\lambda_{ess} = \sup_K \inf_{f\in C_0^\infty(M\backslash K)}\frac{\int_M |\nabla f|^2}{\int_M f^2}
\]
where $K$ is a compact set running through an exhaustion of the manifold. Then
$\lambda_0$ and $\lambda_{ess}$ are the least lower bound of $\sigma(\Delta)$
and $\sigma_{ess}(\Delta)$, respectively.
\end{theorem}

\begin{corollary}
If $\lambda_0 < \lambda_{ess}$, then $\lambda_0$ is an eigenvalue of $M$ with
finite dimensional eigenspace.
\end{corollary}

In this case, $\lambda_0$ is called the ground state.
\\

In the following, we give a non-trivial application of the above principle.

\begin{theorem}
[Lin-Lu] Let $M$ be a complex complete surface embedded in $\mathbb{R}^3$.
Assume that $M$ is not totally geodesic, but asymptotically flat in the sense
that the second fundamental form goes to zero at infinity. Define
\[
\Omega = \{y\in\mathbb{R}^3 | d(y,M) \leqslant a \}
\]
for a small positive number $a > 0$. Then $\Omega$ is a $3$-d manifold with
boundary. The Dirichlet Laplacian of $\Omega$ has a ground state.
\end{theorem}

Sketch of the proof: Since $M$ is asymptotically flat, at infinity
\[
\Omega \approx M \times [-a, a]
\]
As a result
\[
\lambda_{ess} = \frac{\pi^2}{4a^2}
\]
Thus the main difficulty in the proof of the above theorem is to prove
\[
\lambda_0 < \frac{\pi^2}{4a^2}
\]
which can be obtained by careful analysis of the Gauss and the mean curvatures.

\begin{remark}
Exner \textit{et al.} proved that under the condition
\[
\int K \leqslant 0, \int |K| < \infty
\]
and $M$ being asymptotically flat, the ground state exists. Thus we make the
following conjecture to give the complete picture.
\end{remark}

\begin{conjecture}
Let $M$ be a complete, no-totally geodesic, and asymptotically embedded surface
in $\mathbb{R}^3$. Let $\Omega$ be defined as before. Let $K$ be the Gauss
curvature. If
\[
\int_M |K| < +\infty
\]
then the ground state exists.
\end{conjecture}

The difficulty of the above conjecture is that even the surface is 
asymptotically flat, we still don't known the long-range behavior of the
surface.

\subsection{On the theorem of Sturm}

Let $M$be a complete Riemannian manifold. We say that the volume $(M,g)$ grows
uniformly sub-exponentially, if for any $\varepsilon > 0$, there is a constant
$C < \infty$ such that for all $r > 0$ and all $x\in M$, we have
\[
v(B_r(x))\leqslant Ce^{\varepsilon r}v(B_1(X))
\]

\begin{theorem}
[Sturm] If the volume of $(M,g)$ grows uniformly and sub-exponentially, then
the spectrum $\sigma(\Delta_p)$ of $\Delta_p$ acting on $L^p(M)$ is independent
of $p\in[1, \infty)$. In particular, it is a subset of the real line.
\end{theorem}

One feature of the concept ``uniformly and sub-exponentially" is that it is
self-dual. Take the following example: A hyperbolic space is not ``uniformly
and sub-exponentially". On the other side, let $\Gamma$ be a discrete group
acting on the hyperbolic space $H$, such that $\Gamma\backslash H$ has finite
volume. Since the infinity of $\Gamma\backslash H$ are cusps, it is still not
``uniformly and sub-exponentially".
\\

A manifold with non-negative Ricci curvature satisfies the assumption that the
volume grows ``uniformly and sub-exponentially". However, for such a manifold,
its volume is infinite. It doesn't has the finite volume counterpart.
\\

The proof of Sturm's theorem depends on the heat kernel estimates. We begin
with the following
\begin{lemma}
If the volume of $(M,g)$ grows uniformly and sub-exponentially, then for any
$\varepsilon > 0$
\[
\sup_{x\in M}\int_M e^{-\varepsilon 
d(x,y)}(v(B_1(x))^{-\frac{1}{2}}v(B_1(y))^{-\frac{1}{2}} dv(y) < \infty
\]
\end{lemma}
{\bf Proof.}
We take $r = d(x,y)$. Then since
\[
B_1(y) \subset B_{r+1}(x)
\]
we must have
\[
v(B_1(y))^{-\frac{1}{2}} \geqslant v(B_{r+1}(x))^{-\frac{1}{2}} \geqslant
Ce^{-\frac{1}{2}(r+1)}v(B_1(x))^{-\frac{1}{2}}
\]
for any $x$. Thus the integration in the lemma is less than
\[
C\int_M e^{-\varepsilon r}e^{\frac{1}{2}\varepsilon(r+1)} v(B_1(x))^{-1} dy
\]
We let $f(r) = v(\partial B_r(x))$ and $F(r) = \int_0^r f(t)dt$. Then up to a
constant, the above expression is less than
\[
(v(B_1(x)))^{-1} \int_0^\infty e^{-\frac{1}{2}\varepsilon r} f(r)dr
\]
By the volume growth assumption $f(r) \leqslant Cr^{n-1} v B_1(x)$, the lemma
follows.
\qed

In fact, the assertion is true if
\begin{itemize}
\item[\ding{172}] The Ricci curvature of $M$ has a lower bound;
\item[\ding{173}] \[
\sup_{x\in M} \int_M e^{-\varepsilon d(x,y)}(v(B_1(x))^{-\frac{1}{2}}
v(B_1(y))^{-\frac{1}{2}} dv(y) < \infty
\]
\end{itemize}

The hard part is to prove $\sigma(\Delta_p) \subset \sigma(\Delta_2)$ for all
$p\in[1,\infty]$. If this is done, then it is easy to prove
\[
\sigma(\Delta_2) \subset \sigma(\Delta_p)
\]
as follows:
\\

Let $\xi\in\rho(\Delta_p)$. Then $(\Delta_p - \xi)^{-1}$ is a bounded operator
on $L^p(M)$. Let $\frac{1}{p} + \frac{1}{q} = 1$. By dualization 
$(\Delta_q - \zeta)^{-1}$ is bounded in $L^q(M)$. By the interpolation theorem,
$(\Delta_2 - \zeta)^{-1}$ is bounded and this $\zeta\in\rho(\Delta_2)$.
\\

In order to prove $\sigma(\Delta_p)\subset\sigma(\Delta_2)$, or 
$\rho(\Delta_2)\subset\rho(\Delta_p)$, we need some estimates. Let 
$\zeta\in\rho(\Delta_2)$. Then
\[
(\Delta_2 - \zeta)^{-1}
\]
is bounded from $L^2 \rightarrow L^2$. In order to prove that the operator is
bounded on $L^p$, we need to prove that it has a kernel $g(x,y)$ such that
\[
(\Delta_2 - \zeta)^{-n} f = \int g(x,y)f(y)dy
\]

\begin{lemma}
If $g(x,y)$ satisfies
\[
\sup_{x\in M}\int_M |g(x,y)|dy \leqslant C
\]
Then $(\Delta_2 - \zeta)^{-n}$ is a bounded operator on $L^p(M)$.
\end{lemma}
{\bf Proof.}
This is essentially H\"{o}lder inequality:
\begin{eqnarray}
\nonumber & & \int_M \left(\int_M g(x,y)f(y)dy\right)^p dx \\
\nonumber & \leqslant 
& \int_M \left(\int_M g^{\frac{1}{q}}g^{\frac{1}{p}}fdy\right)^p dx \\
\nonumber & \leqslant & \int_M\left(\int_M g\right)^{\frac{1}{q}} \cdot 
\int gf^p dx \\
\nonumber & \leqslant & C^{\frac{1}{q}} \int_M\int_M g(x,y)f^p(y)dydx \\
\nonumber & \leqslant & C^{1+\frac{1}{q}}\int_M f^p(y)dy
\end{eqnarray}
\qed

If we assume that $\sigma(\Delta_p)$ is a no-where dense set in $\mathbb{C}$,
then we have
\begin{lemma}
If $(\Delta_2 - \zeta)^{-n}$ is bounded, so is $(\Delta_2 - \zeta)^{-1}$.
\end{lemma}
{\bf Proof.}
For any $\varepsilon$, let $\zeta '\in\rho(\Delta_\rho)$ and 
$|\zeta - \zeta '| < \varepsilon$. Then from
\[
\|(\Delta_2 - \zeta)^{-n}\| \leqslant C
\]
we get
\[
\|(\Delta_2 - \zeta ')^{-n}\| \leqslant C+1
\]
provided that $\varepsilon$ is small enough. Let 
$dist(\zeta ', \sigma(\Delta_p))$ be the distance to the spectrum of $\Delta_p$,
then we have
\[
C+1 \geqslant \lim_{m\rightarrow \infty} \|(\Delta_2 - 
\zeta)^{-nm}\|^{\frac{1}{m}} \geqslant dist(\zeta ', \sigma(\Delta_p))^{-n}
\]
Thus
\[
dist(\zeta ', \sigma(\Delta_p)) \geqslant \frac{1}{(C+1)^{\frac{1}{n}}}
\]
Since $\zeta '$ is arbitrary, we have 
$dist(\zeta, \sigma(\Delta_p)) > \delta > 0$.
\qed
%**********************Lecture 4************************************************
\section{On the essential spectrum of complete non-compact manifold}\label{Spec_II}
Let  $M$ be a complete non-compact manifold.
We assume that  there exists a small constant $\delta(n) > 0$, depending only on
$n$ such that for some point $q\in M$, the Ricci curvature satisfies
\[
Ric(M) \geqslant -\delta(n)\frac{1}{r^2}
\]
where $r(x)$, the distance from $x$ to $q$ is sufficiently large.
J-P. Wang (cite Wang) proved the following theorem:
\begin{theorem}
Let $M$ be the complete non-compact Riemannian manifold defined above. Then the spectrum of the
Laplacian $\Delta_p$ acting on the space $L^p(M)$ is $[0, \infty)$ for all
$p\in[1,\infty)$.
\end{theorem}
\begin{corollary}
Let $M$ be a complete manifold with non-negative Ricci curvature, then the $L^2$
essential spectrum of the Laplacian is $[0, +\infty)$.
\end{corollary}



By the Bishop volume comparison theorem, we know that for any complete 
non-compact manifold with non-negative Ricci curvature, the volume growth is at
most polynomial. In general, it is not correct to have the lower bound estimate.
However, we have the following:
\begin{theorem}
Let $M$ be a complete non-compact Riemannian manifold, and let 
$Ric(M)\geqslant 0$. Then there is a constant $C = C(n,v(B_1(p)))$ such that
\[
v(B_p(R)) \geqslant C(n,v(B_1(p)))R.
\]
\end{theorem}
{\bf Proof.}
Let $p\in M$ be a fixed point. Let $\rho$ be the distance function with respect
to $p$, Let $R>0$ be a large number. Fixing $x_0\in \partial B_R(p)$. By the 
Laplacian comparison theorem, we have
\[
\Delta\rho^2 \leqslant 2n.
\]
It follows that for any $\varphi\in C_0^\infty(M)$, $\varphi\geqslant 0$, we
have
\begin{equation}\label{4-1}
\int_M \varphi\Delta\rho^2 \leqslant 2n\int_M\varphi.
\end{equation}
We choose a standard cut-off function $\varphi=\psi(\rho(x))$, where
\[
\psi(t) = \left\{\begin{array}{lcl}
1 & & 0 \leqslant t \leqslant R-1 \\
\frac{1}{2}(R+1-t) & & R-1 \leqslant t \leqslant R+1 \\
0 & & t \geqslant R+1
\end{array}
\right..
\]
By the Stoke's theorem
\[
\int_M \varphi\Delta\rho^2 = -2\int_M \rho\nabla\varphi\nabla\rho =
-2\int_M \psi ' \rho.
\]
By the definition of $\psi$, the right hand side of the above is equal to
\[
\int_{B_{R+1}(x_0)\backslash B_{R-1}(x_0)} \rho \geqslant
(R-1)v(B_{R+1}(x_0)\backslash B_{R-1}(x_0)).
\]
Combining the above equation with $\circledast$, we have
\[
(R-1)v(B_{R+1}(x_0)-B_{R-1}(x_0)) \leqslant 2n\int\varphi \leqslant 
2nv_{R+1}(x_0).
\]
Obviously, we have
\[
B_1(p) \subset B_{R+1}(x_0)\backslash B_{R-1}(x_0).
\]
Thus we have
\[
2nv_{R+1}(x_0)\geqslant (R-1)v_1(p).
\]
Since $B_{2(R+1)}(p) \supset B_{R+1}(x_0)$, we have
\[
2nv_{2(R+1)}(p) \geqslant (R-1)v_1(p).
\]
or in other word,
\[
v_{2(R+1)}(p) \geqslant \frac{R-1}{2n}v_1(p).
\]
\qed

What Wang observed was the following inverse Laplacian comparison theorem: We
don't have a lower bound for the Laplacian. However, we have the following:
\begin{eqnarray}
\nonumber \int_{B(R)\backslash K} |\Delta\rho| & \leqslant &
\int_{B(R)\backslash K} \partial n + \int_{B(R)\backslash K} |\partial n - 
\Delta \rho| \\
\nonumber & \leqslant & CR^n - \int_{\partial B(R)}\frac{\partial\rho}
{\partial n} + \int_{\partial K}\frac{\partial\rho}{\partial n} \\
\nonumber & \leqslant & CR^n.
\end{eqnarray}
Thus we can also estimate $\int\Delta\rho$ from below.
\\

Using the above observation, Wang computed the $L^1$-spectrum. Using the theorem
of Sturm, all $L^p$-spectrum, in particular the $L^2$-spectrum we are 
interested, are the same.
\\

It is possible to compute the $L^2$-spectrum directly, but that would be more or
less the same as repeating the proof of Sturm's theorem. In fact, we can get a
little more information than Wang's theorem provided.

\begin{lemma}
Let $M$ be a complete non-compact manifold with non-negative Ricci curvature.
Let $B(R)$ be a very large ball of radius $R$. Let $\lambda$ be a Dirichlet
eigenvalue and let $f$ be its eigenfunction of $B(R)$. Then there is a constant
$C > 0$ such that
\[
\int_{B(R)\cdot B(R-1)} f^2 \leqslant C\int_{B(R)} f^2.
\]
\end{lemma}

For the rest of this lecture we are seeking possible extensions of Wang's
theorem. While we observe that Sturm's theorem is self-dual ($M$ could be of
infinite volume or finite volume), Wang's theorem is not. In what follows, we
shall construct an example that all $L^p$-spectrum are the same of finite 
volume, $L^1$-spectrum computable, but doesn't satisfy the assumption of Wang.
\\
% figure

The manifold we construct is of $2$ dimensional rotational symmetric outside a 
compact set, and the Riemannian metric $g$ can be written as 
\[
% g = dr^2 + f^2(r) d\theta^2, r > 1
g = dr^2 + f(r)^2 d\theta^2, r > 1
\]
where $f(r) = \frac{1}{r^\alpha}$ for some $\alpha$ large.
\\

The Gauss curvature of $g$ is $-f''/f = -\alpha(\alpha + 1)\frac{1}{r^2}$. Thus
the manifold doesn't satisfy Wang's assumption.
\\

We prove that the volume of $M$ grows uniformly and sub-exponentially. To see
this, we observe that for any point $(x)$
\[
v(B_1(x)) \geqslant \frac{C}{r(x)^\alpha}
\]
for some constant $C > 0$. On the other hand, if $\alpha > 1$, the volume of
manifold is finite. Thus
\[
v(B_r(x)) \leqslant C \leqslant e^{\varepsilon r}\frac{C(\varepsilon)}{r^\alpha}
\]
for any $r \gg 0$.
\\

Thus the manifold satisfies the assumption of Sturm and as a result, all 
$L^p$-spectrum of $M$ are the same.
\\

We compute the $L^1$-spectrum concretely. Following Wang, we pick up a large
number $k$. Let $\psi$ be a cut-off function whose support is in $[1,4]$, and
is identically $1$ on $[2,3]$. Consider the function
\[
g = \psi(\frac{r}{k}) e^{i\sqrt{\lambda}r}.
\]
We have
\[
\Delta g = \Delta \psi e^{i\sqrt{\lambda}r} + 2\nabla\psi\nabla 
e^{i\sqrt{\lambda}r} + \psi\Delta e^{i\sqrt{\lambda}r}.
\]
We have the following estimate
\[
\|\nabla\psi\nabla e^{i\sqrt{\lambda}r}\|_{L^1} \leqslant \frac{C}{K}
(V(4k) - V(k)).
\]
where $V(r)$ is the volume of the manifold of radius $r$. A straightforward
computation gives
\[
\|\nabla\psi\nabla e^{i\sqrt{\lambda}r}\|_{L^1} \leqslant \frac{C}{k^\alpha}.
\]
By the same reason
\[
\|\Delta\psi e^{i\sqrt{\lambda}r}\|_{L^1} \leqslant \frac{C}{k^\alpha} +
\frac{C}{k}\int_{B(4k)\backslash B(k)}|\Delta r|.
\]
Since $dS^2 = dr^2 + f(r)^2 d\theta^2$, we have
\[
|\Delta r| \leqslant \frac{\alpha}{r}.
\]
Thus
\[
\int_{B(4k)\backslash(k)}|\Delta r| = \int_k^{4k}\frac{\alpha}{r^{\alpha+1}}dr
\leqslant \frac{C}{k^\alpha}.
\]
Finally
\[
\|\psi\Delta e^{i\sqrt{\lambda}r} + \lambda g\|_{L^1} =
\|\sqrt{\lambda}\psi e^{i\sqrt{\lambda}r} \Delta r\|_{L^1} \leqslant
\frac{C}{k^\alpha}.
\]
On the other hand
\[
\|g\|_{L^1} \geqslant \int_{2k}^{3k} 1 \geqslant V(3k) - V(2k) \geqslant
\frac{C_1}{k^{\alpha-1}}.
\]
Thus if $k$ is sufficiently large
\[
\|\Delta g + \lambda g\|_{L^1} \leqslant \varepsilon\|g\|_{L^1}
\]
Thus there should be a finite volume version of Wang's theorem.
\\

We end the lecture by some speculations of the essential spectrum.
\\

\begin{definition}
A discrete group $G$ is called amenable, if there is a measure such that
\begin{enumerate}
\item The measure is a probability measure;
\item The measure is finitely additive;
\item The measure is left-invariant: given a subset $A$ and an element $g$ of
$G$, the measure of $A$ equals to the measure of $gA$.
\end{enumerate}
\end{definition}

In one sentence, $G$ is amenable if it has finitely-additive left-invariant
probability measure.
\\

The following theorem of R. Brooks~\cite{brooks} is remarkable:
\begin{theorem}
([Brooks] Let $M$ be a compact Riemannian manifold and let $\widetilde M$ be the
universal cover of $M$. We assume that $\widetilde M$ is non-compact, then
\[
\lambda_0(\widetilde M) = 0 \Leftrightarrow \pi_1(M) \mbox{ is amenable}.
\]
\end{theorem}

It would be interesting to ask
\begin{conjecture}
Using the same assumptions as above. Then
\[
\sigma_{ess}(\widetilde M) = [0, \infty).
\]
\end{conjecture}

In the case when $\pi_1(M) = \mathbb{Z}^n$, the above conjecture is true.

\begin{lemma}
Suppose $M = T^n$, $\widetilde M = \mathbb{R}^n$. Then
\[
\sigma_{ess}(\widetilde M) = [0, \infty)
\]
\end{lemma}
Here the metric on $M$ is an arbitrary metric.
{\bf Proof.}
Let $N$ be any finite cover of $M$. Let $\lambda$ be an eigenvalue of $N$. Then
\[
\lambda\in\sigma_{ess}(\widetilde M)
\]
In fact, let $\rho$ be a cut-off function. Since
\[
\Delta f + \lambda f = 0 \mbox{ on } N
\]
Then on $\widetilde M$
\[
\|\Delta(\rho f) + \lambda\rho f\|_{L^2} \leqq \|f\Delta\rho\|_{L^2} +
\|\alpha\nabla\rho\nabla f\|_{L^2}
\]
If $|\nabla\rho|$, $|\Delta\rho|$ are small, then
\[
\|\Delta(f\rho) + \lambda\rho f\|_{L^2} \leqslant \varepsilon \|f\rho\|_{L^2}
\]

If the result is not true, since $\sigma_{ess}(\Delta)$ is a closed set, there
is an interval $(a,b)$ such that for any $N$, there is no eigenvalues in
$(a,b)$.
\\

We prove this by contradiction. Let $\lambda$, $\mu$ be two consecutive
eigenvalues such that $\lambda < a$ and $\mu > b$. By the above argument, we
can find a $C_0^\infty$ function on $\widetilde M$ such that
\begin{eqnarray}
\nonumber \|\Delta f + \lambda f\|_{L^2} & < & \varepsilon\|f\|_{L^2} \\
\nonumber \|\Delta g + \mu g\|_{L^2} & < & \varepsilon\|g\|_{L^2}
\end{eqnarray}
Let $k,l$ be integers such that
\[
\frac{k\int|\nabla f|^2 + l\int|\nabla g|^2}{k\int f^2 + l\int g^2}\in(a,b)
\]
Then by repeating $f$ $k$-times and $g$ $l$-times we are done.
\qed


\begin{remark}
Recently, Lu-Zhou~\cite{lu-zhou} proved that the essential spectrum is $[0,+\infty)$ for any complete non-compact manifold with asymptotic nonnegative Ricci curvature, generalizing Wang's result.
\end{remark}






