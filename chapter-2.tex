
\chapter{The Laplacian}\label{Laplacian}
 
\section{The Laplace operator}\label{Laplace_operator}
Let $M$ be a compact orientable Riemannian manifold and let $\Lambda^p(M)$ be the vector space of $p$ forms. As in the last section, we use $\omega_1,\cdots,\omega_n$ to be the local orthonormal frame, and $\omega_1\wedge\cdots\wedge\omega_n$ to be  the volume form.

We write a $p$-form in the following 
expression
\[
\eta=a_{i_1\cdots i_p}\,\omega_{i_1}\wedge\cdots\wedge\omega_{i_p},
\]
where we assume the coefficients are skew-symmetric. That is, we assume that
\[
a_{\sigma(i_1)\cdots\sigma(i_p)}=(-1)^{sgn\,\sigma}a_{i_1\cdots i_p}
\]
for any permutation $\sigma$ of $\{1,\cdots,p\}$.
The $L^2$ inner product is defined as
\begin{equation}\label{p-3}
(\omega,\eta)=p!\int_M a_{i_1\cdots i_p}b_{i_1\cdots i_p}\omega_1\wedge\cdots\wedge\omega_n.
\end{equation}
We remark that in the above definition, we only require one of the coefficients to be skew-symmetric, which is useful in the following computation.


\begin{ex} Prove the following: if 
\[
\eta=b_{i_1\cdots i_p}\omega_{i_1}\wedge\cdots\wedge\omega_{i_p},
\]
where $\eta=b_{i_1\cdots i_p}$ are not assumed to be skew-symmetric. Then we still have
\[
(\omega,\eta)=p!\int_M a_{i_1\cdots i_p}b_{i_1\cdots i_p}\omega_1\wedge\cdots\wedge\omega_n.
\]
\end{ex}

\begin{lemma} We have
\[
d\omega=(-1)^pa_{i_1\cdots i_p,k}\,\omega_{i_1}\wedge\cdots\wedge\omega_{i_p}\wedge\omega_k.
\]
Since the coefficients above are not skew symmetric, we may also rewrite the expression as
\[
d\omega=\frac{(-1)^p}{p+1}\left(
a_{i_1\cdots i_p,k}-\sum_{s=1}^p a_{i_1\cdots \underset{\underset{sth}{\uparrow}}k\cdots i_p, i_s}\right)
\omega_{i_1}\wedge\cdots\wedge\omega_{i_p}\wedge\omega_k.
\]
\end{lemma}

\begin{proof}
 The proof is through a straightforward computation:
\[
d\omega=da_{i_1\cdots i_p}\wedge\omega_{i_1}\wedge\cdots\wedge\omega_{i_p}+\sum_{s=1}^p(-1)^{s-1}
a_{i_1\cdots i_p}\wedge\omega_{i_1}\wedge\cdots\wedge d\omega_{i_s}\wedge\cdots\wedge\omega_{i_p}.
\]
Using Cartan's formula $d\omega_{i_s}=-\omega_i\wedge\omega_{ii_s}$, we get the first formula. The second formula follows from the  skew-symmetrization.

\end{proof}




\begin{lemma} Let $\delta$ be the dual operator of $d$ with respect to the inner product~\eqref{p-3}.  We then have
\[
\delta\omega=(-1)^pp\,a_{i_1\cdots i_p,i_p}\omega_{i_1}\wedge\cdots \wedge\omega_{i_{p-1}}.
\]
\end{lemma}

\begin{proof}  Let 
\[
\eta=b_{i_1\cdots i_{p-1}}\omega_{i_1}\wedge\cdots\wedge\omega_{i_{p-1}}
\]
be a smooth $(p-1)$-form with compact support.  We verify that 
\[
(\delta\omega,\eta)-(\omega,d\eta)=0.
\]
We have
\[
(\delta\omega,\eta)=(-1)^{p}p!\int_M a_{i_1\cdots i_{p-1}i_p,i_p}b_{i_1\cdots i_{p-1}},
\]
and
\[
(\omega,d\eta)=(-1)^{p-1}p!\int_M a_{i_1\cdots i_p}b_{i_1\cdots i_{p-1},i_p}.
\]
define an $(n-1)$ form $\alpha$ such that
\[
\alpha=\sum_{k=1}^n (-1)^{k-1}a_{i_1\cdots i_{p-1} k}b_{i_1\cdots i_{p-1}}\,\omega_1\wedge\cdots\hat{\omega}_k\cdots\wedge\omega_n.
\]
Then we have
\[
d\alpha=(a_{i_1\cdots i_{p-1}i_p,i_p}b_{i_1\cdots i_{p-1}} +a_{i_1\cdots i_p}b_{i_1\cdots i_{p-1},i_p})\,\omega_1\wedge\cdots\wedge\omega_n.
\]
The lemma follows from the Stokes' theorem
\[
\int_M d\alpha=0.
\]

\end{proof}








\begin{theorem}[Weitzenb\"ock formula]\label{thm6}
 We have
\[
\Delta\,\omega=\left(-a_{i_1\cdots i_p,k,k}+p\sum_{s=1}^{p-1}a_{i_1\cdots \underset{\underset{sth}{\uparrow}}r\cdots i_{p-1} k}R_{ri_si_pk}+p \, a_{i_1\cdots i_{p-1} r}(Ric)_{r i_p}\right)\omega_{i_1}\wedge\cdots\wedge\omega_{i_p}.
\]
\end{theorem}

\begin{proof} By the above two lemmas we have


\[
\delta d\omega=\left(-a_{i_1\cdots i_p,k,k}+\sum_{s=1}^p a_{i_1\cdots \underset{\underset{sth}{\uparrow}}k\cdots i_p, i_s,k}\right)
\omega_{i_1}\wedge\cdots\wedge\omega_{i_p},
\]
and
\[
d\delta\omega=-p\, a_{i_1\cdots i_p,i_p,k}\omega_{i_1}\,\wedge\cdots\wedge\omega_{i_{p-1}}\wedge\omega_k.
\]

By changing the indices and using the skew-symmetry of $a_{i_1\cdots i_p}$ in the above, we get
\[
\delta d\omega=\left(-a_{i_1\cdots i_p,k,k}+p a_{i_1\cdots i_{p-1}k, i_p,k}\right)
\omega_{i_1}\wedge\cdots\wedge\omega_{i_p},
\]
and
\[
d\delta\omega=-p\, a_{i_1\cdots k,k,i_p}\,\omega_{i_1}\,\wedge\cdots\wedge\omega_{i_{p-1}}\wedge\omega_{i_p}.
\]
The theorem follows by applying Theorem~\ref{thm5}.

\end{proof}

Define the raw Laplacian $\nabla^*\nabla$  on a $p$-form to be
\[
\nabla^*\nabla\omega=-\nabla_k\nabla_k\omega.
\]

Then the Weitzenb\"ock formula can be written as
\[
\Delta\omega=\nabla^*\nabla\omega+E(\omega),
\]
where $E$ is a $0$-th order differential operator depending on the curvature.


We would like to list the following special cases of the above theorem as exercises.


\begin{ex}
Using the above notations, we have
\[
\delta=(-1)^{n(p+1)+1}*d*
\]
on $\Lambda^p(M)$.
\end{ex}

\begin{ex} Let $\{e_1,\cdots, e_n\}$ be an orthonormal frame. Then
\[
\delta=-\sum_{j=1}^n\iota(e_j)\nabla_{e_j}.
\]
\end{ex}


\begin{ex} If $f$ is a smooth function, then
\[
\Delta f=-f_{,k,k}=-f_{kk}.
\]
\end{ex}

\begin{ex} If $\omega$ is a one-form, then
\[
\Delta\omega=\nabla^*\nabla\omega+a_r(Ric)_{rk}\omega_k.
\]
\end{ex}



\section{Self-adjoint extension of the Laplace operator}\label{Operator_extension}
In this section, we assume that $M$ is Riemannian manifold, not necessarily compact nor complete. 

Let $\Lambda^p(M)$ be the space of smooth $p$-forms and 
let $L^2(\Lambda^p(M))$ be the metric space completion of $\Lambda^p(M)$ under the inner product product defined in ~\eqref{p-3}. 

\begin{ex}
Prove that we can identify $L^2(\Lambda^p(M))$ to be the space of $p$-forms $\omega$, where
\[
\omega=a_{i_1\cdots i_p}\omega_{i_1}\wedge\cdots\wedge\omega_{i_p},
\]
such that $a_{i_1\cdots i_p}$ are locally $L^2$ integrable functions.
\end{ex}

The reason we would like to use $L^2(\Lambda^p(M))$ in stead of $\Lambda^p(M)$ is, of course, that 
$L^2(\Lambda^p(M))$ is complete, allowing many applications in analysis.

Unfortunately, it is not possible to extend $\Delta$ to $L^2(\Lambda^p(M))$ as a symmetric linear operator. In what follows, we briefly explain the reason.


First, the Laplace operator is not a bounded operator on $\Lambda^p(M)$. We can give counterexamples even in one-dimensional case: let $f$ be a smooth function on $[0,1]$. Then there doesn't exist a constant $C$ such that
\[
\int_0^1(f''(t))^2 dt\leq C\int_0^1(f(t))^2 dt.
\]

Second, like most differential operators, the Laplace operator is a closed-graph operator. That is, if $\eta_j\to \eta$ and $\Delta\eta_j\to\eta'$ in $L^2$ and $\eta\in\Lambda^p(M)$, then we must have $\eta'=\Delta\eta$.
To see this, we consider any smooth $p$ form $\omega$:
We have 
\[
\int\langle\eta'-\Delta\eta,\omega\rangle=\lim_{j\to\infty}\int
\langle\Delta\eta_j,\omega\rangle-\lim_{j\to\infty}\int\langle\eta_j,\Delta\omega\rangle=0.
\]
Thus we have $\eta'=\Delta\eta$.

If $\Delta$ could be extended to a linear operator of $L^2(\Lambda^p(M))$, then by the {\it Closed
Graph Theorem}, $\Delta$ could have been bounded, which is a contradiction.

Because of the above result, the best we can do is to extend the Laplacian operator into a {\it densely defined} self-adjoint operator. Of course, $\Lambda^p(M)$ is dense in $L^2(\Lambda^p(M))$. But for such an operator (i.e, $\Delta$ and its domain ${\mathcal Dom}(\Delta))$, we don't have the so-called spectral theorem. So in order to get meaningful results, we have to extend the Laplace operator first. 

\begin{definition} Let $H$ be a Hilbert space, Given a densely defined linear operator $A$ on $H$, its adjoint $A^*$ is defined as follows:
\begin{enumerate}
\item The domain ${\mathcal Dom}(A^*)$ of $A^*$ consists of vectors $x$ in $H$ such that
\[
y\mapsto \langle x, Ay\rangle
\]
is a bounded linear functional, where $y\in{\mathcal Dom}(A)$;
\item By the {\it Riesz Representation Theorem} for linear functionals, if $x$ is in the domain of $A^*$, there is a unique vector $z$ in $H$ such that
\[
\langle x,Ay\rangle=\langle z,y\rangle
\]
for any $y\in {\mathcal Dom}(A)$.
This vector $z$ is defined to be $A^*x$. It can be shown that the dependence of $z$ on $x$ is linear.
\end{enumerate}
If $A^*=A$ (which implies that ${\mathcal Dom}\, (A^*)={\mathcal Dom}\,(A)$), then $A$ is called self-adjoint.
\end{definition}




For a (densely-defined) self-adjoint operator $A$, we have the {\it Spectral Theorem}. That is, there is a spectral measure such that
\[
A=\int_{-\infty}^{+\infty}\lambda dE.
\]

A densely defined self-adjoint operator $\bar\Delta$ is called an extension of $\Delta$, if ${\mathcal Dom}\, (\Delta)\subset {\mathcal Dom}\, (\bar\Delta)$, and 
\[
\Delta=\bar\Delta\,|_{{\mathcal Dom}\, (\Delta)}.
\]


For the rest of the section, we shall study the extensions of $\Delta$. We shall prove that the extension of $\Delta$ always exists, but in general, they are not unique in general.



Define $H^1(M)$ to be the Sobolev space of the completion of $\Lambda^p(M)$ under the norm
\[
||\eta||_1=\sqrt{\int_M|\eta|^2 dV_M}+\sqrt{\int_M|\nabla\eta|^2\, dV_M}.
\]
Similarly, let $\Lambda_0^p(M)$ be the space of smooth $p$-forms with compact support and define $H_0^1(M)$ to be the Sobolev space of the completion of $\Lambda_0^p(M)$ under the above norm. Then we have
\[
H_0^1(M)\subset H^1(M).
\]

In general, the above two spaces are not equal. However, we have 

\begin{theorem}\label{thm7}
If $M$ is a complete Riemannian manifold, then
\[
H_0^1(M)= H^1(M).
\]
\end{theorem}

\begin{proof}
Let $\phi\in H^1(M)$. Then there exists a sequence $\{\phi_j\}\in \Lambda^p(M)$ such that 
\[
\phi_j\to \phi, j\to\infty
\]
in the $\|\quad\|_1$ norm. 

Let $p\in M$ be a fixed point and let $\rho$ be a smooth function on $\mathbb R$ with compact support  such that in a neighborhood of $0$, $\rho$ is the constant $1$. Let $d(x,y)$ be the distance function. Define
\[
\rho_j(x)=\rho(j^{-1}\,d(p,x)).
\]
Then we can prove that 
\[
\rho_j\,\phi_j\to\phi
\]
in the $\|\quad\|_1$ norm. 
\end{proof}


\begin{ex} Provide the detailed proof of the above theorem.\end{ex}

We define the quadratic form $Q$  by
\[
Q(\omega,\eta)=\int_M\left(\langle d\omega,d\eta \rangle +\langle\delta\omega,\delta\eta\rangle\right) dV_M
\]
for any $\omega,\eta\in H_0^1(M)$.  By the Weitzenb\"ock formula, we have

\begin{proposition}\label{prop1} If the curvature is bounded, then
$\sqrt{Q(\phi,\phi)+\|\phi\|_{L^2}^2}$ is equivalent to the $\|\phi\|_1$.
\end{proposition}


The \emph{Friedrichs extension} $\bar\Delta$ of $\Delta$ is defined by the following. Let
\begin{align*}
&
{\mathcal Dom}(\bar \Delta)=\left\{\phi\in H_0^1(M)\mid\right.\\
&\left.\forall\psi\in \Lambda^p(M), \exists f\in L^2(\Lambda^p(M)), s.t.\, Q(\phi,\psi)=( f,\psi)\right\}.
\end{align*}

\begin{lemma}
Using the above notations, we have
\[
{\mathcal Dom}(\bar\Delta)={\mathcal Dom}(\bar\Delta^*).
\]
\end{lemma}

\begin{proof} For any $\phi,\psi\in{\mathcal Dom}(\bar\Delta)$, $(\bar\Delta\phi,\psi)=Q(\psi,\phi)=(\phi,\bar\Delta\psi)$ is 
a bounded functional. Thus $\phi\in {\mathcal Dom}(\bar\Delta^*)$. On the other hand, if $\phi\in {\mathcal Dom}(\Delta^*)$, then the functional $\psi\mapsto (\Delta\psi,\phi)$ is bounded. By the Riesz representation theorem, there is a unique $f\in L^2(\Lambda^p(M))$ such that $(\bar\Delta\psi,\phi)=(f,\psi)$. Thus we must have $Q(\phi,\psi)=(\bar\Delta\psi,\phi)=(f,\psi)$, and $\phi\in {\mathcal Dom}(\bar\Delta)$.

\end{proof}


\begin{theorem} The Laplace operator has a self-adjoint extension.
\end{theorem}



\begin{theorem}
Let $M$ be a complete Riemannian manifold. Then the extension of the Laplace operator is unique.
\end{theorem}

\begin{proof}
Let $\Delta_2$ be another self-adjoint extension of $\Delta$. Then for $\phi\in {\mathcal Dom}(\bar\Delta)$ the functional 
\[
\psi\mapsto(\phi,\Delta\psi)+(\phi,\psi)=(\Delta_2\phi,\psi)+(\phi,\psi)
\]
for $\psi\in\Lambda_0^p(M)$ is bounded under the norm $\|\quad\|_1$. By Proposition ~\ref{prop1}, it is also bounded under the norm
\[
\sqrt{Q(\cdot,\cdot)+(\cdot,\cdot)}.
\]
By the Riesz representation theorem, there exists an element $\mu\in H^1(M)$ such that 
\[
(\phi,\Delta\psi)+(\phi,\psi)=Q(\mu,\psi)+(\mu,\psi).
\]
Therefore $\phi=\psi\in H^1(M)$. Since $M$ is complete, By Theorem~\ref{thm7}, $\phi\in H_0^1(M)$. For any $\psi\in\Lambda_0^p(M)$, we have
\[
(\Delta_2\phi,\psi)=(\phi,\Delta\psi)=Q(\phi,\psi)
\]
and by taking limit, the above equality is also true for any $\psi\in H_0^1(M)$. Thus we prove
\[
{\mathcal Dom}\,(\Delta_2)\subset {\mathcal Dom}\,(\bar\Delta).
\]
On the other hands, since both $\bar\Delta$ and $\Delta_2$ are self-adjoint, we have
\[
{\mathcal Dom}\,(\bar\Delta)={\mathcal Dom}\,(\bar\Delta^*)\subset{\mathcal Dom}\,(\Delta_2^*)={\mathcal Dom}\,(\Delta_2),
\]
and thus we prove 
\[
{\mathcal Dom}\,(\bar\Delta)={\mathcal Dom}\,(\Delta_2).
\]
Thus $\Delta_2=\bar\Delta$, and the extension is unique. 


\end{proof}

In general, the self-adjoint extension is not unique.

\begin{example} Let $M$ be a compact orientable manifold with smooth boundary. Let $\mathcal C_0^\infty(M)$ be the space of smooth functions whose supports are within the interior of $M$, and let $\mathcal C^\infty(M)$ be the space of smooth functions (smooth up to the boundary). 
We can define two kinds of Laplacians: the first one $\Delta_D$  is called the Dirichlet Laplacian, which is the same as the Friedrichs extension $\bar\Delta$. The second one $\Delta_N$ is called the Neumann Laplacian, defined by
\begin{align*}
&
{\mathcal Dom}(\Delta_N)=\left\{\phi\in H^1(M)\mid\right.\\
&\left.\forall\psi\in \mathcal C^\infty(M) \,\,s.t. \frac{\pa\psi}{\pa n}=0, \exists f\in L^2(M), s.t.\, Q(\phi,\psi)=( f,\psi)\right\}.
\end{align*}
For $\phi,\psi\in \mathcal C^\infty(M)$, we have the Green's formula
\[
(\Delta\phi,\psi)=\int_{\pa M}\frac{\pa\phi}{\pa n}\,\psi-\int_M\phi\,\frac{\pa\psi}{\pa n}+(\phi,\Delta\psi).
\]
Therefore, if $\phi\in {\mathcal Dom}(\Delta_D)\cap\mathcal C^\infty(M)$, we have $\phi|_{\pa M}=0$, and if
$\phi\in{\mathcal Dom}(\Delta_N)\cap \mathcal C^\infty(M)$, we have $\frac{\pa\phi}{\pa n}=0$. These justify the names.

\end{example}


Many results related to the Laplace operator are true even on noncomplete manifolds. 

\begin{theorem}
Let $f\in {\rm ker}\,(\bar\Delta)$. Assume that $f\in L^2(M)$. Then $f$ is a constant. 
\end{theorem}

This is a theorem of Yau in the case of complete manifold. But we shall prove that it is also true for any noncompete manifold.


\begin{proof}
Let $\rho_j$ be the smooth function defined in Theorem~\ref{thm7}. Then we have
\[
0=(\bar\Delta f, \rho_j^2 f)=Q(f,\rho_j^2 f).
\]
Thus we have
\[
\int_M\rho_j^2|\nabla f|^2=-2\int_M \rho_j\, f\nabla\rho_j\nabla f\leq 2\sqrt{\int_M|\nabla \rho_j|^2 f^2\, dV_M}
\cdot\sqrt{\int_M\rho_j^2|\nabla f|^2\, dV_M}.
\]
It follows that 
\[
\int_M\rho_j^2|\nabla f|^2\,dV_M\leq \frac{C}{j^2}\int_Mf^2\, dV_M\to 0
\]
and the theorem is proved. 


\end{proof}

\begin{example}
Let $M=\mathbb R^n-\{0\}$. Then any $L^2$ harmonic function extends to the origin. Thus by the theorem of Yau, we know that it has to be a constant. Let $M$ be a bounded domain. Then the fact that $f$ is in the domain of the Laplace operator implies that $f$ vanishes on the boundary. Hence $f$ is zero.
\end{example}


\section{Variational characterization of spectrum}\label{Var_principle}
Let $\Delta$ be the Laplace operator on functions of a manifold $M$. As in the last section, we make the convention that the operator is a negative operator.

Let $\lambda$ be a complex number, $(\lambda I-\Delta)^{-1}$ is called the resolvent of $\Delta$. If for some $\lambda$, $(\lambda I-\Delta)^{-1}$ doesn't exist, or it does exist but is an unbounded operator, then we call $\lambda$  spectrum point of $\Delta$. We use $\sigma(\Delta)$ or ${\rm Spec}\, (\Delta)$ to denote the set.

\begin{theorem} Using the above notations, we have
\[
{\rm Spec} \,(\Delta)\neq\emptyset.
\]
\end{theorem}

\begin{proof} This is a general fact about the spectrum of any linear operator on a Banach space.
We assume that $\Delta$ is densely defined in a Banach space $\mathcal B$. Assume that 
${\rm Spec} \,(\Delta)=\emptyset$.
Let $f\in \mathcal B$ and let $\ell$ be abounded linear functional on $\mathcal B$, then
\[
\ell((\lambda I-\Delta)^{-1}f)
\]
is a bounded holomorphic function of $\C$, which has to be a constant. More over, it must be the zero function. Since $\ell$ and $f$ are arbitrary, we conclude that $(\lambda I-\Delta)^{-1}=0$ which is not possible for all $\lambda$. This is a contradiction.

\end{proof}




When $M$ is a compact manifold, the best possible things happen: by the Hodge theorem, the spectrum of $\Delta$ is made from eigenvalues. More precisely, we have
\[
0=\lambda_0<\lambda_1\leq\lambda_2\cdots\leq\lambda_k\to +\infty,
\]
such that for each $\lambda_j$, the space
\[
E_j=\{f\mid\Delta f=-\lambda_j f\}
\]
is not trivial and is of finite dimensional. By the above result, we can prove the following variational (or min-max) principal.

\begin{theorem} We have
\begin{equation} \lambda_k = \inf \left\{ \left. \frac{ \int_{\Omega} |\nabla f |^2 }{\int_{\Omega} f^2 } \, \right| \, \, \, \left. f\right|_{\pa \Omega} = 0, \, f \not\equiv 0 = \int_{\Omega} f \phi_j, \, \, \forall \, j < k \right\} , \end{equation} 
where $\varphi_j$ are the eigenfunctions of $\lambda_j$.
\end{theorem}

\begin{proof} Let $\{ f_\alpha\}$ be a sequence such that 
\[
\int_M f_\alpha\phi_j=0, \qquad \forall j<k,
\]
and
\[
\frac{\int_M|\nabla f_\alpha|^2}{\int_M f_\alpha^2}\to\lambda_k.
\]
If we normalize $f_\alpha$ such that $\int_M f_\alpha^2=1$, then the sequence $\{f_\alpha\}$ is bounded in $H_1(M)$. Therefore, there is an $f\in H_1(M)$ such that
\[
f_\alpha\to f
\]
in the weak sense. In particular, for any $\phi$ smooth, we have
\[
\int_M\nabla f_\alpha\nabla\phi\to\int_M\nabla f\nabla\phi.
\]
We have
\[
\left|\int_M\nabla f\nabla\phi\right|\leq\int_M\nabla f_\alpha\nabla\phi\leq ||\nabla f_\alpha||_{L^2}||\nabla\phi||_{L^2}\to\sqrt{\lambda_k}||\nabla\phi||_{L^2}.
\]
Since smooth functions are dense in $H_1(M)$, we have
\[
\int_M|\nabla f|^2\leq\liminf_{k\to\infty}\int_M|\nabla f_\alpha|^2,
\]
which is known as the Fatou Lemma.

On the other hand, by the Rellich Lemma, $f_\alpha\to f$ strongly in $L^2(M)$. Thus we have
\[
\int_Mf^2=1,
\]
hence
\[
\lambda_k=\frac{\int_M|\nabla f|^2}{\int_M f^2}
\]

Let $\phi$ be a smooth function such that 
\[
\int_M\phi\phi_j=0
\]
Then for any $\eps$, we must have
\[
\frac{\int_M|\nabla f+\eps\nabla\phi|^2}{\int_M (f+\eps\phi)^2}\geq\frac{\int_M|\nabla f|^2}{\int_M f^2}
\]
Since this is true for any $\eps$, we must have
\[
\Delta f=-\lambda_k f
\]
in the weak sense. By the elliptic regularity, $f$ has to be smooth.

\end{proof}



Unlike in the case of compact manifold, in general, a complete non-compact  manifold doesn't admit any pure point spectrum. For example, there are no $L^2$-eigenvalues on $\mathbb R^n$. That is, for any $\lambda\in\mathbb R$, if $\Delta f+\lambda f=0$ and $f\in L^2(\mathbb R^n)$, then we have $f\equiv 0$.



Let $\Delta$ be the Laplace operator on a complete non-compact manifold $M$.  We extend $\Delta$ naturally  to a self-adjoint densely defined operator on $L^2(M)$, which we still denote as $\Delta$ for the sake of simplicity.





The pure point spectrum of $\Delta$ are those  $\lambda\in\mathbb R$ such that 
\begin{enumerate}
\item there exists an $L^2$ function $f\neq 0$ such that 
\[
\Delta f+\lambda f=0.
\]
\item the multiplicity of $\lambda$ is finite;
\item in a neighborhood of $\lambda$, it is the only spectrum point.
\end{enumerate}

 From the above discussion, $\sigma(\Delta)$ decomposes as the union of pure point spectrum, and the so-called essential spectrum, which is, by definition, the complement of the pure point spectrum.

The set of the essential spectrum is denoted as $\sigma_{ess}(\Delta)$. Using the above definition, $\lambda\in\sigma_{ess}(\Delta)$, if either
\begin{enumerate}
\item $\lambda$ is an eigenvalue of infinite multiplicity, or
\item $\lambda$ is the limiting point of $\sigma(\Delta)$.
\end{enumerate}

The following theorems in functional analysis are well-known. 



\begin{theorem} A necessary and sufficient condition for the interval $(-\infty,\lambda)$ to intersect the essential spectrum of an self-adjoint densely defined operator $A$ is that, for all $\eps>0$, there exists an infinite dimensional subspace $G_\eps\subset \mathcal Dom(A)$, for which $(Af-\lambda f-\eps f,f)<0$.
\end{theorem}


\begin{theorem} A necessary and sufficient condition for the interval $(\lambda-a,\lambda+a)$ to intersect the essential spectrum of $A$ is that 
there exists an infinite dimensional subspace $G_\eps\subset \mathcal Dom(A)$ for which $||(A-\lambda I)f||\leq a||f||$ for all $f\in G_\eps$.
\end{theorem}






Using the above result, we give the following variational characterization of the lower bound of spectrum and  the lower bound of essential spectrum.

\begin{theorem}
Using the above notations, define
\[
\lambda_0=\inf_{f\in C_0^\infty(M)}\frac{\int_M|\nabla f|^2}{\int_M f^2},
\]
and
\[
\lambda_{ess}=\sup_K\inf_{f\in C_0^\infty(M\backslash K)}\frac{\int_M|\nabla f|^2}{\int_M f^2},
\]
where $K$ is a compact set running through an exhaustion of the manifold. Then $\lambda_0$ and $\lambda_{ess}$ are the least lower bound of $\sigma(\Delta)$ and $\sigma_{ess}(\Delta)$, respectively.
\end{theorem}

\begin{proof} We prove the formula for $\lambda_{ess}$.
The formula for $\lambda_0$ is similar.
Let 
\[
\lambda'_{ess}=\inf \sigma_{ess}(\Delta).
\]
If $\sigma_{ess}(\Delta)=\emptyset$, we define $\lambda'_{ess}=+\infty$. 
When $\lambda'_{ess}=+\infty$,  we prove that $\lambda_{ess}$ is also infinity. 

Assume not, then there is a constant $C$, such that for any compact set $K$, there is a function $f\in \mathcal C_0^\infty(M\backslash K)$, we have
\[
\frac{\int_M |\nabla f|^2}{\int_M f^2}<C.
\]
By inductively choosing $f$, we can make sure that the support of these functions are disjoint. Thus the functions span an infinite dimensional space $G$ with
\[
(-\Delta f-Cf,f)<0,
\]
which contradicts to the fact that the essential spectrum is an empty set.


The proof of the general case is similar. We fist prove that $\lambda_{ess}\leq\lambda_{ess}'$
By  definition, for $\eps>0$, 
\[
(-\infty,\lambda_{ess}'+\eps)\cap \sigma_{ess}(\Delta)\neq\emptyset.
\]
Therefore we can find infinite dimensional space $V$ such that for any $f\in V$
\[
(-\Delta f-(\lambda_{ess}'+\eps) f, f)<0.
\]
In what follows, we shall prove that, using the cut-off functions, there exist infinitely many elements in $V$ such that their supports are disjoint. 




We assume that $K$ is a compact set. Let $K'$ be a large ball containing $K$. Let $\rho$ be the cut-off function such that $\rho=1$ on $K$ but $\rho=0$ outside $K'$. We claim that  for any $\eps>0$, there is an $f\in V$ with $\int f^2=1$ but
\[
\int_M\rho^2f^2<\eps.
\]

If the above is not true, then there is an $\eps_0>0$ such that for any $f\in V$,
\[
\int_M\rho^2f^2\geq\eps_0,
\]
Since the set $f\in V$ is of infinite dimensional, the set $\rho f$ is of infinite dimensional as well, otherwise the above inequality is not valid. Thus we can find an orthogonal basis

\[
\int_M\rho^2f_if_j=0,
\]
if $i\neq j$, while we still keeping $\int_Mf_i^2=1$. We consider
\[
\int_M|\nabla(\rho f_i)|^2\leq 2\int_M|\nabla \rho|^2 f_i^2+2\int_M|\nabla f_i|^2.
\]
By $\int_M|\nabla f_i|^2\leq \lambda'_{ess}+\eps$, we have
Thus we have
\[
\frac{\int_M|\nabla(\rho f_i)|^2}{\int_M(\rho f_i)^2}
\leq \frac{2C+2(\lambda_{ess}'+\eps)}{\eps_0}
\]
for any $i$.  This is a contradiction because on the compact
set $K'$, the are only finitely many eigenvalues (counting multiplicity) below a fixed number.

Now we can prove our theorem.  For all $\eps>0$. We can find an $f$ with $\int_M f^2=1$ but 
\[
\int_M\rho^2f^2<\eps.
\]

We take $K=B(r)$ and $K'=B(R+1)$ of balls of radius $r, R+1$, respectively. Both $r,R$ are large numbers. Let $\tilde\rho$ be a cut-off function such that 
\[
\{\tilde\rho=1\}\subset B(R).
\]
We also assume that $|\nabla\tilde\rho|\leq C$. Let $\rho_1=1-\tilde\rho$. Then we have
\[
{\rm supp}\,\rho_1\cap K=\emptyset.
\]



We have
\[
\int_M|\nabla(\rho_1f)|^2=\int_M\rho_1^2|\nabla f|^2+2\int_M\rho_1f\nabla\rho_1\nabla f+\int_M f^2|\nabla \rho_1|^2.
\]
From
\[
\int_M f^2|\nabla \rho_1|^2\leq C\eps,
\]
and 
\[
2\int_M\rho_1 f\nabla \rho_1\nabla f\leq 2 \sqrt{\int_M |\nabla(\rho_1 f)|^2}\cdot \sqrt{\int_M |\nabla\rho_1|^2f^2}\leq C\sqrt\eps
\sqrt{\int_M |\nabla(\rho_1 f)|^2},
\]
we get
\[
\int_M|\nabla (\rho_1f)|^2\leq\lambda_{ess}'+C\eps.
\]
Since
\[
\int_M\rho_1^2f^2\geq 1-C\eps,
\]
we get
\[\lambda_{ess}\leq\frac{\lambda_{ess}'+C\eps}{1-C\eps}.
\]
and thus $\lambda_{ess}\leq\lambda'_{ess}$. The other direction is easier to prove.


\end{proof}

 It is particularly interesting to get the lower bound estimate for the essential spectrum because of the following theorem.

\begin{theorem} [Variational principal]
Suppose $\lambda_0<\lambda_{ess}$, then $\lambda_0$ is en eigenvalue of $M$ with finite dimensional eigenspace.
\end{theorem}


That is, there exists an $L^2$ function $f\neq 0$, such that 
\[
\Delta f=-\lambda_0 f,
\]
which is a very strong result.


\begin{example}
Let $H=-\Delta+|x|^2$ on $\mathbb R^n$. Then the set of essential spectrum is empty. As a result, all spectrum of $H$ are eingevalues of finite multiplicity.
\end{example}


\section{The new Weyl criteria}\label{Weyl_criteria}

Let $\mathfrak X$ be Hilbert space. Let $\mathscr B(\mathbb C)$ and let $\mathcal P(X)$ be the space of self-adjoint projection operators.  A spectral measure is a function
\[
E: \mathscr(\mathbb C)\to \mathcal (X)
\]
satisfying  the following properties
\begin{enumerate}
\item $E(\emptyset)=0, E(\mathbb C)=I$;
\item $E(\bigcup B_n)=\sum E(B_n)$ for disjoint Borel sets $\{B_n\}$.
\end{enumerate}

Let $f$ be a measurable function and let $x,y\in X$. Define the operator
\[
A=\int_{-\infty}^\infty f\,dE
\]
such that
\[
(Ax,y)=\int_{-\infty}^\infty f \,dE_{(x,y)}.
\]

Let~$H$ be a self-adjoint operator on a Hilbert space $X$.
The norm and inner product in $X$ are respectively
denoted by~$\|\cdot\|$ and $(\cdot,\cdot)$. Let $\sigma(H), \sigma_\mathrm{ess}(H)$ be the spectrum and the essential spectrum of $H$, respectively. Let ${\rm Dom}\,(H)$ be the domain of $H$.
The Classical Weyl criterion states that
\begin{theorem}[Classical Weyl's criterion]\label{Thm.Weyl}
A nonnegative real number~$\lambda$ belongs to $\sigma(H)$ if, and only if,
there exists a sequence $\{\psi_n\}_{n \in \mathbb N} \subset {\rm Dom}(H)$
such that
%
\begin{enumerate}
\item
$
  \forall n\in\mathbb N, \quad
  \|\psi_n\|=1
$\,,
\item
$
  (H-\lambda)\psi_n \to 0, \text{ as } n\to\infty  \text{ in }\mathcal H.$
\end{enumerate}
%
Moreover, $\lambda$ belongs to $\sigma_\mathrm{ess}(H)$ of $H$ if, and only if,
in addition to the above properties
%
\begin{enumerate}
\setcounter{enumi}{2}
\item
$
  \psi_n \to  0 \text{ weakly as  } n\to\infty \text{ in }\mathcal H.
$
\end{enumerate}
%
\end{theorem}
%



We have the following  functional analytic result, which generalizes the weak Weyl criterion. To the authors' knowledge, this result seems to be new.


\begin{theorem}[Charalambous-Lu]\label{Thm.Weyl.bis-2}
Let $f$ be a  bounded positive continuous function over $[0,\infty)$.
A nonnegative real number  $\lambda$ belongs to the spectrum $\sigma(H)$ if, and only if,
there exists a sequence $\{\psi_n\}_{n \in \mathbb N} \subset {\rm Dom}(H)$
such that
%
\begin{enumerate}
\item
$
  \forall n\in\mathbb N, \quad
  \|\psi_n\|=1
$\,,
\item
$
 (f(H) (H-\lambda)\psi_n, (H-\lambda)\psi_n)\to 0, \text{ as } n\to\infty \quad {and}
$
\item
$
(\psi_n, (H-\lambda)\psi_n) \to  0, \text{ as } n\to\infty.
$

\end{enumerate}
%
Moreover, $\lambda$ belongs to $\sigma_\mathrm{ess}(H)$ of $H$ if, and only if,
in addition to the above properties
%
\begin{enumerate}
\setcounter{enumi}{3}
\item
$
  \psi_n \to 0, \text{ weakly as } n\to\infty
$
\text{ in } $\mathcal H$.
\end{enumerate}%
%
\end{theorem}


\begin{proof} Since $H$ is a densely defined self-adjoint operator, the spectral measure $E$ exists and we can write
\begin{equation}\label{decomp}
H=\int_0^\infty \lambda\, dE.
\end{equation}

\noindent Assume that $\lambda\in\sigma(H)$. Then by Weyl's criterion, there exists a sequence $\{\psi_n\}$ such that
\[
\|(H-\lambda)\psi_n\|\to 0, \quad \|\psi_n\|=1
\]
as $n\to\infty$.

\noindent We write
\[
\psi_n=\int_0^\infty d E(t)\psi_n
\]
as its spectral decomposition. Then
\[
(f(H) (H-\lambda)\psi_n, (H-\lambda)\psi_n)=\int_0^\infty f(t)(t-\lambda)^2 d\|E(t)\psi_n\|^2.
\]
Since $f$ is a bounded positive function, we have
\[
(f(H) (H-\lambda)\psi_n, (H-\lambda)\psi_n)\leq C\int_0^\infty (t-\lambda)^2 d\|E(t)\psi_n\|^2=C\|(H-\lambda)\psi_n\|^2.
\]
Moreover,
\[
(\psi_n, (H-\lambda)\psi_n)\leq C\,\|\psi_n\|\cdot\|(H-\lambda)\psi_n\|.
\]
Thus the necessary part of the theorem is proved.

Now assume that $\lambda>0$ and $\lambda\notin\sigma(H)$. Then there is a $\lambda>\eps>0$ such that $E(\lambda+\eps)-E(\lambda-\eps)=0$. We write
\begin{equation}\label{fgh}
\psi_n=\psi_n^1+\psi_n^2,
\end{equation}
where
\[
\psi_n^1=\int_0^{\lambda-\eps} dE(t)\psi_n,
\]
and $\psi_n^2=\psi_n-\psi_n^1$.

\noindent Then
\begin{align*}&
(f(H) (H-\lambda)\psi_n, (H-\lambda)\psi_n) \\
&=(f(H) (H-\lambda)\psi_n^1, (H-\lambda)\psi_n^1)
+(f(H) (H-\lambda)\psi^2_n, (H-\lambda)\psi_n^2)\\
& \geq c_1\|\psi_n^1\|^2+(f(H) (H-\lambda)\psi^2_n, (H-\lambda)\psi_n^2)\geq c_1\|\psi_n^1\|^2,
\end{align*}
where the positive number $c_1$ is the infimum of the function $f(t)(t-\lambda)^2$ on $[0,\lambda-\eps]$.
Therefore
\[
\|\psi_n^1\|\to 0
\]
by {\it (2)}. On the other hand, we similarly get
\[
(\psi_n, (H-\lambda)\psi_n)\geq \eps\|\psi_n^2\|^2-\lambda\|\psi_n^1\|^2.
\]
If the criteria {\it (2), (3)} are satisfied, then, by the two {estimates} above, we conclude that both $\psi_n^1, \psi_n^2$ go to zero. This   contradicts $\|\psi_n\|=1$, and the theorem is proved.

\noindent Note that for $\lambda=0$, $\psi_n^1$ is automatically zero, and the second half of the proof would give the contradiction.

\end{proof}

\begin{theorem}[Charalambous-Lu]
Let $\sigma_p(M)$ be the spectrum of the Laplacian on $p$-forms. If $\lambda\in\sigma_p(\Delta)$ and $\lambda\neq 0$, then $\lambda\in\sigma_{p-1}(\Delta)$ or $\lambda \in\sigma_{p+1}(\Delta)$.
\end{theorem}

\begin{remark} If $M$ is a compact manifold, then the result is trivially true: if $f\neq 0$ is the eigenform  of the eigenvalue $\lambda$.
\[
\Delta f-\lambda f=0.
\]
Since $d\Delta=\Delta d$, $\delta\Delta=\Delta\delta$, we have
\[
\Delta df-\lambda df=0, \quad \Delta\delta f-\lambda \delta f=0.
\]
Since $\lambda\neq 0$, we have either $df\neq 0$, or $\delta f\neq 0$. The result follows. 
\end{remark}

\begin{proof}
If $0\neq\lambda\in\sigma_p(\Delta)$, then for any $\eps>0$, there exists $f\neq 0$ smooth with compact support, such that 
\[
\|\Delta f-\lambda f\|_{L^2}\leq\eps\|f\|_{L^2}.
\]
We then have 
\[
\left|\|df\|^2_{L^2}+\|\delta f\|^2_{L^2}-\lambda\|f\|^2_{L^2}\right|=|(f,\Delta f-\lambda f)|\leq\eps \|f\|_{L^2}^2.
\]
In particular, if $\eps$ is small enough, we must have either 
\[
\|df\|^2_{L^2}\geq\frac\lambda 4\|f\|^2_{L^2},
\]
or
\[
\|\delta f\|^2_{L^2}\geq\frac\lambda 4\|f\|^2_{L^2}.
\]
For the rest of the proof, we assume that the former is correct. Using our new Weyl Criterion, we need to prove 
\[
(df, (\Delta-\lambda) df)|\leq\eps\|f\|^2_{L^2}.
\]
But this follows from 
\[
|(\delta df, (\Delta-\lambda)f)|\leq\|(\Delta-\lambda)f\|_{L^2}\cdot\|\delta df\|_{L^2},
\]
and the fact that 
\[
\int_M|\Delta f|^2=\int_M |df|^2+\int_M|\delta f|^2\geq \int_M | df|^2.
\]
Similarly, we can prove that for function $g(t)=(t+1)^{-1}$, we have
\[
|(g(\Delta)(\Delta-\lambda) df, (\Delta-\lambda) df)|\leq\eps\|df\|^2_{L^2}.
\]


\end{proof}

\begin{corollary}
Let $X$ be a K\"ahler manifold and let $L\to X$ be a positive line bundle. Let $\Delta_0$, $\Delta_1$ be the Laplacians  of $L^m$-valued sections and $(0,1)$-forms, respectively. We assume that both of them are self-adjoint. Then 
\[
\sigma_0(\Delta_0)\subset \{0\}\cup\sigma_1(\Delta_1).
\]
Moreover, if $0\in\sigma_0(\Delta_0)$, then holomorphic sections exists. 
\end{corollary}



\section{Poincar\'e inequality and Sobolev inequality}\label{Inequalities}
Let $M$ be a compact manifold without boundary (we call such a manifold {\it closed}). Then by the Hodge theorem, for any function $f$ such that 
\[
\int_M f=0,
\]
we have
\[
\int_M|\nabla f|^2\geq C\int_M f^2
\]
for a positive constant $C$. For manifold with boundary, we have similar versions of Poincar\'e inequalities with respect to the boundary conditions.

 The other fundamental inequality is the Sobolev inequality
 
 {\bf Sobolev inequality.} Assume that $M$ is a compact manifold with boundary, then there is a constant $C$ such that
 \begin{equation}\label{sobo}
 C\left(\int_Mf^{\frac{n}{n-1}}\right)^{\frac{n-1}n}\leq\int_M|\nabla f|
 \end{equation}
 for any smooth functions satisfies the Dirichlet or Neumann boundary conditions.
 
 
The Soboleve inequality is equivalent to the so-called isoperimetric inequality:

{\bf Isoperimetric inequality.} Let $\Omega$ be a domain of $M$ which is relatively compact. Then
there is a constant $C$ such that
\begin{equation}\label{iso}
C({\rm Vol}\,(\Omega))^{\frac{n-1}{n}}\leq{\rm Vol}\,(\pa\Omega).
\end{equation}

To prove the equivalence,
we first assume the Sobolev inequality~\eqref{sobo}. We take the function
\[
f_\eps(x)=\left\{
\begin{array}{ll}
1,& x\in\Omega, d(x,\pa\Omega)\geq\eps,\\
\frac{d(x,\pa\Omega)}{\eps}, & x\in\Omega, d(x,\pa\Omega)\leq\eps,\\
0& otherwise
\end{array}
\right.
\]
Using the Soboleve inequality on $f_\eps$ and lettting $\eps\to 0$, we get the isoperimetric inequality. On the other hand, we have the following co-area formula

\begin{theorem}[Co-area formula] Let $M$ be a compact manifold with boundary. $f\in H^1(M)$. Then for any nonnegative function $g$ on $M$, we have
\[
\int_M g=\int_{-\infty}^\infty\left(\int_{\{f=\sigma\}}\frac{g}{|\nabla f|}\right)d\sigma.
\]
\end{theorem}

\begin{ex} Proof the above co-area formula.
\end{ex}

For the sake of simplicity, we assume that $f\geq 0$ and we assume that the isoperimetric inequality~\eqref{iso} is valid. By the co-area formula, we have
\[
\int_M|\nabla f|=\int_0^\infty{\rm Area}\,(f=\sigma)d\sigma.
\]
At the same time we have
\[
\int_M|f|^{\frac{n}{n-1}}=\int_0^\infty{\rm Vol}\,(f^{\frac{n}{n-1}}>\lambda)d\lambda
=\frac{n}{n-1}\int_0^\infty{\rm Vol}\,(f>\sigma)\sigma^{\frac{n}{n-1}}d\sigma.
\]
Using the isoperimetric inequality, we have
\[
\int_M|\nabla f|\geq C\int_0^\infty{\rm Vol}\,(f>\sigma)^{\frac{n-1}{n}}d\sigma.
\]
Therefore, we just need to prove that 
\[
\int_0^\infty{\rm Vol}\,(f>\sigma)^{\frac{n-1}{n}}d\sigma\geq C\left(\int_0^\infty{\rm Vol}\,(f>\sigma)\sigma^{\frac{1}{n-1}}d\sigma\right)^{\frac{n-1}{n}}.
\]

Let 
\begin{align*}
& F(\sigma)={\rm Vol}\,(f>\sigma),\\
&\phi(t)=\int^t_0F(\sigma)^{\frac{n-1}{n}}d\sigma,\\
&\psi(t)=\left(\int_0^t F(\sigma)\sigma^{\frac{1}{n-1}}d\sigma\right)^{\frac{n-1}{n}}.
\end{align*}
Then $\phi(0)=\psi(0)$. It is not hard to see that $\phi'(t)\geq \frac{n}{n-1}\psi'(t)$. Thus $\phi(\infty)\geq\frac{n}{n-1}\psi(\infty)$.

\qed