


\chapter{Semigroups and heat kernels}

\section{Semigroups and their generators}

In this section, we define  abstract semigroup and prove some of  the  basic properties. 

\begin{definition}
A one-parameter semigroup on a complex Banach space $\mathcal B$ is a  family $T_t$ of bounded linear operators, where $T_t: \mathcal B\to\mathcal B$ parameterized by real  numbers $t\geq 0$ and satisfies the following relations:
\begin{enumerate}
\item $T_0=1$;
\item If  \,$0\leq s,t<\infty$, then
\[
T_sT_t=T_{s+t}.
\]
\item The map
\[
(t,f)\to T_tf
\]
from $[0,\infty)\times\mathcal B$ to $\mathcal B$ is jointly continuous.
\end{enumerate}
\end{definition}



The (infinitesimal) generator $Z$ of a one-parmeter semigroup $T_t$ is defined by
\[
Zf=\limt t^{-1}(T_tf-f).
\]
The domain ${\mathcal Dom}(Z)$ of $Z$ being the set of $f$ for which the limit exists. It is evident that  ${\mathcal Dom}(Z)$  is a linear space. Moreover, we have

\begin{lemma}
The subspace  ${\mathcal Dom}(Z)$ is dense in $\mathcal B$, and is invariant under $T_t$ in the sense that
\[
T_t( {\mathcal Dom}(Z) )\subset  {\mathcal Dom}(Z)
\]
for all $t\geq 0$. Moreover
\[
T_tZf=ZT_t f
\]
for all $f\in  {\mathcal Dom}(Z)$ and $t\geq 0$.
\end{lemma}


{\bf Proof.} If $f\in\mathcal B$, we define
\[
f_t=\int_0^t T_xf\, dx.
\]
The above integration  exists in the following sense: 
since $T_xf$ is a continuous function of $x$, we can define
the integration as the limit of the corresponding Riemann sums.
In a Banach space, absolute convergence implies conditional convergence. Thus in order to prove the convergence of the Riemann sums, we only need to verify  that 
\[
\int_0^t||T_xf||\, dx
\]
is convergent. But this follows easily from the joint continuity in the definition of the semigroup: $||T_xf||$ must be uniformly bounded for small $x$.

We compute
\begin{align*}
& \limh h^{-1} (T_hf_t-f_t)\\
&=\limh\left\{ h^{-1}\int_h^{t+h}T_xf dx-h^{-1}\int_0^t T_x f dx\right\}\\
&=\limh \left\{ h^{-1}\int_t^{t+h}T_x f dx-h^{-1}\int_0^h T_x f dx\right\}\\
&=T_t f-f.
\end{align*}
Therefore, $f_t\in{\mathcal Dom}(Z)$ and 
\[
Z(f_t)=T_t f-f.
\]
Since $t^{-1} f_t\to f$ in norm as $t\to 0^+$, we see that ${\mathcal Dom}(Z)$ is dense in $\mathcal B$.


\qed

The generator $Z$, in general, is not a bounded operator. However, we can prove the following





\begin{lemma} The generator $Z$ is a closed operator.
\end{lemma}

{\bf Proof.}
We first observe that 
\[
T_tf-f=\int_0^t T_x Zf dx
\]
for  $f\in{\mathcal Dom}(Z)$. To see this, we consider the function $r(t)=T_tf-f-\int_0^t T_x Zf dx$. Obviously we have $r(0)=0$ and $r'(t)\equiv 0$. Thus $r(t)\equiv 0$.

Let $\{f_n\}$ be a sequence in the domain of $Z$ such that $f_n\to f$ and $Zf_n\to g$.
Using the above formula, we have
\[
T_t f-f=\lim_{n\to \infty} (T_t f_n -f_n)
=\lim_{n\to \infty}\int_0^t T_x Zf_n dx.
\]
By the Lebesgue dominated convergence theorem, the above limit is equal to 
\[
\int_0^t T_x g dx.
\]
Thus we have
\[
\limt t^{-1}(T_t f-f)=g,
\]
and therefore, $f\in{\mathcal Dom}(Z), Zf=g$.

\qed


\begin{remark}
If $Z$ is not a bounded operator on ${\mathcal Dom}(Z)$, it is not possible to extend $Z$ to the whole banach space $\mathcal B$ because otherwise since $Z$ is closed, $Z$ has to be bounded by the closed graph theorem.
\end{remark}


\begin{theorem}
Let $\mathfrak B$ be a Hilbert space. If the operators $T_t$ are self-adjoint operators, then $Z$ is a densely defined self-adjoint operator. 
\end{theorem}



\section{Heat kernel}
In this section, we construct the heat kernels. That is, we are going to find smooth function $H(x,y,t)$ such that
\[
T_t(f)(x)=\int_M H(x,y,t) f(y) dy.
\]

For the sake of simplicity, we shall only consider the Laplace operator on functions. Moreover, the sign convention is that on the Euclidean space, $\Delta=\sum\frac{\pa^2}{\pa x_j^2}$.
The semigroup is formally defined as $T_t=e^{\Delta t}$.

The main result of this section is the following



\begin{theorem}
Let $M$ be a  Riemannian manifold, then there is a heat kernel
\[
H(x,y,t)\in C^\infty(M\times M\times\mathbb R^+),
\]
such that 
\[
(T_t f)(x)=\int_M H(x,y,t) f(y)dy
\]
satisfying
\begin{enumerate}
\item $H(x,y,t)=H(y,x,t)$;
\item $\underset{t\to 0^+}{\lim}\, H(x,y,t)=\delta_x(y)$;
\item $(\Delta-\frac{\pa}{\pa t})H=0$;
\item $H(x,y,t)=\int_M H(x,z,t-s)H(z,y,s) dz$ for any $0<s\leq t$.
\end{enumerate}
\end{theorem}


Before proving the theorem, we first {\it formally} construct  the heat kernel. This formal construction also outlines the proof of the theorem.



The $k$-simplex $\Delta_k$ is the following subset of $\mathbb R^k$
\[
\{(t_1,\cdots,t_k)\mid 0\leq t_1\leq\cdots\leq t_k\leq 1\}.
\]
For $t>0$, we write $t\Delta_k$ for the rescaled simplex 
\[
\{(t_1,\cdots,t_k)\mid 0\leq t_1\leq\cdots\leq t_k\leq t\}.
\]

We assume that $U(x,y,t)$ be a function on $M\times M\times\mathbb R^+$ such that 
\[
\limt U(x,y,t)=\delta_x(y)
\]
 be the the Dirac function. Let
\[
R(x,y,t)=\frac{d U(x,y,t)}{dt}-\Delta_x U(x,y,t),
\]
where $\Delta_x$ means the Laplace  operator for the variable $x$. Note that formally, any function $g(x,y,t)$ defines one-parameter family of operators $G_t$ by the formula
\[
G_t f(x)=\int_M g(x,y,t) f(y) dy.
\]
We use the $R_t, U_t$ to denote  the corresponding families of operators with respect to the functions $R(x,y,t)$ and $U(x,y,t)$, respectively. For any $k\geq 1$,
define the operator
\[
Q_t^k=\int_{t\Delta_k} U_{t-t_k}R_{t_k-t_{k-1}}\cdots R_{t_2-t_1}R_{t_1} dt_1\cdots dt_k,
\]
and $Q_t^0=U_t$. Let
\[
R^{(k)} (s)=\int_{s\Delta_{k-1}}R_{s-t_{k-1}}\cdots R_{t_2-t_1} R_{t_1} dt_1\cdots dt_{k-1},
\]
and $R^{(0)}(s)=0$.
Since the derivative of the integral of the form $$\int_0^t a(t-s) b(s) ds$$ is equal to 
\[
\int_0^t\frac{da}{dt}(t-s) b(s) ds+a(0) b(t),
\]
we have
\[
\left(\frac{\pa}{\pa t}-\Delta\right )Q_t^k=R^{(k+1)} (t)+R^{(k)} (t).
\]
As a result, we have
\[
\left(\frac{\pa}{\pa t}-\Delta\right )\sum_{k=0}^\infty (-1)^k Q_t^k=0.
\]

\medskip

\medskip

For $\mathbb R^n$, the fundamental solution of the heat equation
\[
(\Delta-\frac{\pa}{\pa t}) u=0
\]
is 
\[
\frac{1}{(4\pi t)^{n/2}}e^{-\frac{r^2}{4t}},
\]
where $r=d(x,y)$ is the Euclidean distance of $x$ and $y$.

We wish to find the the following form of the  fundamental solution of the heat equation:
\begin{equation}\label{formal}
U(x,y,t)\sim (4\pi t)^{-\frac n2} e^{-d^2(x,y)/4t}\left\{\sum_{i\geq 0}\phi_i(x,y) t^i\right\}
\end{equation}
where $d(x,y)$ is the distance function on the Riemannian manifold. $U(x,y,t)$ should satisfy 
\begin{enumerate}
\item  $\limt U(x,y,t)=\delta_x(y)$, where $\delta_x(y)$ is the Dirac  function at $x$;
\item For any $N$, $\limt (\Delta-\frac{\pa}{\pa t})U(x,y,t)=O(t^N)$.
\end{enumerate}

The function $U(x,y,t)$ is called the {\it paramatrix} of the heat kernel.

We pick a normal coordinate system 
$(y_1,\cdots,y_n)$, and let $r=d(x,y)$ be the Riemannian distance. 
We identify a neighborhood of $y$ to a small ball of $T_y(M)$ by the exponential map. Under this map, the coordinates of $x$ can be written as $(x_1,\cdots,x_n)$. On the other hand, let $(\theta_1,\cdots,\theta_{n-1})$ be a coordinate system on $S^{n-1}$, then $(r,\theta_1,\cdots,\theta_{n-1})$ gives a coordinate system at $y$ also, and this coordinate system is called the polar coordinates.


\begin{ex} Let $ds^2$ be the Riemannian metric. Then we can write
\[
ds^2=dr^2+\sum_{i,j=1}^{n-1} r^2s_{ij}(x) d\theta_i d\theta_j.
\]
That is, prove that $\frac{\pa}{\pa r}$ is orthogonal to any $\frac{\pa}{\pa\theta_j}$.
\end{ex}

Let $\psi(r)$ be a function of $r$, and let $g=\det (s_{ij})$. Then we have
\begin{align*}
& \Delta\psi=\frac{d^2\psi}{dr^2}+\frac nr\cdot\frac{\pa\psi}{\pa r}+\left(\frac{d\log \sqrt g}{dr}\right)\frac{d\psi}{dr},\\
&\Delta(\phi\psi)=\phi\Delta\psi+\psi\Delta\phi+2\frac{d\phi}{dr}\frac{d\psi}{dr}.
\end{align*}

We let
\begin{align*}
& \psi=\frac{1}{(4\pi t)^{n/2}}e^{-\frac{r^2}{4t}},\\
&\phi=\phi_0+\phi_1 t+\cdots+\phi_Nt^N,
\end{align*}
where $\phi_j=\phi_j(x,y)$ are smooth local  functions on $M\times M$. We assume that under our coordinate system, $y$ is the origin\footnote{We use the obvious fact that there exists a smooth family of normal coordinate systems parametrized by any point of $M$. So all the functions we define below are smooth not only with respect to $x$ but $y$.}. Then
\[
u_N=\psi\phi=\frac{1}{(4\pi t)^{n/2}}e^{-\frac{r^2}{4t}}
\sum_{i=0}^N\phi_i t^i.
\]
Therefore
\[
\left(\Delta-\frac{\pa}{\pa t}\right) u_N=\phi\left(\Delta\psi-\frac{\pa\psi}{\pa t}\right)+\psi\left(\Delta\phi-\frac{\pa\phi}{\pa t}\right)+2\frac{\pa\phi}{\pa r}\frac{d\psi}{dr}.
\]
Since 
\begin{align*}
&\Delta\psi-\frac{\pa\psi}{\pa t}=\frac{d\log\sqrt g}{dr}\frac{d\psi}{dr},\\
&\frac{d\psi}{dr}=-\frac{r}{2t}\psi.
\end{align*}
we have
\begin{equation}\label{17}
\left(\Delta-\frac{\pa}{\pa t}\right)u_N=\frac{\psi}{t}\sum_{k=0}^N\left[
\Delta\phi_{k-1}-\left(k+\frac r2\frac{d\log\sqrt g}{dr}\right)\phi_k-r\frac{d\phi_k}{dr}\right] t^k.
\end{equation}
Thus in order to find the paramatrix, we set
\[
r\frac{d\phi_k}{dr}+\left(k+\frac r2\frac{d\log\sqrt g}{dr}\right)\phi_k=\Delta\phi_{k-1}
\]
for $k=0,\cdots,N$, where we let $\phi_{-1}=0$. The solutions of the above ordinary differential equations are
\begin{align}\label{phi-1}
\begin{split}
& \phi_0(x,y)=g^{-\frac 14}(x);\\
&\phi_k(x,y)=g^{-\frac 14}(x)r(x,y)^{-k}\int_0^{r(x,y)} r^{k-1}(\Delta\phi_{k-1})\left(\frac{rx}{r(x,y)}\right) g\left(\frac{rx}{r(x,y)}\right)^{\frac 14}dr.
\end{split} 
\end{align}
Thus we have
\[
\left(\Delta-\frac{\pa}{\pa t}\right)u_N=\frac{\psi}{t}(\Delta\phi_N)t^N.
\]

From the above, we prove that
\begin{lemma}
There is a unique formal solution $U(x,y,t)$ of the heat equation
\[
\left(\Delta-\frac{\pa}{\pa t}\right)U(x,y,t)=0
\]
of the form ~\eqref{formal} such that $\phi_i(x,y)$ are defined in ~\eqref{phi-1}.
\end{lemma}

\qed

 
 Let $\eta$ be a smooth function such that $\eta=1$ for $t<1$ and $\eta=0$ for $t>2$. Let
 \[
 p(x,y)=\eta\left(\frac{2r(x,y)}{\delta}\right)
 \]
 be the cut-off function, where $\delta$ is the injectivity  radius at $y$. Then for any $N$, we consider the function $u_N(x,y,t)=p(x,y)u_N(x,y,t)$. We shall prove that 
 
 \begin{lemma}
 For any $N$ sufficiently large, we have
 \begin{enumerate}
 \item $\limt u_N(x,y,t)=\delta_x(y)$;
 \item The kernel $R_N(x,y,t)=\left(\Delta_y-\frac{\pa}{\pa t}\right)u_N(x,y,t)$ satisfies the estimate
 \[
 ||R_N(x,y,t)||_{\mathcal  C^l}\leq C(l)t^{N-l/2-1}.
 \]
 \end{enumerate}
 \end{lemma}
 
 \qed
 
 Using ~\eqref{17} and the solutions of $\phi_k$, we know that for $N\gg 0$, we have
 \[
  ||R_N(x,y,t)||_{\mathcal  C^l}\leq Ct^\alpha.
  \]
  
  It follows that
  \[
  ||R^{(k)}(s)||_{\mathcal  C^l}\leq\frac{s^{k\alpha}(\alpha!)^k}{(\alpha k)!}.
  \]
  
  Since
  \[
  \sum_k\frac{s^{k\alpha}(\alpha!)^k}{(\alpha k)!}<+\infty,
  \]
 our formal construction is convergent to the heat kernel.
 
 \begin{theorem} The $\Delta_p$ on $L^p(M)$ is well defined as the infinitesimal generator of the heat semi-group.
 \end{theorem}
 
 \qed
 
For any self-adjoint extension $\tilde\Delta$ of $\Delta$, the corresponding heat kernel is
 \begin{theorem}
 Using the above notations, we have
 \[
\tilde  H(x,y,t)=U(x,y,t)-\int_0^te^{\tilde \Delta(t-s)}\left(\frac{\pa}{\pa t}-\Delta_x\right) U(x,y,s) ds.
 \]
 \end{theorem}
 
 
\begin{ex} Construct heat kernels of the Laplacian of  bundle-valued $p$-forms. \end{ex}
 

